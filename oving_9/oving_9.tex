
\begin{oppgave}[5.6.5]
    La $\triangle ABC$ være en trekant, og la $m_A$ være midtnormalen til linjen motsatt av hjørnet $A$. 
    Vi definerer $m_B$ og $m_C$ tilsvarende. 
    Vi ønsker å vise at disse tre linjene skjærer hverandre i et felles punkt. 

    Vi viser først at $m_A$ og $m_B$ skjærer hverandre, altså at de ikke er paralelle. 
    Vi antar dermed at $m_A\parallel m_B$ og prøver å konstruere en selvmotsigelse. 
    Fra del 3 av teorem 4.7.3  må da enten $\overleftrightarrow{AC}=\overleftrightarrow{BC}$ eller $\overleftrightarrow{AC}\parallel \overleftrightarrow{BC}$, noe som er umulig, siden $\overline{AC}$ og $\overline{BC}$ er sider i en trekant. 
    Altså må de skjære hverandre. 
    Kall dette skjæringspunktet $P$. 

    Av punktvis karakterisering av midtnormaler (teorem 4.3.7) er $AP=CP$, siden $P$ ligger på $m_B$, og $BP=CP$ siden $P$ ligger på $m_A$. 
    Sammen gir disse likhetene at $AP=BP$, som fra punktvis karakterisering av midtnormaler betyr at $P\in m_C$, altså skjærer alle midtnormalene i punktet $P$. 

    Legg merke til at dette også beviser at omsenteret ligger like langt fra alle hjørnene, ettersom vi har vist $AP=BP=CP$. 

    \begin{figure}[H]
        \centering
        
\definecolor{qqqqff}{rgb}{0,0,1}
\begin{tikzpicture}[line cap=round,line join=round,>=triangle 45,x=1cm,y=1cm]
\clip(-6,-2) rectangle (5,5);
\draw [line width=2pt] (-4,0)-- (1,4);
\draw [line width=2pt] (1,4)-- (2,-1);
\draw [line width=2pt] (2,-1)-- (-4,0);
\draw [line width=2pt,domain=-14.220165289256189:13.416198347107425] plot(\x,{(-0.5--5*\x)/-4});
\draw [line width=2pt,domain=-14.220165289256189:13.416198347107425] plot(\x,{(--6--1*\x)/5});
\draw [line width=2pt,domain=-14.220165289256189:13.416198347107425] plot(\x,{(-5.5-6*\x)/-1});
\draw [line width=1.2pt] (-4,0)-- (-0.7413793103448276,1.0517241379310345);
\draw [line width=1.2pt] (-2.3986123728637536,0.6123767188616307) -- (-2.342766937481075,0.4393474190694038);
\draw [line width=1.2pt] (-0.7413793103448276,1.0517241379310345)-- (2,-1);
\draw [line width=1.2pt] (0.683782531799213,0.09864423475113619) -- (0.5748381578559573,-0.046920096820102425);
\draw [line width=1.2pt] (-0.7413793103448276,1.0517241379310345)-- (1,4);
\draw [line width=1.2pt] (0.05103518539776811,2.5720947654708484) -- (0.20758550425740296,2.4796293724601868);
\draw [fill=qqqqff] (-4,0) circle (2.5pt);
\draw[color=qqqqff] (-4.274710743801652,-0.43545454545454465) node {$A$};
\draw [fill=qqqqff] (1,4) circle (2.5pt);
\draw[color=qqqqff] (1.2889256198347074,4.3827272727272675) node {$C$};
\draw [fill=qqqqff] (2,-1) circle (2.5pt);
\draw[color=qqqqff] (2.2889256198347065,-0.6172727272727262) node {$B$};
\draw[color=black] (-4,4.5) node {$m_B$};
\draw[color=black] (-5,0.5) node {$m_A$};
\draw[color=black] (-1.7,-1.5) node {$m_C$};
\draw [fill=qqqqff] (-0.7413793103448276,1.0517241379310345) circle (2pt);
\draw[color=qqqqff] (-0.4565289256198366,1.400909090909089) node {$P$};
\end{tikzpicture} 
    \end{figure}
\end{oppgave}

\begin{oppgave}[5.6.11]
    Vi skal vise at en trekant $\triangle ABC$ er likesidet hvis og bare hvis omsenteret og sentroiden sammenfaller. 
    Vi minner oss selv om at sentroiden er definert som skjæringspunktet til de tre medianene i trekanten, og at omsenteret er definer som skjæringspunktet til de tre midtnormalene til sidene i trekanten. 

    Anta at $\triangle ABC$ er likesidet. 
    For å vise at sentroiden og omsenteret sammenfaller, er det nok å vise at medianene og midtnormalene til trekanten sammenfaller -- da har vi vist at linjene som definerer de to punktene er like. 
    La $M$ være midtpunktet på $\overline{AB}$. 
    Vi viser at medianen $\overleftrightarrow{CM}$ er midtnormalen til $\overline{AB}$. 
    Beviset for de andre medianene og midtnormalene er identisk. 

    Vi vet at $AC=BC$ siden $\triangle ABC$ er likesidet. 
    Dermed sier teorem 4.3.7 at $C$ ligger på midtnormalen til $\overline{AB}$. 
    Men, da vet vi at $M$ og $C$ både ligger på medianen gjennom $C$ og på midtnormalen til $\overline{AB}$. 
    Siden to punkt bestemmer en unik linje, må derfor medianen og midtnormalen sammenfalle. 

    \begin{figure}[H]
        \centering
        
\definecolor{qqqqff}{rgb}{0,0,1}
\begin{tikzpicture}[line cap=round,line join=round,>=triangle 45,x=1cm,y=1cm]
\clip(-4,-1) rectangle (4,6);
\draw [line width=2pt] (-3,0)-- (0,5);
\draw [line width=2pt] (0,5)-- (3,0);
\draw [line width=2pt] (3,0)-- (0,0);
\draw [line width=2pt] (0,0)-- (-3,0);
\draw [line width=2pt] (0,-8.9) -- (0,8.88);
\draw [fill=qqqqff] (-3,0) circle (2.5pt);
\draw[color=qqqqff] (-3.3,0.43) node {$A$};
\draw [fill=qqqqff] (3,0) circle (2.5pt);
\draw[color=qqqqff] (3.32,0.43) node {$B$};
\draw [fill=qqqqff] (0,5) circle (2.5pt);
\draw[color=qqqqff] (0.32,5.43) node {$C$};
\draw [fill=qqqqff] (0,0) circle (2.5pt);
\draw[color=qqqqff] (0.32,0.43) node {$M$};
\end{tikzpicture} 
    \end{figure}

    Anto nå at omsenteret og sentroiden sammenfaller, kall dette punktet $P$. 
    Dette må bety at medianene og midtnormalene til trekanten sammenfaller. 
    Betrakt for eksempel medianen gjennom $C$ og midtnormalen til $\overline{AB}$. 
    Per antagelse ligger $P$ både på medianen gjennom $C$ og midtnormalen til $\overline{AB}$, og per definisjon ligger også midtpunktet til $\overline{AB}$, som vi kaller $M$, på begge disse linjene. 
    Siden de to punktene $M$ og $P$ bestemmer en unik linje, må disse linjene sammenfalle. 
    Vi viser at $AC=BC$, beviset for at $AB=BC$ er identisk. 
    Vi vet at $\overline{CM}\perp \overline{AB}$, ettersom $\overleftrightarrow{CM}$ er medianen gjennom $C$, som vi vet sammenfaller med midtnormalen til $\overline{AB}$. 
    Vi betrakter de rettvinklede trekantene $\triangle AMC$ og $\triangle BMC$. 
    Linjestykket $\overline{MC}$ er felles i de to trekantene, og $AM=BM$ siden $M$ er midtpunktet. 
    Dermed gir SVS at $\triangle AMC\cong \triangle MBC$, som spesielt betyr at $AC=BC$. 
    Dermed er trekanten likesidet, som var det vi ville vise. 

    \begin{figure}[H]
        \centering
        
\definecolor{qqwuqq}{rgb}{0,0.39215686274509803,0}
\definecolor{qqqqff}{rgb}{0,0,1}
\begin{tikzpicture}[line cap=round,line join=round,>=triangle 45,x=1cm,y=1cm]
\clip(-4,-1) rectangle (4,6);
\draw[line width=2pt,color=qqwuqq,fill=qqwuqq,fill opacity=0.10000000149011612] (0.42426406871192873,0) -- (0.42426406871192884,0.4242640687119287) -- (0,0.42426406871192873) -- (0,0) -- cycle;  
\draw [line width=2pt] (-3,0)-- (0,5);
\draw [line width=2pt] (0,5)-- (3,0);
\draw [line width=2pt] (3,0)-- (0,0);
\draw [line width=2pt] (1.5,-0.12) -- (1.5,0.12);
\draw [line width=2pt] (0,0)-- (-3,0);
\draw [line width=2pt] (-1.5,-0.12) -- (-1.5,0.12);
\draw [line width=2pt] (0,-8.9) -- (0,8.88);
\draw [fill=qqqqff] (-3,0) circle (2.5pt);
\draw[color=qqqqff] (-3.3,0.43) node {$A$};
\draw [fill=qqqqff] (3,0) circle (2.5pt);
\draw[color=qqqqff] (3.32,0.43) node {$B$};
\draw [fill=qqqqff] (0,5) circle (2.5pt);
\draw[color=qqqqff] (0.32,5.43) node {$C$};
\draw [fill=qqqqff] (0,0) circle (2.5pt);
\draw[color=qqqqff] (-0.35,0.43) node {$M$};
\end{tikzpicture} 
    \end{figure}
\end{oppgave}

\begin{oppgave}[5.6.14]
    La $A$ og $B$ være to ulike punkter. 
    \begin{punkt}
        Vi skal vise at for alle reelle tall $x\neq -1$ finnes et unikt punkt $X$ på $\overleftrightarrow{AB}$ slik at $AX/XB=x$.
        Denne brøken tolkes som en ``sensed ratio'', altså at $AX/XB$ er positiv hvis $A\ast X\ast B$ og negativ ellers. 
        Vi deler beviset inn i fire deler: $x=0$, $x>0$, $-1<x<0$ og $x<-1$. 
        Merk at vi bruker notasjonen $AX/XB$ når vi snakker om brøken som en ``sensed ration'', og $\frac{AX}{XB}$ når vi snakker om brøken som en vanlig brøk. 
        Først bruker vi teorem 3.2.16 til å finne en koordinatfunksjon $f:\overleftrightarrow{AB}\longrightarrow \R$, slik at $f(A)=0$ og $f(B)>0$. 

        \begin{itemize}
            \item $\mathbf{x=0:}$
            I dette tilfelle kan vi velge $X=A$. Da er $AX=0$, og siden $A\neq B$ har vi $BX\neq 0$, slik at $AX/XB=0$. 
            Punktet $X$ er i dette tilfellet også unikt, ettersom $AX/XB=0$ impliserer $AX=0$, som kun er tilfellet dersom $X=A$. 
            
            \item $\mathbf{x>0:}$
            Siden $x$ er positiv må vi ha $A\ast X\ast B$ for å oppfylle kravet om ``sensed ratio''. 
            Fra teorem 3.2.17 må vi derfor ha $0=f(A)<f(X)<f(B)$. 
            Siden $f$ er en koordinatfunksjon vet vi i dette tilfellet at 
            $$AX=|f(A)-f(X)|=f(X)$$
            og at 
            $$BX=|f(B)-f(X)|=f(B)-f(X),$$
            der vi har brukt $f(X)<f(B)$ for å fjerne absoluttverditegnet. 
            Dermed kan vi utrykke $\frac{AX}{XB}$ ved hjelp av $f$:
            $$\frac{AX}{XB}=\frac{f(X)}{f(B)-f(X)}=x.$$
            Løser vi denne likningen for $f(X)$ får vi 
            $$f(X)=\frac{x}{x+1}f(B).$$
            Så, for å finne et punkt $X$ slik at $\frac{AX}{XB}=x$ trenger vi å finne $X\in \overleftrightarrow{AB}$ slik at $f(X)=\frac{x}{x+1}f(B)$.
            Siden $f$ er en koordinatfunksjon vet vi at $f$ er en bijeksjon, så en slik $X$ eksisterer, og er unik. 
            
            \item $\mathbf{-1<x<0:}$
            Dersom vi har en $X$ slik at $AX/XB=x$, må vi fra kraved om ``sensed ratio'', ha $X\ast A\ast B$ eller $A\ast B\ast X$.
            I dette tilfellet ønsker vi -- dersom vi behandler $\frac{AX}{XB}$ som en vanlig brøk -- at vi har $X$ slik at $\frac{AX}{XB}<1$, noe som betyr at vi ønsker $AX<XB$. 
            Dette betyr at vi har $X\ast A\ast B$. 
            Formulert ved funksjonen $f$ krever vi altså at $f(X)<0=f(A)<f(B)$.
            Da er 
            $$AX=|f(A)-f(X)|=-f(X)$$
            og at 
            $$BX=|f(B)-f(X)|=f(B)-f(X),$$
            der vi har brukt $f(X)<f(B)$ for å fjerne absoluttverditegnet. 
            Dermed kan vi uttrykke den vanlige brøken $\frac{AX}{XB}$ ved hjelp av $f$:
            $$\frac{AX}{XB}=\frac{-f(X)}{f(B)-f(X)}=-x.$$
            Vi løser denne likningen for $f(X)$ og får
            $$f(X)=\frac{x}{x+1} f(B).$$
            En unik $X$ som tilfredstiller denne likningen finnes igjen grunnet at $f$ er en koordinatfunksjon, og dermed en bijeksjon. 
            
            \item $\mathbf{x<-1:}$
            Dersom vi har en $X$ slik at $AX/XB=x$, må vi fra kraved om ``sensed ratio'' som over ha $X\ast A\ast B$ eller $A\ast B\ast X$.
            I dette tilfellet ønsker vi en $X$ slik at $\frac{AX}{XB}>1$, noe som betyr at vi ønsker $AX>XB$. 
            Dette betyr at vi har $A\ast B\ast X$. 
            Formulert ved funksjonen $f$ krever vi altså at $0=f(A)<f(B)<f(X)$.
            Da er 
            $$AX=|f(A)-f(X)|=f(X)$$
            og at 
            $$BX=|f(B)-f(X)|=f(X)-f(B),$$
            der vi har brukt $f(B)<f(X)$ for å fjerne absoluttverditegnet. 
            Dermed kan vi uttrykke $\frac{AX}{XB}$ ved hjelp av $f$:
            $$\frac{AX}{XB}=\frac{f(X)}{f(X)-f(B)}=x.$$
            Vi løser denne likningen for $f(X)$ og får
            $$f(X)=\frac{x}{x+1}f(B).$$
            En unik $X$ som tilfredstiller denne likningen finnes igjen grunnet at $f$ er en koordinatfunksjon, og dermed en bijeksjon. 
        \end{itemize}
    \end{punkt}

    \begin{punkt}
        Vi skal vise at det ikke finnes et punkt $X\in\overleftrightarrow{AB}$ slik at $AX/XB = -1$, derbrøken fremdeles tolkes som en ``sensed ratio''. 
        Fra korollar 3.2.19 har vi enten at $X\in \overline{AB}$, $X \ast A \ast B$ eller $A \ast B \ast X$.
        \begin{itemize}
            \item Hvis $X \in AB$, sier definisjonen av ´´sensed ratio'' at $AX/XB$ er positiv. 
            Dermed er spesielt $AX/XB \neq -1$.

            \item Hvis $A \ast B \ast X$, er $AB + BX = AX$ per definisjon av mellomliggenhet.
            Dermed er 
            $$\frac{AX}{XB}=\frac{AB+BX}{BX}=1+\frac{AB}{BX}>1,$$
            slik at $AX/XB \neq -1$.

            \item Hvis $X \ast A \ast B$ er $AX + AB = XB$ av definisjonen av mellomliggenhet.
            Dermed er
            $$\frac{AX}{XB}=\frac{AX}{AX+AB}<1,$$
            slik at $AX/XB \neq -1$.
        \end{itemize}
    \end{punkt}

    \begin{punkt}
        Vi skal lage en graf som illustrerer hvordan $AX/XB$ varierer med punktet $X\in \overleftrightarrow{AB}$. 
        For å lage en graf antar vi at $A$ og $B$ er to tall på den reelle tallinjen med $A<B$, og vi lar $\overleftrightarrow{AB}$ være $x$-aksen vår. 
        For $X\in \R$ er da $AX=|X-A|$ og $BX=|X-B|$, slik at 
        $$
        \frac{AX}{XB}=
        \begin{cases}
            \frac{X-A}{X-B}, \text{ for } X>B, \\
            \frac{X-A}{B-X}, \text{ for } A<X<B, \\
            \frac{X-A}{X-B}, \text{ for } X<A.
        \end{cases}
        $$
        Her har vi bare brukt definisjonen av absoluttverdi for å skrive ut uttrykket for den vanlige brøken $\frac{AX}{XB}$. 
        Men vi skal lage en graf av $AX/BX$, altså en ``sensed ratio'', og $AX/XB$ er definert slik at fortegnet er positivt når $A < X < B$, og negativt ellers. 
        Ergo
        $$
        AX/XB=
        \begin{cases}
            -\frac{X-A}{X-B}, \text{ for } X>B, \\
            \frac{X-A}{B-X}, \text{ for } A<X<B, \\
            -\frac{X-A}{X-B}, \text{ for } X<A.
        \end{cases}
        $$
        Men, det er lett å se at disse tre uttrykkene alltid er like, slik at $AX/BX=\frac{X-A}{B-X}$.
        For å faktisk lage en graf velger vi $A=0$ og $B=2$, slik at $AX/XB=\frac{x}{2-x}$. 
        Da får vi 

        \begin{figure}[H]
           \centering
            
\definecolor{qqwuqq}{rgb}{0,0.39215686274509803,0}
\begin{tikzpicture}[line cap=round,line join=round,>=triangle 45,x=1cm,y=1cm]
\begin{axis}[
x=0.6cm,y=0.6cm,
axis lines=middle,
ymajorgrids=true,
xmajorgrids=true,
xmin=-6,
xmax=6,
ymin=-6,
ymax=6,
xtick={-6,-4,...,6},
ytick={-6,-4,...,6},]
\clip(-6,-6) rectangle (6,6);
\draw[line width=2pt,color=qqwuqq,smooth,samples=1000,domain=-23.983343337153112:32.018955735219606] plot(\x,{(\x)/(2-(\x))});
\begin{scriptsize}
\draw[color=qqwuqq] (-23.654348107615675,-1.0414431448788832) node {$f$};
\end{scriptsize}
\end{axis}
\end{tikzpicture} 
        \end{figure}
    \end{punkt}
\end{oppgave}

\begin{oppgave}[6.1.1]
    Anta at alle trekanter i nøytral geometri har samme defekt, og kall denne defekten $c$. 
    Vi skal vise at vi har $c=0$. 

    La $\triangle ABC$ være en trekant, og la $D$ være midtpunktet på $\overline{BC}$. 
    Fra teorem 4.8.2 har vi da
    $$\delta(\triangle ABC) = \delta(\triangle ABD)+\delta(\triangle DCA),$$
    der $\delta$ måler defekten til trekantene. 
    Siden vi antok at defekten til alle trekanter er den samme, blir denne likningen
    $$c=c+c=2c,$$
    som medfører at $c=0$. 

    Hvis alle trekanter har samme defekt har vi vist at denne defekten må være $0$. 
    At alle trekanter har defekt lik $0$ er av teorem 4.7.4 ekvivalent med euklids paralellpostulat. 
    Dermed kan ikke alle trekanter i hyperbolsk geometri ha samme defekt. 

    \begin{figure}[H]
        \centering
        
\definecolor{qqqqff}{rgb}{0,0,1}
\begin{tikzpicture}[line cap=round,line join=round,>=triangle 45,x=1cm,y=1cm]
\clip(-4,-0.2) rectangle (4,6);
\draw [line width=2pt] (-3,0)-- (0,5);
\draw [line width=2pt] (0,5)-- (3,0);
\draw [line width=2pt] (-3,0)-- (3,0);
\draw [line width=2pt] (-3,0)-- (1.5,2.5);
\draw [fill=qqqqff] (-3,0) circle (2.5pt);
\draw[color=qqqqff] (-3.3,0.43) node {$A$};
\draw [fill=qqqqff] (3,0) circle (2.5pt);
\draw[color=qqqqff] (3.32,0.43) node {$B$};
\draw [fill=qqqqff] (0,5) circle (2.5pt);
\draw[color=qqqqff] (0.32,5.43) node {$C$};
\draw [fill=qqqqff] (1.5,2.5) circle (2pt);
\draw[color=qqqqff] (1.82,2.89) node {$D$};
\end{tikzpicture} 
    \end{figure}
\end{oppgave}

\begin{oppgave}[6.1.2]
    La $\square ABCD$ være en Saccheri firkant med base $\overline{AB}$. 
    La $N$ og $M$ være midtpunktene på $\overline{CD}$ og $\overline{AB}$ respektivt. 
    Vi ønsker å vise at $MN<BC$.

    \begin{figure}[H]
        \centering
        
\definecolor{qqwuqq}{rgb}{0,0.39215686274509803,0}
\definecolor{qqqqff}{rgb}{0,0,1}
\begin{tikzpicture}[line cap=round,line join=round,>=triangle 45,x=1.1cm,y=1.1cm]
\clip(-5,-0.3) rectangle (5,4.5);
\draw[line width=2pt,color=qqwuqq,fill=qqwuqq,fill opacity=0.10000000149011612] (-0.30782033307559736,3) -- (-0.3078203330755975,2.6921796669244027) -- (0,2.6921796669244027) -- (0,3) -- cycle; 
\draw[line width=2pt,color=qqwuqq,fill=qqwuqq,fill opacity=0.10000000149011612] (0,0.30782033307559736) -- (0.3078203330755974,0.30782033307559736) -- (0.30782033307559736,0) -- (0,0) -- cycle; 
\draw[line width=2pt,color=qqwuqq,fill=qqwuqq,fill opacity=0.10000000149011612] (3,0.30782033307559736) -- (2.6921796669244027,0.3078203330755974) -- (2.6921796669244027,0) -- (3,0) -- cycle; 
\draw[line width=2pt,color=qqwuqq,fill=qqwuqq,fill opacity=0.10000000149011612] (-2.6921796669244027,0) -- (-2.6921796669244027,0.3078203330755973) -- (-3,0.30782033307559736) -- (-3,0) -- cycle; 
\draw [line width=2pt] (-3,0)-- (3,0);
\draw [line width=2pt] (3,0)-- (3,4);
\draw [line width=2pt] (2.9129352620380575,2) -- (3.087064737961943,2);
\draw [line width=2pt] (-3,0)-- (-3,4);
\draw [line width=2pt] (-3.087064737961943,2) -- (-2.9129352620380575,2);
\draw [shift={(0,8)},line width=2pt]  plot[domain=4.068887871591405:5.355890089177974,variable=\t]({1*5*cos(\t r)+0*5*sin(\t r)},{0*5*cos(\t r)+1*5*sin(\t r)});
\draw [line width=2pt] (0,3)-- (0,0);
\draw [fill=qqqqff] (-3,0) circle (2.5pt);
\draw[color=qqqqff] (-3.3,0.3083615192232222) node {$A$};
\draw [fill=qqqqff] (3,0) circle (2.5pt);
\draw[color=qqqqff] (3.3,0.3083615192232222) node {$B$};
\draw [fill=qqqqff] (3,4) circle (2.5pt);
\draw[color=qqqqff] (3.2303608292399337,4.313339465472579) node {$C$};
\draw [fill=qqqqff] (-3,4) circle (2.5pt);
\draw[color=qqqqff] (-2.762595300473785,4.313339465472579) node {$D$};
\draw [fill=qqqqff] (0,3) circle (2pt);
\draw[color=qqqqff] (0.22662736955291224,3.2830733995895924) node {$N$};
\draw [fill=qqqqff] (0,0) circle (2.5pt);
\draw[color=qqqqff] (-0.3,0.3083615192232222) node {$M$};
\end{tikzpicture} 
    \end{figure}
    
    Fra del 3 av teorem 4.8.10, vet vi at $\overline{NM}\perp\overline{AB}$ og $\overline{NM}\perp\overline{CD}$. 
    Dette betyr at $\square MBCN$ er en Lambert firkant. 
    Dermed kan vi fra teorem 6.1.7 konkludere med at $MN<BC$, som var det vi ville vise. 
\end{oppgave}

\begin{oppgave}[6.1.3]
    La $\square ABCD$ være en Saccheri firkant med base $\overline{AB}$. 
    La $N$ og $M$ være midtpunktene på $\overline{CD}$ og $\overline{AB}$ respektivt (se figuren til oppgave 6.1.2).
    Vi ønsker å vise at $AB<CD$. 
    
    Fra del 3 av teorem 4.8.10, har vi $\overline{MN}\perp \overline{AD}$ og $\overline{MN}\perp\overline{CD}$.
    Dermed er både $\square MBCN$ og $\square AMND$ Lambert firkanter. 
    Fra teorem 6.1.7 kan vi dermed konkludere med at $MB<NC$ og $AM<DN$. 
    Siden $M$ og $N$ er midtpunktene på linjene $\overline{AB}$ og $\overline{CD}$, har vi spesielt at $A\ast M\ast B$ og $C\ast N\ast D$. 
    Dette betyr at vi har $AB=AM+MB$ og $CD=CN+ND$. 
    Så ved å addere de to ulikhetene vi fikk fra teorem 6.1.7 får vi $AC<CD$, som var det vi ønsket å vise. 
\end{oppgave}
