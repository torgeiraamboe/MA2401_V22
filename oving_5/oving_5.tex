

\begin{oppgave}[4.2.3]
    Vi antar at vi har to trekanter $\triangle ABC$ og $\triangle DEF$ slik at $AC=DF$, $\angle BAC\cong \angle EDF$ og $CB=FE$. 
    Vi skal vise at vi enten har $\mu(\angle ABC) = \mu(\angle DEF)$ eller $\mu(\angle ABC) + \mu(\angle DEF)=180$. 
    
    Vi antar uten tap av generalitet at  $AB\geq DE$. 
    Ved hjelp av linjalpostulatet kan vi da finne et punkt $P\in \overline{AB}$ slik at $AP=DE$. 
    Vi har to muligheter: $P=B$ eller $A\ast P\ast B$.

    Dersom $P=B$ kan vi bruke side-vinkel-side-postulatet (SVS) på trekantene for å konkludere med at $\triangle ABC \cong \triangle DEF$, noe som da spesielt betyr at  $\mu(\angle ABC) = \mu(\angle DEF)$.

    Dersom $A\ast P\ast B$ kan vi istedet bruke SVS på trekantene $\triangle APC$ og $\triangle DEF$, slik at vi har $\triangle APC\cong \triangle DEF$.
    Spesielt må vi da ha $\overline{PC}\cong \overline{EF}$, som sammen med antagelsen vår om at $\overline{EF}\cong \overline{BC}$ gir oss at $\overline{BC}\cong \overline{PC}$.
    Dermed er trekanten $\triangle PBC$ en likebeint trekant. 

    \begin{figure}[H]
        \centering
        
\definecolor{qqqqff}{rgb}{0,0,1}
\begin{tikzpicture}[line cap=round,line join=round,>=triangle 45,x=1cm,y=1cm]
\clip(-1,-0.5) rectangle (12.5,4);
\draw [line width=2pt] (0,0)-- (4,3);
\draw [line width=2pt] (4,3)-- (6,0);
\draw [line width=2pt] (6,0)-- (3,0);
\draw [line width=2pt] (3,0)-- (0,0);
\draw [line width=2pt] (3,0)-- (4,3);
\draw [line width=2pt] (8,0)-- (11,0);
\draw [line width=2pt] (12,3)-- (11,0);
\draw [line width=2pt] (12,3)-- (8,0);
\draw [fill=qqqqff] (0,0) circle (2pt);
\draw[color=qqqqff] (-0.02,0.39) node {$A$};
\draw [fill=qqqqff] (6,0) circle (2.5pt);
\draw[color=qqqqff] (6.22,0.39) node {$B$};
\draw [fill=qqqqff] (4,3) circle (2.5pt);
\draw[color=qqqqff] (4.32,3.33) node {$C$};
\draw [fill=qqqqff] (3,0) circle (2.5pt);
\draw[color=qqqqff] (3.4,0.33) node {$P$};
\draw [fill=qqqqff] (8,0) circle (2.5pt);
\draw[color=qqqqff] (8,0.39) node {$D$};
\draw [fill=qqqqff] (11,0) circle (2.5pt);
\draw[color=qqqqff] (11.44,0.39) node {$E$};
\draw [fill=qqqqff] (12,3) circle (2.5pt);
\draw[color=qqqqff] (12.32,3.33) node {$F$};
\end{tikzpicture}

    \end{figure}

    Vi kan da anvende likebeint trekant-teoremet som gir oss at $\angle BPC \cong \angle ABC$. 
    Vinklene $\angle BPC$ og $\angle APC$ danner også et lineært par. 
    Fra lineært par-teoremet får vi da at $\mu(\angle BPC)+\mu(\angle APC)=180$. 
    Siden vi vet at $\triangle APC\cong \triangle DEF$ vet vi også at $\angle APC\cong\angle DEF$. 
    Vi kan kombinere dette med likningen vi fikk fra lineært par-teoremet til å få $$\mu(\angle ABC)+\mu(\angle DEF)=180,$$ som var det vi ville vise.

\end{oppgave}


\begin{oppgave}[4.2.5]
    Vi har en trekant $\triangle ABC$ og et linjestykke $\overline{DE}$ slik at $\overline{DE}\cong \overline{AB}$. 
    Vi skal vise at for et halvplan $H$ bestemt av linjen $\overleftrightarrow{DE}$ finnes et unikt punkt $F\in H$ slik at $\triangle ABC\cong \triangle DEF$.
    
    Fra gradskivepostulatet (del 3) kan vi finne en unik stråle $\overrightarrow{DP}$ slik at $P\in H$ og $\angle EDP\cong \angle BAC$.
    Fra linjalpostulatet (nærmere bestemt teorem 3.2.23) kan vi finne et unikt punkt $F\in \overrightarrow{DP}$ slik at $DF=AC$. 
    Av strpleteoremet (teorem 3.3.9) ligger også punktet $F$ i halvpplanet $H$.

    \begin{figure}[H]
        \centering
        
\definecolor{qqqqff}{rgb}{0,0,1}
\begin{tikzpicture}[line cap=round,line join=round,>=triangle 45,x=1cm,y=1cm]
\clip(-0.5,-0.1) rectangle (14.5,6);
\draw [line width=2pt] (1,0)-- (5,3);
\draw [line width=2pt] (5,3)-- (6,0);
\draw [line width=2pt] (7,0)-- (12,0);
\draw [line width=2pt] (11,3)-- (12,0);
\draw [line width=2pt] (11,3)-- (7,0);
\draw [line width=2pt] (1,0)-- (6,0);
\draw [line width=2pt] (11,3)-- (15,6);
\draw [fill=qqqqff] (1,0) circle (2pt);
\draw[color=qqqqff] (0.92,0.39) node {$A$};
\draw [fill=qqqqff] (6,0) circle (2.5pt);
\draw[color=qqqqff] (6.22,0.39) node {$B$};
\draw [fill=qqqqff] (5,3) circle (2.5pt);
\draw[color=qqqqff] (5.26,3.40) node {$C$};
\draw [fill=qqqqff] (7,0) circle (2.5pt);
\draw[color=qqqqff] (7,0.39) node {$D$};
\draw [fill=qqqqff] (12,0) circle (2.5pt);
\draw[color=qqqqff] (12.24,0.39) node {$E$};
\draw [fill=qqqqff] (11,3) circle (2.5pt);
\draw[color=qqqqff] (11.26,3.60) node {$F$};
\draw [fill=qqqqff] (13.7968,5.0976) circle (2.5pt);
\draw[color=qqqqff] (13.68,4.63) node {$P$};
\end{tikzpicture}

    \end{figure}
    
    Dermed vet vi at 
    \begin{itemize}
        \item $\overline{AB}\cong \overline{DE}$
        \item $\angle CAB\cong \angle FDE$
        \item $\overline{AC}\cong \overline{DF}$
    \end{itemize}
    som vil si at vi kan anvende side-vinkel-side-postulatet på trekantene $\triangle ABC$ og $\triangle DEF$ til å få $\triangle ABC\cong \triangle DEF$. 
    Punktet $F$ er unikt siden både gradskivepostulatet og linjalpostulatet ga oss respektivt en unik stråle og et unikt punkt på strålen. 
\end{oppgave}


\begin{oppgave}[4.3.1]
    La $A$, $B$ og $C$ være tre ikke-kolineære punkter slik at $\mu(\angle ACB)>\mu(\angle BAC)$. 
    Vi vil vise at $AB>BC$.

    Fra trikotomi vet vi at en av følgende må være sanne: $AB=BC$, $AB<BC$ eller $AB>BC$. 

    Anta at vi har $AB=BC$. I dette tilfelle er trekanten $\triangle ABC$ likebeint. 
    Fra likebeint trekant-teoremet får vi da at $\mu(\angle ACB)=\mu(\angle BAC)$, noe som motsier antagelsen vår om at $\mu(\angle ACB)>\mu(\angle BAC)$.
    Dermed kan vi konkludere med at $AB\neq BC$. 

    Anta nå at $AB<BC$. 
    Fra den delen av teoremet som bevises i boka vet vi da at $\mu(\angle ACB)<\mu(\angle BAC)$, noe som motsier antagelsen vår. 
    Dermed kan vi ikke ha $AB<BC$, som vil si at vi kan konkludere med at $AB>BC$, som var det vi ville vise. 
\end{oppgave}


\begin{oppgave}[4.3.2]
    Vi skal vise følgende påstand: hvis $A$, $B$ og $C$ er tre ikke-kolineære punkter, så er $AC<AB+BC$. 

    Fra linjalpostulatet (mer spesifikt teorem 3.2.23) kan vi finne et punkt $D$ på linjen $\overleftrightarrow{AB}$ slik at $A\ast B\ast D$ og $BD=BC$. 

    \begin{figure}[H]
        \centering
        
\definecolor{qqqqff}{rgb}{0,0,1}
\begin{tikzpicture}[line cap=round,line join=round,>=triangle 45,x=1cm,y=1cm]
\clip(-1,-1) rectangle (9, 5);
\draw [line width=2pt] (0,0)-- (3,4);
\draw [line width=2pt] (3,4)-- (4,0);
\draw [line width=2pt] (0,0)-- (4,0);
\draw [line width=2pt] (4,0)-- (8.52,0);
\draw [line width=2pt] (3,4)-- (8.52,0);
\draw [fill=qqqqff] (0,0) circle (2pt);
\draw[color=qqqqff] (-0.12,0.35) node {$A$};
\draw [fill=qqqqff] (4,0) circle (2.5pt);
\draw[color=qqqqff] (4.22,0.35) node {$B$};
\draw [fill=qqqqff] (3,4) circle (2.5pt);
\draw[color=qqqqff] (3.32,4.43) node {$C$};
\draw [fill=qqqqff] (8.52,0) circle (2.5pt);
\draw[color=qqqqff] (8.84,0.35) node {$D$};
\end{tikzpicture}

    \end{figure}

    Dersom vi klarer å vise at $\mu(\angle ADC)<\mu(\angle ACD)$ vil vi være i mål. 
    Dette er fordi teorem 4.3.1 (som vi fullførte beviset for i forrige oppgave) vil gi oss at $$AC<AD=AB+BD = AB+BC$$.
    
    Trekanten $\triangle BCD$ er likebeint per konstruksjon, ettersom vi har valgt $D$ slik at $BD=BC$. 
    Av likebeint trekant-teoremet har vi da $\angle ADC\cong \angle BCD$. 
    Siden vi har $A\ast B\ast D$ vet vi fra teorem 3.3.10 at strålen $\overrightarrow{CB}$ ligger mellom strålene $\overrightarrow{CA}$ og $\overrightarrow{CD}$. 
    Del 4 av gradskivepostulatet gir oss dermed at $$ \mu(\angle ACD)=\mu(\angle ACB)+\mu(\angle BCD)>\mu(\angle BCD).$$
    Siden vi også vet at $\angle ADC\cong \angle BCD$ får vi at 
    $$\mu(\angle ACD)>\mu(\angle ADC),$$
    noe som vi over viste at fullfører beviset. 

\end{oppgave}


\begin{oppgave}[4.3.3]
    La $A$, $B$ og $C$ være tre punkter slik at $AC=AB+BC$. 
    Vi vil vise at de tre punktene er kolineære, altså at de alle ligger på en felles linje. 

    Anta at punktene ikke ligger på en linje. Fra trekantulikheten vet vi da at $AC<AB+BC$, noe som motsier antagelsen vår. 
    Dermed må vi ha at punktene $A$, $B$ og $C$ ligger på en felles linje. 
\end{oppgave}


\begin{oppgave}[4.3.4]
    La $A$, $B$ og $C$ være tre ulike punkter. 
    Vi vil vise at vi har $AB+BC\geq AC$.
    Vi deler beviset inn i to tilfeller: punktene er kolineære; punktene er ikke-kolineære. 

    Anta først at punktene ikke er kolineære. 
    Fra trekantulikheten vet vi da at $AC<AB+BC$, noe som stemmer med utsagnet vi skal vise. 

    Anta nå at punktene er kolineære. 
    Fra korollar 3.2.19 vet vi at et av punktene må ligge mellom de to andre.
    \begin{itemize}
        \item Hvis $A\ast B\ast C$ har vi $AC = AB+BC$ per definisjon av mellomliggenhet, noe som stemmer med utsagnet vi vil vise. 
        \item Hvis $A\ast C\ast B$ har vi $AB=AC+BC$ per definisjon av mellomliggenhet. Spesielt betyr dette at $$AC<AB<AB+BC,$$ som stemmer overens med utsagnet vårt. 
        \item Hvis $B\ast A\ast C$ har vi $BC=AB+AC$ per definisjon av mellomliggenhet. Spesielt har vi $AC<BC<AB+BC$, som stemmer overens med utsagnet vi skal vise. 
    \end{itemize}
    Vi har nå vist at $AB+BC\geq AC$ i alle de mulige tilfellene, noe som betyr at vi har bevist påstanden. 
\end{oppgave}


\begin{oppgave}[4.3.8]
    La $A$, $B$ og $C$ være tre punkter som ikke ligger på en linje, og $P$ et punkt i det indre av vinkelen $\angle BAC$. 
    Vi vil vise at $P$ ligger på halvveringsstrålen til $\angle BAC$ hvis og bare hvis $d(P, \overleftrightarrow{AB}=d(P, \overleftrightarrow{AC}))$. 
    Siden oppgaven er å vise en ``hvis og bare hvis'' påstand, deler vi beviset inn i to: hvis-delen og bare hvis-delen. 

    \textbf{Hvis:} 
    Anta først at punktet $P$ ligger på vinkelhalveringsstrålen til $\angle BAC$. 
    Vi må nå vise at vi har $d(P, \overleftrightarrow{AB}=d(P, \overleftrightarrow{AC}))$. 
    Fra teorem 4.1.3 kan vi finne et pukt $D\in \overleftrightarrow{AB}$ slik at $\overleftrightarrow{AB}\perp \overleftrightarrow{DP}$.
    På samme måte kan vi finne et punkt $E$ på $\overleftrightarrow{AC}$ slik at $\overleftrightarrow{AC}\perp \overleftrightarrow{EP}$.

    \begin{figure}[H]
        \centering
        
\definecolor{qqqqff}{rgb}{0,0,1}
\begin{tikzpicture}[line cap=round,line join=round,>=triangle 45,x=1cm,y=1cm]
\clip(-1,-1) rectangle (10,6);
\draw [line width=2pt] (0,0)-- (7.64,5.06);
\draw [line width=2pt] (0,0)-- (9,0);
\draw [line width=2pt] (0,0)-- (5.497819587688473,1.6555273523804552);
\draw [line width=2pt] (5.497819587688473,1.6555273523804552)-- (10.790436445640157,3.2492631660528857);
\draw [line width=2pt] (4.583671561936495,3.035782474266841)-- (5.497819587688473,1.6555273523804552);
\draw [line width=2pt] (5.497819587688473,1.6555273523804552)-- (5.497819587688474,0);
\draw [fill=qqqqff] (0,0) circle (2pt);
\draw[color=qqqqff] (-0.02,0.43) node {$A$};
\draw [fill=qqqqff] (7.64,5.06) circle (2.5pt);
\draw[color=qqqqff] (7.96,5.49) node {$C$};
\draw [fill=qqqqff] (9,0) circle (2.5pt);
\draw[color=qqqqff] (9.32,0.33) node {$B$};
\draw [fill=qqqqff] (5.497819587688473,1.6555273523804552) circle (2.5pt);
\draw[color=qqqqff] (5.78,2.05) node {$P$};
\draw [fill=qqqqff] (10.790436445640157,3.2492631660528857) circle (2.5pt);
\draw[color=qqqqff] (10.96,3.67) node {$F$};
\draw [fill=qqqqff] (4.583671561936495,3.035782474266841) circle (2pt);
\draw[color=qqqqff] (4.9,3.7) node {$E$};
\draw [fill=qqqqff] (5.497819587688474,0) circle (2pt);
\draw[color=qqqqff] (5.82,0.39) node {$D$};
\end{tikzpicture}

    \end{figure}

    Ettersom vi har antatt at $P$ ligger på vinkelhalveringsstrålen, og vi vet at både $\angle PEA$ og $\angle PDA$ er rette vinkler, vet vi at $\angle DAP\cong \angle EAP$, $\angle PDA\cong \angle PEA$ og $AP=AP$.
    Dermed kan vi bruke vinkel-vinkel-side-teoremet (VVS) til å konkludere med at $\triangle APD\cong \triangle APE$. 
    Spesielt betyr dette at $PD=PE$, som betyr at $d(P, \overleftrightarrow{AB}=d(P, \overleftrightarrow{AC}))$. 

    \textbf{Bare hvis:} Anta nå at $d(P, \overleftrightarrow{AB}=d(P, \overleftrightarrow{AC}))$. 
    Vi må vise at $P$ ligger på vinkelhalveringsstrålen til $\angle BAC$, altså at $\angle PAB\cong \angle PAC$. 
    La punktene $D$ og $E$ være definert som tidligere. 
    Per antagelse, sammen med definisjonen av avstand mellom et punkt og en linje, har vi $PD=PE$. 
    Dette betyr at vi har $AP=AP$, $PD=PE$ og $\mu(\angle PEA)=\mu(\angle PDA)=90$. 
    Vi kan dermed anvende hypotenus-katet-teoremet (teorem 4.2.5), noe som gir oss at $\triangle APE\cong\triangle APD$. 
    Spesielt betyr dette at vi har $\angle DAP\cong \angle EAP$, som var det vi ville vise. 

\end{oppgave}
