

\begin{oppgave}[4.7.2]
    Anta først at det euklidske parallellpostulatet holder. 
    Vi vil vise at dersom $l \parallel m$, og en linje $t\neq l$ skjærer $l$, så må $t$ også skjære $m$. 
    Dette kalles ofte Proclus' aksiom. 
    Vi bruker et bevis med selvmotsigelse, og antar derfor at $t$ skjærer $l$, men at $t$ \emph{ikke} skjærer $m$. 
    Med andre ord betyr dette at $t\parallel m$. 

    La $P$ være skjæringspunktet mellom $t$ og $l$. 
    Da er $t$ og $l$ to linjer som begge går gjennom $P$ og er paralelle til $m$. 
    Dette motsier det euklidske parallellpostulatet. 
    Dermed kan vi konkludere med at $t$ skjærer $m$, som var det vi ville vise. 
    
    Anto nå at Proclus' aksiom holder, og at vi har en linje $l$ og et punkt $P$ som ikke ligger på $l$. 
    Vi må vise at det finnes en unik linje $m$ som inneholder $P$ og som er parallell med $l$. 
    Fra teorem 4.1.3 vet vi at vi kan finne et unikt punkt $A$ slik at $\overleftrightarrow{PA}\perp l$. 
    Fra del 3 av gradskivepostulatet kan vi finne et punkt $Q$ slik at $\mu(\angle APQ)=90$. 
    Merk at vi må egentlig spesifisere et halvplan for å bruke denne påstanden, 
    men i dette tilfelle har det ikke noe å si hvilket halvplan vi velger. 
    Det alternerende indre vinkel-teoremet gir oss at $\overleftrightarrow{PQ}\parallel l$, 
    siden $\overleftrightarrow{PA}$ skjærer både $l$ og $\overleftrightarrow{PQ}$ på en slik måte at alle indre vinkler er rette. 
    Dermed har vi funnet en linje $m=\overleftrightarrow{PQ}$ slik at $P\in m$ og $m\parallel l$. 
    Merk her at vi ikke har brukt hverken det euklidske parallellpostulatet eller Proclus' aksiom, så denne konstruksjonen av en parallell linje er fungerer derfor i nøytral geometri.
    Konstruksjonen kalles ofte for den doble perpendikulær konstruksjonen. 

    Det gjenstår å vise at denne linja er unik. 
    Anta defor at vi har en linje $n\neq m$ som er paralell med $l$ og har $P\in n$. 
    Dermed skjærer $n$ linjen $m$ i punktet $P$.  
    Proclus' aksiom sier oss da at $n$ også må skjære $l$, noe som motsier at $n$ og $l$ er paralelle. 
    Dermed er $m$ unik, og beviset er fullført. 
\end{oppgave}

\begin{oppgave}[4.7.6]
    Vi skal vise at det euklidske parallellpostulatet (sammen med alle aksiomene i nøytral geometri) impliserer at alle trekanter har vinkelsum lik $180$. 
    Vi kommer til å bruke at boka har vist at det euklidske parallellpostulatet er ekvivalent med motsatsen til det alternerende indre vinkel-teoremet (teorem 4.7.1).
    Resultatet vi skal vise er en del av teorem 4.7.4, så vi tar oss friheten til å bruke resultatene vist før dette teoremet. 

    Ved hjelp av del 3 av gradskivepostulatet kan vi finne et punkt $D$ slik at $\mu(\angle BCD)=\mu(\angle ABC)$, og slik at $D$ og $A$ ligger på motsatt side av $\overleftrightarrow{BC}$. 
    Siden $\mu(\angle BCD) = \mu(\angle ABC)$, gir indre vinkel-teoremet oss at $\overleftrightarrow{CD}\parallel \overleftrightarrow{AB}$. 
    Vi kan så bruke linjalpostulatet til å finne et punkt $E$ på $\overleftrightarrow{CD}$ slik at $E\ast C\ast D$. 
    Siden $\overleftrightarrow{EC}=\overleftrightarrow{CD}$, vet vi at $\overleftrightarrow{EC}\parallel \overleftrightarrow{AB}$. 
    Dermed gir motsatsen til det alternerende indre vinkel-teoremet (teorem 4.7.1) oss at $\mu(\angle ACE)=\mu(\angle CAB)$. 

    Til nå har vi vist at $\mu(\angle BCD)=\mu(\angle ABC)$ og $\mu(\angle ACE)=\mu(\angle CAB)$. 
    For å vise at 
    $$\mu(\angle ABC)+\mu(\angle CAB)+\mu(\angle ACB)=180,$$
    er det dermed nok å vise at 
    $$\mu(\angle BCD)+\mu(\angle ACE)+\mu(\angle ACB)=180.$$
    Men, denne siste ligningen følger av å bruke lineært par-teoremet fordi dette teoremet gir oss at $\mu(\angle ACE)=\mu(\angle ACD)=180$. 
    Av del 4 av gradskivepostulatet vet vi at $\mu(\angle ACD)=\mu(\angle ACB)+\mu(\angle BCD)$, slik at vi til sammen har 
    \begin{align*}
        180 
        &= \mu(\angle ACE)+\mu(\angle ACD) \\
        &= \mu(\angle ACE)+\mu(\angle ACB)+\mu(\angle BCD) \\
        &= \mu(\angle ABC)+\mu(\angle CAB)+\mu(\angle ACB), 
    \end{align*}
    der vi i siste ligning har brukt at $\mu(\angle BCD)=\mu(\angle ABC)$ og $\mu(\angle ACE)=\mu(\angle CAB)$. 
    Dermed har vi vist resultatet vi ønsket å vise. 
\end{oppgave}

\begin{oppgave}[4.8.1]
    La $\triangle ABC$ være en trekant og $E$ et punkt i det indre av $\overline{BC}$. 
    Vi skal vise at $\delta(\triangle ABC)=\delta(\triangle ABE)+\delta(\triangle ECA)$. 

    Ved å skrive ut definisjonen av $\delta$ ser vi at denne påstanden er ekvivalent med å vise at 
    $$180 - \sigma(\triangle ABC) = 180-\sigma(\triangle ABE)+180-\sigma(\triangle ECA).$$
    Med litt enkel algebraisk manipulasjon kan vi kan skrive om denne ligningen til
    $$\sigma(\triangle ABC)+180 = \sigma(\triangle ABE)+\sigma(\triangle ECA).$$
    Denne siste likningen, som vi altså må vise at stemmer, er nettopp innholdet i lemma 4.5.4, som vil si at vi har vist resultatet vi var ute etter å vise. 
\end{oppgave}

\begin{oppgave}[4.8.2]
    Vi skal vise at når $\square ABCD$ er en konveks firkant, er $\delta \square ABCD = \delta (\triangle ABC)+\delta(\triangle ACD)$. 

    En del av definisjonen av konveks (definisjon 4.6.2), er at $A$ ligger i det indre av $\angle BCD$. 
    Dermed gir del 4 av gradskivepostulatet oss at 
    $$\mu(\angle BCA)+\mu(\angle ACD)=\mu(\angle BCD),$$
    og på akkuratt samme måte viser vi at 
    $$\mu(\angle DAC)+\mu(\angle CAB)=\mu(\angle DAB).$$
    Vi kan nå fullføre beviset med litt enkel algebra
    \begin{align*}
        \delta(\square ABCD)
        &= 360 - \sigma(\square ABCD) \\
        &= 360 - (\mu(\angle ABC)+\mu(\angle CDA)+\mu(\angle DAB)+\mu(\angle BCD))\\
        &= 360 - (\sigma(\triangle ABC)+\sigma(\triangle ACD))\\
        &= 180 - \sigma(\triangle ABC)+ 180-\sigma(\triangle ACD)\\
        &= \delta(\triangle ABC)+\delta(\triangle ACD),
    \end{align*}
    der vi i den tredje ligningen har brukt de to ligningene vi fant for $\mu(\angle BCD)$ og $\mu(\angle DAB)$, sammen med definisjonen av $\sigma$.
    Det meste vi gjør i denne oppgaven er altså å bruke definisjonene til $\delta$ og $\sigma$. 
\end{oppgave}

\begin{oppgave}[4.8.5]
    Vi skal vise teorem 4.8.10. I denne oppgaven vil derfor $\square ABCD$ være en Saccheri-firkant, altså at $\angle ABC$ og $\angle DAB$ er rette vinkler og at $\overline{AD}\cong \overline{BC}$. 
    Siden teloremet har 6 deler, må vi vise 6 påstander. 
    \begin{enumerate}
        \item $\overline{AC}\cong \overline{BD}$: 
        Vi betrakter de to trekantene $\triangle ABD$ og $\triangle BAC$, se figuren. 
        Per antagelse vet vi følgende
        $$\overline{AD}\cong \overline{BC},\quad \overline{AB}\cong \overline{BA}\quad \text{ og } \quad\mu(\angle ABD)=\mu(\angle BAC)=90.$$
        Dermed gir SVS (side-vinkel-side) oss at $\triangle ABD\cong \triangle BAC$, som spesielt betyr at $\overline{AC}\cong \overline{BD}$. 

        \item $\angle BCD \cong \angle ADC$: 
        Denne gangen vender vi blikket mot trekantene $\triangle ADC$ og $\triangle BCD$. 
        Siden vi nettopp viste at $\overline{AC}\cong \overline{BD}$, vet vi nå at alle sidene i disse to trekantene er kongruente. 
        Dermed gir SSS (side-side-side) oss at ${triangle ACD}\cong \triangle BDC$, noe som spesielt betyr at $\angle BCD \cong \angle ADC$. 

        \item Linjestykket fra midtpunktet til $\overline{AB}$ til midtpunktet av $\overline{CD}$ står vinkelrett på $\overline{AB}$ og $\overline{CD}$: 
        La $M$ være midtpunktet til $\overline{AB}$ og $N$ være midtpunktet til $\overline{CD}$. 
        Vi begynner med å kikke på trekantene $\triangle AMD$ og $\triangle BMC$. 
        Siden $\overline{AM}\cong \overline{BM}$ (ettersom $M$ er midtpunktet mellom de), kan vi bruke SVS til å si at 
        $$\triangle AMD\cong \triangle BMC.$$
        Spesielt har vi $\overline{DM}\cong \overline{CM}$. 
        Fra dette har vi nå at alle sidene i trekantene $\triangle DMN$ og $\triangle CMN$ er kongruente, slik at SSS gir oss at $\triangle DMN \cong \triangle CMN$. 
        Spesielt betyr dette at $\mu(\angle MND)=\mu(\angle MNC)$, og siden $\angle MND$ og $\angle MNC$ er supplementærvinkler vet vi fra lineært par-teoremet at 
        $$ 180 = \mu(\angle MND)+\mu(\angle MNC)=1\mu(\angle MND).$$
        Dette betyr at $\mu(\angle MND)=90$, som viser at vi har $\overline{MN}\perp \overline{CD}$. 
        
        Det gjenstår å vise at $\overline{MN}\perp \overline{AB}$, men beviset for denne påstanden er veldig likt første del, så vi skisserer bare raskt. 
        Siden vi fra punkt 2 av denne oppgaven at $\angle BCD \cong \angle ADC$, kan vi bruke SVS til å konkludere med at $\triangle BCN \cong \triangle ADN$, og da spesielt at $\overline{AN}\cong \overline{BN}$. 
        Som i første del kan vi da bruke lineært par-teoremet til å konkludere med at vinkelen $\angle AMN$ er rett. 

        \item $\square ABCD$ er et paralellogram: 
        Siden linjen $\overleftrightarrow{AB}$ skjærer $\overleftrightarrow{AD}$ og $\overleftrightarrow{BC}$ slik at skjæringsvinklene er rette, gir alternerende indre vinkel-teoremet at $\overleftrightarrow{AD}\parallel\overleftrightarrow{BC}$. 
        Tilsvarende skjærer $\overleftrightarrow{MN}$ linjene $\overleftrightarrow{AB}$ og $\overleftrightarrow{CD}$ slik at skjæringsvinklene er rette (dette er resultatet fra punkt 3 av denne oppgaven). 
        Vi vet da at $\mu(\angle AMN) = \mu(\angle DNM)=90$. 
        Fra lineært par-teoremet vet vi da at $\mu(\angle CNM)=90$. 
        Dette betyr at de to indre vinklene $\angle AMN$ og $\angle CNM$ er kongruente. 
        Det alternerende indre vinkel-teoremet gir oss da at $\overline{AB}\cong \overline{CD}$. 

        \item $\square ABCD$ er konveks: 
        Vi har vist at $\square ABCD$ er et paralellogram, og i forrige øving viste vi teorem 4.6.6 som sier at alle paralellogram er konvekse. 
        Dermed er vi ferdige.

        \item Vinklene $\angle BCD$ og $\angle CDA$ er enten rette elle spisse: 
        Vi må vise at $\mu(\angle BCD)\leq 90$ og at $\mu(\angle CDA)\leq 90$. 
        Fra teorem 4.6.4 vet vi at 
        $$\mu(\angle ABC)+\mu(\angle BCD)+\mu(\angle CDA)+\mu(\angle DAB)\leq 360.$$
        Vi vet at $\mu(\angle ABC)=\mu(\angle DAB)=90$, og fra punkt 2 av denne oppgaven vet vi at $\mu(\angle CDA)=\mu(\angle BCD)$. 
        Setter vi dette inn i ligningen over får vi at $2\mu(\angle CDA)\leq 180$, slik at vi har $\mu(\angle CDA)=\mu(\angle BCD)\leq 90$. 
    \end{enumerate} 
\end{oppgave}

\begin{oppgave}[4.8.8]
    La $\square ABCD$ være en Lambert-firkant, altså en firkant hvor vinkelen ved de tre hjørnene $A$, $B$ og $C$ er rette. 
    
    \begin{figure}
        
    \end{figure}

    Vi skal vise 4 utsagn.
    \begin{enumerate}
        \item $\square ABCD$ er et paralellogram: 
        Dette følger lett fra alternerende indre vinkel-teoremet, nærmere bestem fra korollar 4.4.8. 
        Vi har at $\overleftrightarrow{AB}\perp \overleftrightarrow{AD}$ og at $\overleftrightarrow{AB}\perp \overleftrightarrow{BC}$ siden $\square ABCD$ er en Lambert-firkant. 
        Siden vi åpenbart ikke har at $\overleftrightarrow{AD}=\overleftrightarrow{BC}$ gir korollar 4.4.8 oss at $\overleftrightarrow{AD}\parallel \overleftrightarrow{AB}$. 
        
        Helt tilsvarende har vi at $\overleftrightarrow{BC}\perp\overleftrightarrow{CD}$ og at $\overleftrightarrow{BC}\perp\overleftrightarrow{AB}$, slik at korollar 4.4.8  gir at $\overleftrightarrow{CD}\parallel\overleftrightarrow{AB}$. 
        Dermed er firkanten et paralellogram. 

        \item $\square ABCD$ er konveks: 
        Vi har vist at $\square ABCD$ er et paralellogram, og i forrige øving viste vi teorem 4.6.6 so sa at alle paralellogrammer en konvekse. 
        Dermed er vi ferdige. 

        \item $\mu(\angle CDA)\leq 90$: 
        Fra teorem 4.6.4 vet vi at 
        $$\mu(\angle ABC)+\mu(\angle BCD)+\mu(\angle CDA)+\mu(\angle DAB)\leq 360.$$
        Siden alle andre vinkler enn $\angle CDA$ er rette, gir dette at 
        $$270+\mu(\angle CDA)\leq 360,$$
        som vil si at $\mu(\angle CDA)\leq 90$. 

        \item $BC\leq AD$: 
        Vi antar at $BC> AD$ og viser at dette fører til en selvmotsigelse. 
        Fra teorem 3.2.23 kan vi finne et punkt $P\in \overrightarrow{BC}$ slik at $BP=AD$.
        Siden $BP=AD<BC$ per antagelse, gir korollar 3.2.18 at $B\ast P\ast C$, slik at $P\in \overline{BC}$. 

        \begin{figure}
            
        \end{figure}

        Da er $\square ABPD$ en Saccheri-firkant. 
        Fra siste del av teorem 4.8.10, altså forrige oppgave, vet vi da at $\mu(\angle BPD)\leq 90$. 
        Men $\angle BPD$ er også en ytre vinkel til trekanten $\triangle PCD$, slik at ytre vinkel-teoremet gir oss
        $$\mu(\angle BPD)>\mu(\angle PCD) =90.$$
        De to siste ligningene kan selvsagt ikke stemme samtidig, altså har vi nådd en selvmotsigelse. 
        Dermed kan ikke antagelsen vår være sann, altså har vi $BC\leq AD$. 
    \end{enumerate}
\end{oppgave}

\begin{oppgave}[4.8.10]
    La $\angle BAC$ være en spiss vinkel og la $P$ og $Q$ være to punkter på $\overrightarrow{AB}$ slik at $A\ast P\ast Q$. 
    Vi kan finne vinkelrette linjer fra $P$ og $Q$ ned på linjen $\overleftrightarrow{AC}$. 
    Kall skjæringspunktene $E$ og $F$ respektivt. Vi viser først at $QF>PE$. 

    Siden $\angle BAC$ er spiss må skjæringspunktene $P$ og $Q$ ligge på strålen $\overrightarrow{AC}$. 
    Vinkelen $\angle EPQ$ må også være stump. 
    Vi må ha en av følgende tre muligheter: $QF<PE$, $QF=PE$ eller $QF>PE$.
    Vi viser at de to første mulighetene fører til en selvmotsigelse. 
    
    Anta først at $QF=PE$. 
    Da er firkanten $\square EFQP$ en Saccheri-firkant. 
    Dermed vet vi fra teorem 4.8.10 del 6, altså oppgave 4.8.5 i denne øvingen, at vinkelen $\angle EPQ$ enten rett eller spiss. 
    Men dette motsier påstanden vår, så vi kan ikke ha at $QF=PE$.

    Anta nå at vi har $QF<PE$. Fra linjalpostulatet kan vi finne et punkt $P'$ mellom $E$ og $P$ slik at $P'E=QF$.
    Da er firkanten $\square EFQP'$ en Saccheri-firkant, som igjen vil si at vinkelen $\angle EP'Q$ er enten rett eller spiss (oppgave 4.8.5 del 6). 
    Men dette motsier ytre vinkel-teoremet, ettersom vinkelen $\angle EP'Q$ er en ytre vinkel til $\triangle PP'Q$ og $\angle EPQ$ er en indre vinkel til den samme trekanten. 
    Dermed kan vi ikke ha at $QF<PE$. 

    Den eneste muligheten som gjenstår er dermed $QF>PE$ som var det vi ville vise. 

    Vi viser nå andre del av teoremet, nemlig at for ethvert reelt tall $d_0$ så finnes et punkt $R$ på $\overrightarrow{AB}$ slik at $d(R, \overleftrightarrow{AC})=d_0>d_0$. 
    La $d_0$ være et gitt reellt tall. 
    Definer $B_0=A$ og $B_1=B$. 
    Vi kan bruke linjalpostulatet til å finne punkter $B_2, B_3, \ldots$ på $\overrightarrow{AB}$ slik at for hver $i\geq 1$ har vi $B_0\ast B_i\ast B_{i+1}$ og $B_i B_{i+1}=B_0 B_i$. 
    Vi bruker oppgave 4.8.9\footnote{Det burde kanskje skrives ned et kjapt bevis på denne oppgaven for å bruke den...} som gir oss $d(B_{i+1}, \overleftrightarrow{AC})\geq 2 d(B_i, \overleftrightarrow{AC})$. 
    Det følger ved matematisk induksjon at vi har $d(B_n, \overleftrightarrow{AC})=2^{n-1}d(B_1, \overleftrightarrow{AC})$ for hver $n\geq 0$. 

    Fra Arkimedes' aksiom for de reelle tallene kan vi finne et naturlig tall $k$ slik at $2^{k-1}d(B_1, \overleftrightarrow{AC})\geq d_0$. 
    Punktet $R=B_k$ har de egenskapene vi er ute etter. 
\end{oppgave}

\begin{oppgave}[5.1.2]
    Vi skal vise teorem 5.1.10. 
    La $\square ABCD$ være et paralellogram, altså at $\overleftrightarrow{AB}\parallel\overleftrightarrow{CD}$ og $\overleftrightarrow{AD}\parallel\overleftrightarrow{BC}$. 
    Vi skal vise 4 utsagn. 
    \begin{enumerate}
        \item $\triangle ABC\cong \triangle CDA$ og $\triangle ABD\cong \triangle CBD$: 
        Vi viser kun $\triangle ABC\cong \triangle CDA$, da den andre kongruensen vises på akkuratt samme måte. 
        Vi vet at $\overleftrightarrow{AD}\parallel\overleftrightarrow{BC}$, og at disse linjene skjæres av $\overleftrightarrow{AC}$ i henholdsvis $A$ og $C$. 
        Det motsatte alternerende indre vinkel-teoremet (MAIVT, teorem 5.1.1) gir derfor at 
        $$\angle DAC\cong \angle BCA.$$
        På tilsvarende vis vet vi at $\overleftrightarrow{AB}\parallel\overleftrightarrow{CD}$ og at disse to linjene skjæres av $\overleftrightarrow{AC}$ i henholdsvis $A$ og $C$. 
        Igjen gir MAIVT dermed at 
        $$\angle CAB\cong \angle ACD.$$
        Vi vet nå at $\angle DAC\cong \angle BCA$, $\angle CAB\cong\angle ACD$ og $\overline{AC}\cong\overline{CA}$. 
        Side-vinkel-side-postulatet (VSV) gir oss dermed at $\triangle ABC\cong \triangle CDA$. 

        \item $\overline{AB}\cong \overline{CD}$ og $\overline{BC}\cong\overline{AD}$: 
        Fra det første punktet i oppgaven vet vi at $\triangle ABC\cong \triangle CDA$. 
        Da må vi spesielt ha $\overline{AB}\cong \overline{CD}$. 
        På tilsvarende vis får vi $\overline{BC}\cong \overline{AD}$ fordi $\triangle ABD\cong \triangle CBD$. 

        \item $\angle DAB\cong\angle BCD$ og $\angle ABC\cong\angle CDA$: 
        Som i punkt 2 følger dette punktet direkte fra kongruensene i punkt 1. 

        \item Diagonalene $\overline{AC}$ og $\overline{BD}$ skjærer hverandre i et punkt $P$ slik at $AP=PC$ og $BP=PD$: 
        Vi vet fra forrige øving (teorem 4.6.6) at $\square ABCD$ er konveks. 
        Videre vet vi fra teorem 4.6.8 at diagonalene i en konveks firkant skjærer hverandre i det indre, og vi vet dermed at $\overline{AC}$ og $\overline{BD}$ skjærer hverandre i et punkt $P$. 
        Vi må vise at $AP=PC$ og $BP=PD$. 

        Vi vet at $\overleftrightarrow{AB}\parallel\overleftrightarrow{CD}$ og at disse to linjene skjæres av $\overleftrightarrow{AC}$ i henholdvis $A$ og $C$. 
        MAIVT gir oss dermed at 
        $$\angle PAB\cong \angle PCD.$$
        De samme linjene skjæres også av $\overleftrightarrow{BD}$ i henholdsvis $B$ og $D$, og MAIVT gir igjen at 
        $$\angle ABP\cong \angle PDC.$$
        Fra punkt 2 i denne oppgaven vet vi også at $\overline{AB}\cong \overline{CD}$, 
        og disse tre kongruensene lar oss bruke VSV (vinkel-side-vinkel) til å konkludere med at 
        $$\triangle ABP\cong \triangle CDP,$$
        noe som spesielt betyr at $AP=PC$ og $BP=PD$.  
    \end{enumerate}
\end{oppgave}

\begin{oppgave}[5.1.3]
    La $\square ABCD$ være en firkant slik at $\overleftrightarrow{AB}\parallel\overleftrightarrow{CD}$ og $\overline{AB}\cong \overline{CD}$. 
    Vi skal vise at $\square ABCD$ er et paralellogram, altså at vi også har $\overleftrightarrow{AD}\parallel\overleftrightarrow{BC}$. 
    Linjen $\overleftrightarrow{AC}$ skjærer de to paralelle linjene $\overleftrightarrow{AB}$ og $\overleftrightarrow{CD}$ i henholdsvis $A$ og $C$. 
    Det motsatte alternerende indre vinkel-teoremet (MAIVT) gir oss da at 
    $$\angle CAB\cong \angle ACD.$$
    Nå har vi at $\overline{AC}\cong \overline{AC}$, $\overline{AB}\cong \overline{CD}$ og $\angle CAB\cong \angle ACD$. 
    Fra SVS (side-vinkel-side) får vi da at $\triangle ABC\cong\triangle CDA$, noe som spesielt betyr at $\angle DAC\cong \angle BCA$. 
    Men dette betyr at $\overleftrightarrow{AC}$ skjærer linjene $\overleftrightarrow{AD}$ og $\overleftrightarrow{BC}$ i punktene $A$ og $C$, slik at $\angle DAC\cong \angle BCA$. 
    Det alternerende indre vinkel-teoremet (AIVT) gir oss dermed at $\overleftrightarrow{AD}\parallel\overleftrightarrow{BC}$, som var det vi ville vise. 
\end{oppgave}