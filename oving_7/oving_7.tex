

\begin{oppgave}[4.7.2]
    Anta først at det euklidske parallellpostulatet holder. 
    Vi vil vise at dersom $l \parallel m$, og en linje $t\neq l$ skjærer $l$, så må $t$ også skjære $m$. 
    Dette kalles ofte Proclus' aksiom. 
    Vi bruker et bevis med selvmotsigelse, og antar derfor at $t$ skjærer $l$, men at $t$ \emph{ikke} skjærer $m$. 
    Med andre ord betyr dette at $t\parallel m$. 

    La $P$ være skjæringspunktet mellom $t$ og $l$. 
    Da er $t$ og $l$ to linjer som begge går gjennom $P$ og er paralelle til $m$. 
    Dette motsier det euklidske parallellpostulatet. 
    Dermed kan vi konkludere med at $t$ skjærer $m$, som var det vi ville vise. 
    
    Anto nå at Proclus' aksiom holder, og at vi har en linje $l$ og et punkt $P$ som ikke ligger på $l$. 
    Vi må vise at det finnes en unik linje $m$ som inneholder $P$ og som er parallell med $l$. 
    Fra teorem 4.1.3 vet vi at vi kan finne et unikt punkt $A$ slik at $\overleftrightarrow{PA}\perp l$. 
    Fra del 3 av gradskivepostulatet kan vi finne et punkt $Q$ slik at $\mu(\angle APQ)=90$. 
    Merk at vi må egentlig spesifisere et halvplan for å bruke denne påstanden, 
    men i dette tilfelle har det ikke noe å si hvilket halvplan vi velger. 
    Det alternerende indre vinkel-teoremet gir oss at $\overleftrightarrow{PQ}\parallel l$, 
    siden $\overleftrightarrow{PA}$ skjærer både $l$ og $\overleftrightarrow{PQ}$ på en slik måte at alle indre vinkler er rette. 
    Dermed har vi funnet en linje $m=\overleftrightarrow{PQ}$ slik at $P\in m$ og $m\parallel l$. 
    Merk her at vi ikke har brukt hverken det euklidske parallellpostulatet eller Proclus' aksiom, så denne konstruksjonen av en parallell linje er fungerer derfor i nøytral geometri.
    Konstruksjonen kalles ofte for den doble perpendikulær konstruksjonen. 

    Det gjenstår å vise at denne linja er unik. 
    Anta defor at vi har en linje $n\neq m$ som er paralell med $l$ og har $P\in n$. 
    Dermed skjærer $n$ linjen $m$ i punktet $P$.  
    Proclus' aksiom sier oss da at $n$ også må skjære $l$, noe som motsier at $n$ og $l$ er paralelle. 
    Dermed er $m$ unik, og beviset er fullført. 
\end{oppgave}

\begin{oppgave}[4.7.6]

\end{oppgave}

\begin{oppgave}[4.8.1]

\end{oppgave}

\begin{oppgave}[4.8.2]

\end{oppgave}

\begin{oppgave}[4.8.5]

\end{oppgave}

\begin{oppgave}[4.8.8]

\end{oppgave}

\begin{oppgave}[4.8.10]

\end{oppgave}

\begin{oppgave}[5.1.2]

\end{oppgave}

\begin{oppgave}[5.1.3]

\end{oppgave}