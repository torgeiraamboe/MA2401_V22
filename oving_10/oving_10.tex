
\begin{oppgave}[6.1.4]
    La $\square ABCD$ vær en Saccheri-firkant med base $\overline{AB}$. 
    En Saccheri-firkant er et paralellogram, men vi viser at alle konklusjonene i teorem 5.1.10 er usanne for firkanten $\square ABCD$ i hyperbolsk geometri. 
    \begin{enumerate}
        \item $\triangle ABC \ncong \triangle CDA$ fordi $AB\neq CD$ fra korrolar 6.1.10. 
        \item Fra korollar 6.1.10 vet vi også at motsatte sidene $\overline{AB}$ og $\overline{CD}$ er ikke kongruente. 
        \item Fra korollar 6.1.4 vet vi at $\angle ABC \ncong \angle CDA$.
        \item La $E$ være punktet der de to diagonalene skjærer hverandre. 
        Dette punktet vet vi eksisterer ved teorem 4.6.8. 
        Anta nå at $AE=EC$ og $BE=ED$.
        Da er $\triangle AEB \cong \triangle CED$ fra vertikale vinkler teoremet og side-vinkel-side postulatet. 
        Men, dette impliserer at vi har $AB=CD$, noe som motsier korollar 6.1.10. 
    \end{enumerate}
\end{oppgave}

\begin{oppgave}[6.1.5]
    La $\triangle ABC$ være en rettvinklet trekant med rett vinkel i $C$. 
    La $M$ være midtpunktet på $\overline{AC}$ og $N$ være midtpunktet på $\overline{BC}$. 
    Vi nedfeller normaler fra hjørnene i $\triangle ABC$ til $\overleftrightarrow{MN}$ og kaller føttene $D$, $E$ og $F$. 

    \begin{punkt}
        Situasjonen er vist i følgende figur: 

        \begin{figure}[H]
            \centering
            
\definecolor{qqwuqq}{rgb}{0,0.39215686274509803,0}
\definecolor{qqqqff}{rgb}{0,0,1}
\begin{tikzpicture}[line cap=round,line join=round,>=triangle 45,x=1.1cm,y=1.1cm]
\clip(-4,-2.5) rectangle (6.5,3);
%\draw[line width=2pt,color=qqwuqq,fill=qqwuqq,fill opacity=0.1] (0.17887107180558842,1.7615052375925486) -- (0.05962369060186289,1.5826341657869603) -- (0.2384947624074513,1.8211289281944116) -- (0,2) -- cycle; 
\draw[line width=2pt,color=qqwuqq,fill=qqwuqq,fill opacity=0.1] (-0.17887107180558842,1.7615052375925486) -- (0.05962369060186289,1.5826341657869603) -- (0.2384947624074513,1.8211289281944116) -- (0,2) -- cycle; 
\draw[line width=2pt,color=qqwuqq,fill=qqwuqq,fill opacity=0.1] (-2.997931412455465,-0.29629992407600797) -- (-2.6998131791916093,-0.2966618913452729) -- (-2.6994512119223444,0.00145634191858267) -- (-2.9975694451862,0.0018183091878476267) -- cycle; 
\draw[line width=2pt,color=qqwuqq,fill=qqwuqq,fill opacity=0.1] (0.2956876784500551,-0.002180276457112524) -- (0.29604964571931996,0.29593795680674306) -- (-0.0020685875445355874,0.29629992407600797) -- (-0.002430554813800503,-0.001818309187847627) -- cycle; 
\draw[line width=2pt,color=qqwuqq,fill=qqwuqq,fill opacity=0.1] (5.05,-0.3064344436253554) -- (5.349276580879597,-0.3067964108946202) -- (5.349276580879597,-0.008678177630764604) -- (5.05,-0.00831621036149978) -- cycle; 
\draw [line width=2pt] (-3,-2)-- (0,2);
\draw [line width=2pt] (-3,-2)-- (5.346846026065797,-2.0101345195493474);
\draw [line width=2pt] (0,2)-- (2.6734230130328984,-0.005067259774673705);
\draw [line width=2pt] (2.6734230130328984,-0.005067259774673705)-- (5.346846026065797,-2.0101345195493474);
\draw [line width=2pt,domain=-6.440882646808224:8.174693547092996] plot(\x,{(-0.007600889662010557-0.005067259774673705*\x)/4.173423013032899});
\draw [line width=2pt] (-2.9975694451861994,0.001818309187847627)-- (-3,-2);
\draw [line width=2pt] (-0.0024305548138005024,-0.001818309187847627)-- (0,2);
\draw [line width=2pt] (5.349276580879597,-0.00831621036149978)-- (5.346846026065797,-2.0101345195493474);
\draw [fill=qqqqff] (0,2) circle (2.5pt);
\draw[color=qqqqff] (0.22044727233521633,2.3060707093168884) node {$C$};
\draw [fill=qqqqff] (-3,-2) circle (2.5pt);
\draw[color=qqqqff] (-3.3,-1.6991593053579692) node {$A$};
\draw [fill=qqqqff] (-1.5,0) circle (2pt);
\draw[color=qqqqff] (-1.7,0.2683221053595047) node {$M$};
\draw [fill=qqqqff] (5.346846026065797,-2.0101345195493474) circle (2.5pt);
\draw[color=qqqqff] (5.574807397216336,-1.713212744005951) node {$B$};
\draw [fill=qqqqff] (2.6734230130328984,-0.005067259774673705) circle (2pt);
\draw[color=qqqqff] (2.9046540540997676,0.2683221053595047) node {$N$};
\draw [fill=qqqqff] (-2.9975694451861994,0.001818309187847627) circle (2pt);
\draw[color=qqqqff] (-2.772935159684937,0.28237554400748666) node {$D$};
\draw [fill=qqqqff] (-0.0024305548138005024,-0.001818309187847627) circle (2pt);
\draw[color=qqqqff] (-0.3,0.2683221053595047) node {$F$};
\draw [fill=qqqqff] (5.349276580879597,-0.00831621036149978) circle (2pt);
\draw[color=qqqqff] (5.574807397216336,0.2683221053595047) node {$E$};
\end{tikzpicture} 
        \end{figure}
    \end{punkt}

    \begin{punkt}
        Siden $\mu(\angle MCN)=90$, og vinkelsummen i trekanten er mindre enn $180$, må både $\angle CMN$ og $\angle MNC$ være spisse. 
        Vi kan derfor bruke lemma 4.8.6 til å konkludere med at $M\ast F\ast N$. 
        La oss vise at $\triangle BEN\cong \triangle CFN$. 
        Fra teorem 3.5.13 vet vi at $\angle BNE \cong \angle CNF$ siden de er toppvinkler. 
        Ettersom $N$ er midtpunktet på $\overline{BC}$, må vi ha $BN=NC$.
        Vi vet også at $\angle BEN$ og $\angle CFN$ er rette vinkler. 
        Dermed gir vinkel-vinkel-side (VVS) oss at $\triangle BEN \cong \triangle CFN$. 
        Et helt tilsvarende bevis gir også at $\triangle ADM \cong \triangle CFM$. 

        \begin{figure}[H]
            \centering
            
\definecolor{qqwuqq}{rgb}{0,0.39215686274509803,0}
\definecolor{qqqqff}{rgb}{0,0,1}
\begin{tikzpicture}[line cap=round,line join=round,>=triangle 45,x=1.2cm,y=1.2cm]
\clip(-1,-0.3) rectangle (6,5);
\draw[line width=2pt,color=qqwuqq,fill=qqwuqq,fill opacity=0.10000000149011612] (-0.175252336908348,3.7663302174555358) -- (0.058417445636116075,3.591077880547188) -- (0.23366978254446402,3.824747663091652) -- (0,4) -- cycle; 
\draw[line width=2pt,color=qqwuqq,fill=qqwuqq,fill opacity=0.10000000149011612] (-2.70791277181942,2) -- (-2.70791277181942,2.2920872281805797) -- (-3,2.29208722818058) -- (-3,2) -- cycle; 
\draw[line width=2pt,color=qqwuqq,fill=qqwuqq,fill opacity=0.10000000149011612] (5.333333333333333,2) -- (5.33333333333333,1.7079127718194198) -- (5.0333333333333333,1.7079127718194198) -- (5.033333333333333,2) -- cycle; 
\draw[line width=2pt,color=qqwuqq,fill=qqwuqq,fill opacity=0.10000000149011612] (0,2.2920872281805797) -- (0.29208722818057997,2.2920872281805797) -- (0.29208722818058,2) -- (0,2) -- cycle; 
\draw [shift={(2.6666666666666665,2)},line width=2pt,color=qqwuqq,fill=qqwuqq,fill opacity=0.10000000149011612] (0,0) -- (-36.86989764584402:-0.41307371948894117) arc (-36.86989764584402:0:-0.41307371948894117) -- cycle;
\draw [shift={(2.6666666666666665,2)},line width=2pt,color=qqwuqq,fill=qqwuqq,fill opacity=0.10000000149011612] (0,0) -- (-36.86989764584402:0.41307371948894117) arc (-36.86989764584402:0:0.41307371948894117) -- cycle;
\draw [line width=2pt] (-3,0)-- (0,4);
\draw [line width=2pt] (5.333333333333333,0)-- (-3,0);
\draw [line width=2pt,domain=-6.364079332225208:7.955809610058086] plot(\x,{(--8.333333333333332-0*\x)/4.166666666666666});
\draw [line width=2pt] (-3,0)-- (-3,2);
\draw [line width=2pt] (5.333333333333333,0)-- (5.333333333333333,2);
\draw [line width=2pt] (0,4)-- (0,2);
\draw [shift={(2.6666666666666665,2)},line width=2pt,color=qqwuqq] (-36.86989764584402:0.41307371948894117) arc (-36.86989764584402:0:0.41307371948894117);
\draw[line width=2pt,color=qqwuqq] (2.195582878915828,2.16702792925028) -- (2.34527322527274,2.1030464480464646);
\draw [shift={(2.6666666666666665,2)},line width=2pt,color=qqwuqq] (-36.86989764584402:0.41307371948894117) arc (-36.86989764584402:0:0.41307371948894117);
\draw[line width=2pt,color=qqwuqq] (3.0095582878915828,1.885702792925028) -- (3.107527322527274,1.8530464480464646);
\draw [line width=2pt] (0,4)-- (2.6666666666666665,2);
\draw [line width=2pt] (1.3829021796720073,3.066091795118231) -- (1.283764486994661,2.933908204881769);
\draw [line width=2pt] (2.6666666666666665,2)-- (5.333333333333333,0);
\draw [line width=2pt] (4.049568846338673,1.066091795118231) -- (3.950431153661326,0.9339082048817692);
\draw [fill=qqqqff] (-3,0) circle (2.5pt);
\draw[color=qqqqff] (-2.784107096654384,0.29315527527798785) node {$A$};
\draw [fill=qqqqff] (0,4) circle (2.5pt);
\draw[color=qqqqff] (0.2175619316319216,4.299970354320723) node {$C$};
\draw [fill=qqqqff] (5.333333333333333,0) circle (2pt);
\draw[color=qqqqff] (5.559982037022228,0.2656170273120584) node {$B$};
\draw [fill=qqqqff] (-1.5,2) circle (2pt);
\draw[color=qqqqff] (-1.2832725825112312,2.2621400048419433) node {$M$};
\draw [fill=qqqqff] (2.6666666666666665,2) circle (2pt);
\draw[color=qqqqff] (2.8887719843270743,2.2621400048419433) node {$N$};
\draw [fill=qqqqff] (-3,2) circle (2pt);
\draw[color=qqqqff] (-2.784107096654384,2.2621400048419433) node {$D$};
\draw [fill=qqqqff] (5.333333333333333,2) circle (2pt);
\draw[color=qqqqff] (5.559982037022228,2.2621400048419433) node {$E$};
\draw [fill=qqqqff] (0,2) circle (2pt);
\draw[color=qqqqff] (-0.2175619316319216,2.2621400048419433) node {$F$};
\end{tikzpicture} 
        \end{figure}
    \end{punkt} 

    \begin{punkt}
        Av forrige deloppgave vet vi at $DM=MF$ siden $\triangle ADM \cong \triangle CFM$, og at $FN=NE$ siden $\triangle BEN \cong \triangle CFN$. 
        Dermed er 
        $$DE=DM+MF+FN+NE=2(MF+FN)=2MN.$$

        Vi påstår at $\square EDAB$ er en Saccheri-firkant. 
        Det er klart at vinklene $\angle EDA$ og $\angle BED$ er rette.
        Siden $\triangle ADM \cong CFM$, er $AD=CF$ og siden $\triangle BEN \cong \triangle CFN$ er $CF=BE$. 
        Dermed er $BE=AD$, slik at $\square ABCD$ er en Saccheri-firkant. 

        Korollar 6.1.10 gir oss dermed at $DE<AB$, og ettersom $MN=\frac{1}{2}DE$ er $MN<\frac{1}{2}AB$. 
    \end{punkt}

    \begin{punkt}
        Anta at Pythagoras teorem holder for $\triangle MNC$. 
        Da er 
        $$MN^2=MC^2+NC^2.$$
        Hvis vi setter inn $MC=\frac{1}{2}AC$ og $NC=\frac{1}{2}BC$ i denne ligningen, får vi at 
        $$MN^2 = \frac{1}{4}AC^2+\frac{1}{4}BC^2,$$
        og hvis vi ganger begge sidene med $4$, får vi 
        $$(2MN)^2=AC^2+BC^2.$$
        Men, vi vet fra forrige deloppgave at $2MN<AB$, som i lys av forrige ligning må bety at 
        $$AC^2+BC^2= (2MN)^2<AB^2.$$
        Dermed kan ikke Pythagoras teorem holde for trekanten $\triangle ABC$, ettersom vi da måtte hatt $AB^2=AC^2+BC^2$.
    \end{punkt}
\end{oppgave}

\begin{oppgave}[6.2.1]
    La $l$ og $m$ være to linjer slik at $l\parallel m$, og la $P$ og $Q$ være to punkter på $m$ som ligger like langt fra $l$. 
    Vi vil vise at $l$ og $m$ har en felles normal linje. 

    Vi nedfeller vinkelrette linjer fra $P$ og $Q$ til $l$ og kaller føttene $R$ og $S$ respektivt.
    Siden $m\parallel l$ vet vi at $R$ og $S$ er på samme side av $l$. 
    Dermed er $\square RSQP$ en Saccheri-firkant. 
    La $M$ være midtpunktet på $\overline{RS}$ og la $N$ væøre midtpunktet på $\overline{PQ}$. 
    Da får vi fra teorem 4.8.10 del 3 at $\overleftrightarrow{MN}$ er en felles normal linje for $l$ og $m$. 

    \begin{figure}[H]
        \centering
        
\definecolor{qqwuqq}{rgb}{0,0.39215686274509803,0}
\definecolor{qqqqff}{rgb}{0,0,1}
\begin{tikzpicture}[line cap=round,line join=round,>=triangle 45,x=1cm,y=1cm]
\clip(-6,-1) rectangle (5,4);
\draw[line width=2pt,color=qqwuqq,fill=qqwuqq,fill opacity=0.10000000149011612] (0,2.5757359312880714) -- (0.42426406871192845,2.5757359312880714) -- (0.4242640687119284,3) -- (0,3) -- cycle; 
\draw[line width=2pt,color=qqwuqq,fill=qqwuqq,fill opacity=0.10000000149011612] (0,0.4242640687119284) -- (0.42426406871192834,0.4242640687119285) -- (0.4242640687119284,0) -- (0,0) -- cycle; 
\draw[line width=2pt,color=qqwuqq,fill=qqwuqq,fill opacity=0.10000000149011612] (-3,0.4242640687119284) -- (-2.5757359312880714,0.4242640687119284) -- (-2.5757359312880714,0) -- (-3,0) -- cycle; 
\draw[line width=2pt,color=qqwuqq,fill=qqwuqq,fill opacity=0.10000000149011612] (3,0.4242640687119284) -- (3.4242640687119286,0.4242640687119284) -- (3.4242640687119286,0) -- (3,0) -- cycle; 
\draw [line width=2pt,domain=-6.12:22.68] plot(\x,{(--18-0*\x)/6});
\draw [line width=2pt,domain=-6.12:22.68] plot(\x,{(-0-0*\x)/6});
\draw [line width=2pt] (-3,3)-- (-3,0);
\draw [line width=2pt] (-2.88,1.5) -- (-3.12,1.5);
\draw [line width=2pt] (0,3)-- (0,0);
\draw [line width=2pt] (3,3)-- (3,0);
\draw [line width=2pt] (3.12,1.5) -- (2.88,1.5);
\draw [fill=qqqqff] (-3,0) circle (2.5pt);
\draw[color=qqqqff] (-2.68,-0.43) node {$R$};
\draw [fill=qqqqff] (3,0) circle (2.5pt);
\draw[color=qqqqff] (3.32,-0.43) node {$S$};
\draw [fill=qqqqff] (-3,3) circle (2.5pt);
\draw[color=qqqqff] (-2.68,3.43) node {$P$};
\draw [fill=qqqqff] (3,3) circle (2.5pt);
\draw[color=qqqqff] (3.32,3.43) node {$Q$};
\draw[color=black] (-4.18,2.73) node {$m$};
\draw[color=black] (-4.26,-0.33) node {$l$};
\draw [fill=qqqqff] (0,3) circle (2pt);
\draw[color=qqqqff] (0.32,3.39) node {$N$};
\draw [fill=qqqqff] (0,0) circle (2pt);
\draw[color=qqqqff] (0.32,-0.43) node {$M$};
\end{tikzpicture} 
    \end{figure}
\end{oppgave}

\begin{oppgave}[6.2.2]
    La $l$ og $m$ være paralelle linjer som har en felles normal. 
    Vi ønsker å vise at denne linjen er unik. 

    Anta at det finnes to ulike felles normaler $t$ og $s$. 
    Fra indre vinkel-teoremet vet vi at $t\parallel s$. 
    La $A$, $B$, $C$ og $D$ være de fire skjæringspunktene mellom linjene $l$, $m$, $t$ og $s$, slik som vist på følgende figur: 

    \begin{figure}[H]
        \centering
        
\definecolor{qqwuqq}{rgb}{0,0.39215686274509803,0}
\definecolor{qqqqff}{rgb}{0,0,1}
\begin{tikzpicture}[line cap=round,line join=round,>=triangle 45,x=1cm,y=1cm]
\clip(-5,-1) rectangle (5,5);
\draw[line width=2pt,color=qqwuqq,fill=qqwuqq,fill opacity=0.10000000149011612] (-3,2.5757359312880714) -- (-2.5757359312880714,2.5757359312880714) -- (-2.5757359312880714,3) -- (-3,3) -- cycle; 
\draw[line width=2pt,color=qqwuqq,fill=qqwuqq,fill opacity=0.10000000149011612] (3,2.5757359312880714) -- (2.6,2.5757359312880714) -- (2.6,3) -- (3,3) -- cycle; 
\draw[line width=2pt,color=qqwuqq,fill=qqwuqq,fill opacity=0.10000000149011612] (3,0.4242640687119284) -- (2.6,0.4242640687119285) -- (2.6,0) -- (3,0) -- cycle; 
\draw[line width=2pt,color=qqwuqq,fill=qqwuqq,fill opacity=0.10000000149011612] (-3,0.4242640687119284) -- (-2.5757359312880714,0.4242640687119284) -- (-2.5757359312880714,0) -- (-3,0) -- cycle; 
\draw [line width=2pt,domain=-6.12:14.84] plot(\x,{(--18-0*\x)/6});
\draw [line width=2pt,domain=-6.12:14.84] plot(\x,{(-0-0*\x)/6});
\draw [line width=2pt] (-3,-9.6) -- (-3,4.58);
\draw [line width=2pt] (3,-9.6) -- (3,4.58);
\draw [fill=qqqqff] (-3,0) circle (2.5pt);
\draw[color=qqqqff] (-3.4,0.43) node {$A$};
\draw [fill=qqqqff] (3,0) circle (2.5pt);
\draw[color=qqqqff] (3.32,0.43) node {$B$};
\draw [fill=qqqqff] (-3,3) circle (2.5pt);
\draw[color=qqqqff] (-2.68,3.43) node {$D$};
\draw [fill=qqqqff] (3,3) circle (2.5pt);
\draw[color=qqqqff] (3.32,3.43) node {$C$};
\draw[color=black] (-4.18,2.73) node {$m$};
\draw[color=black] (-4.26,-0.31) node {$l$};
\draw[color=black] (-2.7,4.25) node {$t$};
\draw[color=black] (3.3,4.31) node {$s$};
\end{tikzpicture} 
    \end{figure}
    
    Firkanten $\square ABCD$ er da et rektangel. 
    Men dette motsier teorem 6.1.6, som viser at vi ikke kan ha to ulike normaler for linjene $l$ og $m$. 
\end{oppgave}