
\begin{oppgave}[8.1.1]
    La $t$ være en linje og la $\gamma = C(O, r)$ være en sirkel slik at $t$ og $\gamma$ skjærer hverandre i punktet $P$. 
    Vi antar at $\overleftrightarrow{OP}\perp t$, og ønsker å vise at $t$ er tangenten til $\gamma$ i $P$. 

    La $Q$ være et punkt på $t$ slik at $Q\neq P$. 
    Siden $P$ er foten til normalen fra $O$ til $t$, har vi fra teorem 4.3.4 at $OP<OQ$.
    Dermed kan ikke $Q$ ligge på $\gamma$, og vi får $t\cap\gamma = \{P\}$, som vil si at $t$ er tangenten til $\gamma$ i $P$. 
    
    \begin{figure}[H]
        \centering
        
\definecolor{qqqqff}{rgb}{0,0,1}
\begin{tikzpicture}[line cap=round,line join=round,>=triangle 45,x=1cm,y=1cm]
\clip(-5,-4.5) rectangle (5,3.5);
\draw [line width=2pt] (0,0) circle (3.1622776601683795cm);
\draw [line width=2pt,domain=-15.92:14.48] plot(\x,{(--10-1*\x)/-3});
\draw [line width=2pt] (0,0)-- (1,-3);
\draw [line width=2pt] (0,0)-- (3.115,-2.295);
\draw [fill=qqqqff] (0,0) circle (2pt);
\draw[color=qqqqff] (0.32,0.38) node {$O$};
\draw [fill=qqqqff] (1,-3) circle (2.5pt);
\draw[color=qqqqff] (1.26,-2.44) node {$P$};
\draw[color=black] (-14.1,-7.56) node {$t$};
\draw [fill=qqqqff] (3.115,-2.295) circle (2.5pt);
\draw[color=qqqqff] (3.26,-1.76) node {$Q$};
\end{tikzpicture}
 
    \end{figure}
\end{oppgave}

\begin{oppgave}[8.1.2]
    Dette følger direkte fra teorem 4.3.4. Se løsningen over for oppgave 8.1.1.
\end{oppgave}

\begin{oppgave}[8.1.3]
    La $\gamma = C(O,r)$ være en sirkel, og la $l$ være sekanten som skjærer $\gamma$ i de to ulike punktene $P$ og $Q$.
    Vi ønsker å vise at $O$ ligger på midtnormalen til korden $\overline{PQ}$. 
    
    Siden $P$ og $Q$ begge ligger på $\gamma$ må vi ha $OP=r=OQ$.
    Vi kan nå anvende teorem 4.3.7 om punktvis karakterisering av midtnormaler for å konkludere med at $O$ ligger på midtnormalen til $\overline{PQ}$.  
\end{oppgave}

\begin{oppgave}[8.1.6]
    La $\gamma=C(O,r)$ være en sirkel, og $l$ og $m$ to ikke-paralelle tangenter til $\gamma$ i punktene $P$ og $Q$ respektivt. 
    Ettersom $l$ og $m$ ikke er paralelle skjærer de hverandre i et punkt, $A$. 

    \begin{punkt}
        Fra definisjonen av vinkelhalveringsstråle (definisjon 3.4.6) må vi vise at $O$ ligger i det indre av $\angle PAQ$ og at $\angle PAO \cong \angle QAO$. 
        \begin{figure}[H]
            \centering
            
\definecolor{qqqqff}{rgb}{0,0,1}
\begin{tikzpicture}[line cap=round,line join=round,>=triangle 45,x=0.8cm,y=0.8cm]
\clip(-4,-4) rectangle (12,4);
\draw [line width=2pt] (0,0) circle (0.8*3.1622776601683795cm);
\draw [line width=2pt,domain=-12.94:25.46] plot(\x,{(--10-1*\x)/-3});
\draw [line width=2pt] (0,0)-- (1,-3);
\draw [line width=2pt] (0.613841995766061,-1.4620526680779784) -- (0.38615800423393937,-1.5379473319220205);
\draw [line width=2pt,domain=-12.94:25.46] plot(\x,{(--10-1*\x)/3});
\draw [line width=2pt] (0,0)-- (1,3);
%\draw [line width=2pt] (0,0)-- (10,0);
\draw [line width=2pt] (0.38615800423393937,1.5379473319220205) -- (0.613841995766061,1.4620526680779784);
\draw [fill=qqqqff] (0,0) circle (2pt);
\draw[color=qqqqff] (-0.5,0.2) node {$O$};
\draw [fill=qqqqff] (1,-3) circle (2.5pt);
\draw[color=qqqqff] (1.26,-2.42) node {$Q$};
\draw[color=black] (6,-0.9) node {$m$};
\draw [fill=qqqqff] (1,3) circle (2.5pt);
\draw[color=qqqqff] (1.32,3.42) node {$P$};
\draw[color=black] (6,1.8) node {$l$};
\draw [fill=qqqqff] (10,0) circle (2pt);
\draw[color=qqqqff] (10.12,0.6) node {$A$};
\end{tikzpicture}
 
        \end{figure}
        For å vise den første påstanden viser vi at $\overline{OP}\cap m = \emptyset$. 

        Merk at vi ikke kan ha $P\in \overline{OP}\cap m$, ettersom da ville både $A$ og $P$ vært felles punkter for $l$ og $m$. 
        Hvis $A\in \overline{OP}$, har vi fra definisjonen av linjestykke at $r=OP=OA+AP$, slik at $OA\leq r$. 
        Dersom $A\in m$ og $A\neq P$, sier teorem 8.1.8. at $A$ ligger utenfor $\gamma$, altså at $OA > r$. 
        Dermed ser vi at vi ikke kan ha $A\in \overline{OP}\cap m$, siden dette ville implisert både $OA\leq r$ og $OA>r$. 
        Dermed må vi ha $\overline{OP}\cap m = \emptyset$. 
        På tilsvarende måte får vi at $\overline{OQ}\cap l=\emptyset$. 
        Dette betyr at $O$ ligger i det indre av $\angle PAQ$. 

        For å vise at $\angle PAO\cong \angle QAO$ bruker vi først at både $\angle OPA$ og $\angle OQA$ er rette vinkler fra tangentlinjeteoremet (teorem 8.1.7). 
        Dermed er $P$ foten til normalen fra $O$ til $l$, og $Q$ er foten til normaln fra $O$ til $m$. 
        Dette betyr at $r=OP=d(O,l)$ og $r=OQ=d(O,m)$. 
        Punitvis karakterisering av vinkelhalveringsstråle (teorem 4.3.6) gir dermed at $\angle PAO\cong \angle QAO$. 


    \end{punkt}

    \begin{punkt}
        Betrakt trekantene $\triangle OPA$ og $\triangle OQA$.
        \begin{figure}[H]
            \centering
            
\definecolor{qqqqff}{rgb}{0,0,1}
\begin{tikzpicture}[line cap=round,line join=round,>=triangle 45,x=0.8cm,y=0.8cm]
\clip(-4,-4) rectangle (12,4);
\draw [line width=2pt] (0,0) circle (0.8*3.1622776601683795cm);
\draw [line width=2pt,domain=-12.94:25.46] plot(\x,{(--10-1*\x)/-3});
\draw [line width=2pt] (0,0)-- (1,-3);
\draw [line width=2pt] (0.613841995766061,-1.4620526680779784) -- (0.38615800423393937,-1.5379473319220205);
\draw [line width=2pt,domain=-12.94:25.46] plot(\x,{(--10-1*\x)/3});
\draw [line width=2pt] (0,0)-- (1,3);
\draw [line width=2pt] (0,0)-- (10,0);
\draw [line width=2pt] (0.38615800423393937,1.5379473319220205) -- (0.613841995766061,1.4620526680779784);
\draw [fill=qqqqff] (0,0) circle (2pt);
\draw[color=qqqqff] (-0.5,0.2) node {$O$};
\draw [fill=qqqqff] (1,-3) circle (2.5pt);
\draw[color=qqqqff] (1.26,-2.42) node {$Q$};
\draw[color=black] (6,-0.9) node {$m$};
\draw [fill=qqqqff] (1,3) circle (2.5pt);
\draw[color=qqqqff] (1.32,3.42) node {$P$};
\draw[color=black] (6,1.8) node {$l$};
\draw [fill=qqqqff] (10,0) circle (2pt);
\draw[color=qqqqff] (10.12,0.6) node {$A$};
\end{tikzpicture}
 
        \end{figure}
        Vi vet at 
        \begin{itemize}
            \item $\angle PAO\cong \angle QAO$ fra forrige deloppgave,
            \item $\angle OPA\cong \angle OQA$ siden begge er rette vinkler,
            \item $OA$ er en felles side for trekantene. 
        \end{itemize}

        Dermed gir vinkel-vinkel-side (VVS) oss at $\triangle OPA\cong \triangle OQA$, som spesielt betyr at $PA=QA$. 
    \end{punkt}

    \begin{punkt}
        Siden $PA=QA$ fra forrige deloppgave, gir punktvis karakterisering av midtnormaler (teorem 4.3.7.) oss at punktet $A$ ligger på midtnormalen til linjestykket $\overline{QA}$. 
        Fra sekantlinjeteoremet (teorem 8.1.9.) vet vi også at $O$ ligger på denne midtnormalen. 
        Siden en linje er entydig bestemt av to punkter, må $\overline{OA}$ være midtnormalen til $\overline{PQ}$. 
        Spesielt betyr dette at $\overleftrightarrow{PQ}\perp \overleftrightarrow{OA}$, som var det vi skulle vise. 
    \end{punkt}

\end{oppgave}

\begin{oppgave}[8.3.1]
    La $\triangle ABC$ være en trekant og $M$ være midtpunktet på $AB$. 
    Vi ønsker å vise at dersom $\angle ACB$ er rett, oså er $AM = MC$. 
    
    Vi antar først at $CM > AM$, og jakter en motsigelse. 
    Bruk linjalpostulatet til å finne et punkt $D$ slik at $M \ast D \ast C$ og $AM = MD$, se figuren. 
    Vet å bruke teorem 8.3.1 på $\triangle ABD$, vet vi at $\angle ADB$ er rett. 
    Siden $\angle BDM$ er en ytre vinkel til $\triangle ADC$ og $\angle ADM$ er en ytre vinkel til $\triangle BDC$, gir ytre vinkel-teoremet at
    \begin{itemize}
        \item $\mu(\angle BDM)>\mu(\angle BCM)$
        \item $\mu(\angle ADM)>\mu(\angle ACM)$.
    \end{itemize}
    Men dette må bety at 
    \begin{align*}
        90 = \mu(\angle BDA) 
        &= \mu(\angle ADM) + \mu(\angle BDM) \\
        &> \mu(\angle ACM) + \mu(\angle BCM) \\
        &= \mu(\angle BCA) = 90.
    \end{align*}
    som åpenbart er en motsigelse siden det er en ekte ulikhet i utregningen. 
    Derfor kan vi ikke ha at $CM > AM$, og samme bevis gir at vi ikke kan ha $AM > CM$ -- 
    i dette tilfellet finner vi et punkt $D$ slik at $M \ast C \ast D$ og $MD = AM$, og fullfører beviset som over.

    \begin{figure}[H]
        \centering
        
\definecolor{qqwuqq}{rgb}{0,0.39215686274509803,0}
\definecolor{qqqqff}{rgb}{0,0,1}
\begin{tikzpicture}[line cap=round,line join=round,>=triangle 45,x=1.2cm,y=1.2cm]
\clip(-5,-0.3) rectangle (5,5.5);
\draw [shift={(-1.1427056580220962,3.8333045507922128)},line width=2pt,color=qqwuqq,fill=qqwuqq,fill opacity=0.10000000149011612] (0,0) -- (-126.70035731675141:0.5454545454545456) arc (-126.70035731675141:-73.40071463350279:0.5454545454545456) -- cycle;
\draw [shift={(-1.453214891782001,4.874934519523697)},line width=2pt,color=qqwuqq,fill=qqwuqq,fill opacity=0.10000000149011612] (0,0) -- (-117.58366833369702:0.5454545454545456) arc (-117.58366833369702:-73.40071463350279:0.5454545454545456) -- cycle;
\draw [shift={(-1.1427056580220962,3.8333045507922128)},line width=2pt,color=qqwuqq,fill=qqwuqq,fill opacity=0.10000000149011612] (0,0) -- (286.59928536649716:0.5454545454545456) arc (286.59928536649716:323.2996426832486:0.5454545454545456) -- cycle;
\draw [shift={(-1.453214891782001,4.874934519523697)},line width=2pt,color=qqwuqq,fill=qqwuqq,fill opacity=0.10000000149011612] (0,0) -- (286.5992853664972:0.5454545454545456) arc (286.5992853664972:318.204685061087:0.5454545454545456) -- cycle;
\draw [line width=2pt] (-4,0)-- (0,0);
\draw [line width=2pt] (-2,0.10909090909090915) -- (-2,-0.10909090909090915);
\draw [line width=2pt] (0,0)-- (4,0);
\draw [line width=2pt] (2,0.10909090909090915) -- (2,-0.10909090909090915);
\draw [line width=2pt] (0,0)-- (-1.1427056580220962,3.8333045507922128);
\draw [line width=2pt] (-0.6758974985781071,1.8854875756318672) -- (-0.4668081594439877,1.9478169751603445);
\draw [line width=2pt] (-4,0)-- (-1.453214891782001,4.874934519523697);
\draw [line width=2pt] (-1.453214891782001,4.874934519523697)-- (4,0);
\draw [line width=2pt] (-1.1427056580220962,3.8333045507922128)-- (-1.453214891782001,4.874934519523697);
\draw [line width=2pt] (-1.1427056580220962,3.8333045507922128)-- (-4,0);
\draw [line width=2pt] (-1.1427056580220962,3.8333045507922128)-- (4,0);
\draw [fill=qqqqff] (-4,0) circle (2.5pt);
\draw[color=qqqqff] (-4.267272727272733,0.4263636363636333) node {$A$};
\draw [fill=qqqqff] (0,0) circle (2pt);
\draw[color=qqqqff] (0.2963636363636315,0.35363636363636053) node {$M$};
\draw [fill=qqqqff] (4,0) circle (2.5pt);
\draw[color=qqqqff] (4.296363636363633,0.39) node {$B$};
\draw [fill=qqqqff] (-1.1427056580220962,3.8333045507922128) circle (2.5pt);
\draw[color=qqqqff] (-1.4672727272727326,4.135454545454544) node {$D$};
\draw [fill=qqqqff] (-1.453214891782001,4.874934519523697) circle (2.5pt);
\draw[color=qqqqff] (-1.1581818181818235,5.262727272727272) node {$C$};
\end{tikzpicture}
 
    \end{figure}
\end{oppgave}

\begin{oppgave}[8.3.2]
    Vi antar at vi har en trekant $\triangle ABC$ med vinkler av størrelse $30$, $60$ og $90$, og skal vise at siden motsatt vinkelen på $30$ er halvparten så lang som hypotenusen. 
    
    Vi antar uten tap av generalitet at vi har $\mu(\angle ABC) = 30$, $\mu(\angle CAB) = 60$ og at $\mu(\angle BCA) = 90$, da må vi vise at $AB = 2AC$ – se også figuren. 
    La $M$ være midtpunktet på $AB$. 
    Siden $\angle BCA$ er rett, gir teorem 8.3.3 at $AM = CM$. 
    Dermed gir likebeint trekant-teoremet at $\mu(\angle MCA) = \mu(\angle CAB) = 60$. 
    Da vinkelsummen i en trekant er $180$ i euklidsk geometri, må
    $$\mu(\angle AMC) = 180 - 60 - 60 = 60.$$
    Derfor kan vi anvende det motsatte av likebeint trekant-teoremet (teorem 4.2.2) på $\triangle AMC$, og får at $AM = AC$. 
    Siden $M$ er midtpunktet på $AB$, følger det at
    $$AB = 2AM = 2AC,$$
    som var det vi skulle vise.

    \begin{figure}[H]
        \centering
        
\definecolor{qqwuqq}{rgb}{0,0.39215686274509803,0}
\definecolor{qqqqff}{rgb}{0,0,1}
\begin{tikzpicture}[line cap=round,line join=round,>=triangle 45,x=1.3cm,y=1.3cm]
\clip(-5,-0.3) rectangle (5,4.1);
\draw [shift={(-2.0001991547465097,3.46398662545792)},line width=2pt,color=qqwuqq,fill=qqwuqq,fill opacity=0.10000000149011612] (0,0) -- (-119.99835297616809:0.3837045962897815) arc (-119.99835297616809:-59.99670595233618:0.3837045962897815) -- cycle;
\draw [shift={(-4,0)},line width=2pt,color=qqwuqq,fill=qqwuqq,fill opacity=0.10000000149011612] (0,0) -- (0:0.3837045962897815) arc (0:60.001647023831936:0.3837045962897815) -- cycle;
\draw [shift={(0,0)},line width=2pt,color=qqwuqq,fill=qqwuqq,fill opacity=0.10000000149011612] (0,0) -- (120.00329404766386:0.3837045962897815) arc (120.00329404766386:180:0.3837045962897815) -- cycle;
\draw [shift={(4,0)},line width=2pt,color=qqwuqq,fill=qqwuqq,fill opacity=0.10000000149011612] (0,0) -- (150.00164702383194:0.3837045962897815) arc (150.00164702383194:180:0.3837045962897815) -- cycle;
\draw [line width=2pt] (-4,0)-- (0,0);
\draw [line width=2pt] (-2,0.07674091925795612) -- (-2,-0.07674091925795612);
\draw [line width=2pt] (0,0)-- (4,0);
\draw [line width=2pt] (0,0)-- (-2.0001991547465097,3.46398662545792);
\draw [line width=2pt] (-1.06655695685698,1.693619032270402) -- (-0.9336421978895276,1.7703675931875182);
\draw [line width=2pt] (-2.0001991547465097,3.46398662545792)-- (-4,0);
\draw [line width=2pt] (-2.9336388888167537,1.6936247635623334) -- (-3.0665602659297555,1.7703618618955868);
\draw [line width=2pt] (-2.0001991547465097,3.46398662545792)-- (4,0);
\draw [fill=qqqqff] (-4,0) circle (2.5pt);
\draw[color=qqqqff] (-4.1975108358443824,0.3063594032136353) node {$A$};
\draw [fill=qqqqff] (0,0) circle (2pt);
\draw[color=qqqqff] (0.13835110223014874,0.3191495564232947) node {$M$};
\draw [fill=qqqqff] (4,0) circle (2.5pt);
\draw[color=qqqqff] (4.205619822901833,0.2807790967943166) node {$B$};
\draw [fill=qqqqff] (-2.0001991547465097,3.46398662545792) circle (2.5pt);
\draw[color=qqqqff] (-1.7034309599608024,3.785281076240979) node {$C$};
\draw[color=qqwuqq] (-2,2.8) node {$60^\circ$};
\draw[color=qqwuqq] (-3.35,0.3) node {$60^\circ$};
\draw[color=qqwuqq] (-0.6,0.3) node {$60^\circ$};
\draw[color=qqwuqq] (3.1,0.25519879037499793) node {$30^\circ$};
\end{tikzpicture} 
    \end{figure}
\end{oppgave}

\begin{oppgave}[8.3.3]
    Vi skal vise følgende: hvis $\triangle ABC$ er en rettvinklet trekant slik at den ene kateten er halvparten så lang som hypotenusen, så må vinklene i trekanten være av størrelse $30$, $60$, $90$. 
    
    Vi antar uten tap av generalitet at $\mu(\angle BCA) = 90$ og $AC = \frac{1}{2} AB$, se også figuren. 
    La $M$ være midtpunktet på $AB$. Da er $AC = \frac{1}{2} AB = AM$, siden vi antar at $AC = \frac{1}{2} AB$ og $M$ er midtpunktet på $AB$. 
    Av teorem 8.3.3 får vi også at $AM = MC$ siden $\angle BCA$ er rett. 
    Altså er $\triangle AMC$ en likesidet trekant, og det følger at alle vinklene i $\triangle AMC$ er like store. 
    For eksempel kan vi argumentere ved likebeint trekant-teoremet:
    \begin{itemize}
        \item Siden $AM = MC$, må $\angle CAM \cong \angle MCA$ av likebeint trekant-teoremet.
        \item Siden $AC = AM$, må $\angle MCA \cong \angle AMC$ av likebeint trekant-teoremet.
    \end{itemize}
    Siden alle vinklene i $\triangle AMC$ er like store og vinkelsummen i en trekant er $180$, må alle vinklene i $\triangle AMC$ være av størrelse $60$. 
    Spesielt må $\mu(\angle CAM) = 60$, og da $\angle CAM = \angle CAB$ følger det at $\mu(\angle CAB) = 60$. 
    Siden vinkelsummen i $\triangle ABC$ også er $180$, får vi at
    $$ \mu(\angle ABC) = 180 - \mu(\angle BCA) - \mu(\angle CAB) = 180 - 90 - 60 = 30,$$
    og dermed er resultatet vist.

    \begin{figure}[H]
        \centering
        
\definecolor{qqwuqq}{rgb}{0,0.39215686274509803,0}
\definecolor{qqqqff}{rgb}{0,0,1}
\begin{tikzpicture}[line cap=round,line join=round,>=triangle 45,x=1.3cm,y=1.3cm]
\clip(-5,-0.3) rectangle (5,5);
\draw [shift={(-2.0001991547465097,3.46398662545792)},line width=2pt,color=qqwuqq,fill=qqwuqq,fill opacity=0.10000000149011612] (0,0) -- (-119.99835297616809:0.3837045962897815) arc (-119.99835297616809:-59.99670595233618:0.3837045962897815) -- cycle;
\draw [shift={(-4,0)},line width=2pt,color=qqwuqq,fill=qqwuqq,fill opacity=0.10000000149011612] (0,0) -- (0:0.3837045962897815) arc (0:60.001647023831936:0.3837045962897815) -- cycle;
\draw [shift={(0,0)},line width=2pt,color=qqwuqq,fill=qqwuqq,fill opacity=0.10000000149011612] (0,0) -- (120.00329404766386:0.3837045962897815) arc (120.00329404766386:180:0.3837045962897815) -- cycle;
\draw [line width=2pt] (-4,0)-- (0,0);
\draw [line width=2pt] (-2,0.07674091925795612) -- (-2,-0.07674091925795612);
\draw [line width=2pt] (0,0)-- (4,0);
\draw [line width=2pt] (0,0)-- (-2.0001991547465097,3.46398662545792);
\draw [line width=2pt] (-1.06655695685698,1.693619032270402) -- (-0.9336421978895276,1.7703675931875182);
\draw [line width=2pt] (-2.0001991547465097,3.46398662545792)-- (-4,0);
\draw [line width=2pt] (-2.9336388888167537,1.6936247635623334) -- (-3.0665602659297555,1.7703618618955868);
\draw [line width=2pt] (-2.0001991547465097,3.46398662545792)-- (4,0);
\draw [shift={(-2.0001991547465097,3.46398662545792)},line width=2pt,color=qqwuqq] (-119.99835297616809:0.3837045962897815) arc (-119.99835297616809:-59.99670595233618:0.3837045962897815);
\draw[line width=2pt,color=qqwuqq] (-2.000184677913514,3.128245104016474) -- (-2.000180541675516,3.0323189550332037);
\draw [shift={(-4,0)},line width=2pt,color=qqwuqq] (0:0.3837045962897815) arc (0:60.001647023831936:0.3837045962897815);
\draw[line width=2pt,color=qqwuqq] (-3.709241725891705,0.16787493996115405) -- (-3.6261679332893357,0.21583920852148375);
\draw [shift={(0,0)},line width=2pt,color=qqwuqq] (120.00329404766386:0.3837045962897815) arc (120.00329404766386:180:0.3837045962897815);
\draw[line width=2pt,color=qqwuqq] (-0.29076551243469473,0.167862402603993) -- (-0.37384137313032134,0.21582308906227712);
\draw [fill=qqqqff] (-4,0) circle (2.5pt);
\draw[color=qqqqff] (-4.1975108358443824,0.3063594032136353) node {$A$};
\draw [fill=qqqqff] (0,0) circle (2pt);
\draw[color=qqqqff] (0.13835110223014874,0.3191495564232947) node {$M$};
\draw [fill=qqqqff] (4,0) circle (2.5pt);
\draw[color=qqqqff] (4.205619822901833,0.2807790967943166) node {$B$};
\draw [fill=qqqqff] (-2.0001991547465097,3.46398662545792) circle (2.5pt);
\draw[color=qqqqff] (-1.7034309599608024,3.785281076240979) node {$C$};
\end{tikzpicture}
 
    \end{figure}
\end{oppgave}