
\begin{oppgave}[3.4.1]
    La $A$, $B$, $C$ være tre punkter som ikke ligger på en linje. Vi skal finne et punkt $D$ som 
    ligger i det indre av $\angle BAC$ slik at $\mu(\angle BAD)=\mu(\angle DAC)$. La $H_C$ være 
    halvplanet bestemt av linje $\overleftrightarrow{AB}$ og punktet $C$. 
    
    Av tredje del av gradskivepostulatet (aksiom 3.4.1) finnes det en unik stråle $\overrightarrow{AD}$ 
    slik at $D$ ligger i $H_C$ og $\mu(\angle BAD) = \frac{\mu(\angle BAC)}{2}$. Siden 
    $\mu(\angle BAD) < \mu(\angle BAC)$, gir teorem 3.4.5 at $\overrightarrow{AD}$ ligger mellom strålene
    $\overrightarrow{AB}$ og $\overrightarrow{AC}$, som per definisjon betyr at $D$ ligger i det indre av
    $\angle BAC$. Av del 4 av gradskivepostulatet (vinkeladdisjonspostulatet) er 
    $$\mu(\angle BAC) = \mu(\angle BAD)+\mu(\angle DAC),$$
    og siden $\mu(\angle BAD) = \frac{\mu(\angle BAC)}{2}$ per konstruksjon må 
    $\mu(\angle BAD) = \mu(\angle DAC) = \frac{\mu(\angle BAC)}{2}$. 
    
    Siden strålen $\overrightarrow{AD}$ er unik av del tre av gradskivepostulatet, har vi også vist 
    unikhetsdelen av utsagnet. 
\end{oppgave}

\begin{oppgave}[3.4.2]
    \begin{punkt}
        Vi må vise at $f$ er injektiv (\emph{en-til-en}) og surjektiv (\emph{på}). 
        
        \textbf{Surjektiv:} La $c\in (0, 180)$ være gitt. Av del tre av gradskivepostulatet finnes en 
        stråle $\overrightarrow{AE}$ slik at $E\in H$ og $\mu(\angle BAE) = c$. Med andre ord er 
        $f(\angle BAE)=c$, som viser at $f$ er surjektiv. 

        \textbf{Injektiv:} Anta at $\mu(\angle BAD) = \mu(\angle BAE)$ for 
        $\angle BAD, \angle BAE \in \mathcal{A}$. For et hvert tall $r\in (0, 180)$ sier del tre av 
        gradskivepostulatet at det finnes en \emph{unik} stråle $\overrightarrow{AE}$ alik at 
        $\mu(\angle BAE)=r$, og unikhetsdelen gir oss at $\overrightarrow{AE}=\overrightarrow{AD}$ 
        ettersom $\mu(\angle BAD) = \mu(\angle BAE)$. Da vi har definert en vinkel som unionen av to stråler
        får vi at 
        $$ \angle BAE 
        = \overrightarrow{AB}\cup \overrightarrow{AE}
        = \overrightarrow{AB}\cup \overrightarrow{AD} 
        = \angle BAD, $$
        som viser at $f$ er injektiv. 
    \end{punkt}

    \begin{punkt}
        Av teorem 3.4.5 vet vi at $\overrightarrow{AF}$ ligger mellom strålene $\overrightarrow{AB}$ og 
        $\overrightarrow{AE}$ hvis og bare hvis $\mu(\angle BAF)<\mu(\angle BAE)$. Men vi har definert 
        $f(\angle BAF)=\mu(\angle BAF)$ og $f(\angle BAE)=\mu(\angle BAE)$. Derfor har vi 
        $\mu(\angle BAF)<\mu(\angle BAE)$ hvis og bare hvis $f(\angle BAF)<f(\angle BAE)$, 
        som beviser utsagnet. Vi burde også egentlig argumentere for at $f(\angle BAF)>0$, men dette 
        følger direkte fra gradskivepostulatet. 
    \end{punkt}
\end{oppgave}

\begin{oppgave}[3.5.1]
    Det at $m\perp l$ betyr per definisjon at det finnes et punkt $A$ som ligger på både $m$ og $l$, og 
    to punkter $B\in l$, $C\in m$ slik at $\mu(\angle BAC) = 90$. Ved hjelp av linjalpostulatet kan vi 
    finne et punkt $D$ på $l$ og et punkt $E$ PÅ $m$ slik at $D\ast A\ast B$ og $E\ast A\ast C$. Da er 
    $\overrightarrow{AB}$ og $\overrightarrow{AD}$ motsatte stråler, så vi kan anvende teorem 3.5.5 som 
    gir oss at 
    $$ \mu(\angle BAC)+\mu(\angle CAD) = 180 .$$
    Siden vi vet at $\mu(\angle BAC)=90$, må vi også ha $\mu(\angle CAD)=90$. Helt tilsvarende kan man 
    vise at $\mu(\angle DAE)=\mu(\angle EAB) = 90$, slik at de fire strålene $\overrightarrow{AB}$, 
    $\overrightarrow{AC}$, $\overrightarrow{AD}$ og $\overrightarrow{AE}$ står vinkelrett på hverandre. 
\end{oppgave}

\begin{oppgave}[3.5.2]
    Vi viser først eksistens, og så unikhet. 

    \textbf{Eksistens:} La $l$ være en linje og $A$ et punkt på $l$. Vi velger et annet punkt $B$ på 
    $l$, slik at $\overleftrightarrow{AB}=l$. Fra planseparasjonsaksiomet deler linjen planet inn i to 
    halvplan $H_1$ og $H_2$ -- vi velger å bruke $H_1$ her, men kunne like gjerne 
    brukt $H_2$. Fra gradskivepostulatet finnes det for etthvert reellt tall $r$ mellom $0$ og $180$
    en unik stråle $\overrightarrow{AE}$ slik at $E\in H_1$ og $\mu(\angle BAE)=r$. Ved å velge 
    $r =  90$ får vi at linjen $\overleftrightarrow{AE}$ er perpendikulær på linjen 
    $\overleftrightarrow{AB}$. Altså eksisterer det en linje $m$ slik at $A\in m$ og $m\perp l$. 

    \textbf{Unikhet:} Anta at vi har to linjer $m$ og $m'$ slik at de begge er perpendikulære til $l$ og
    at punktet $A$ ligger på både $m$ og $m'$. Fra definisjonen av å være perpendikulær, finnes det to 
    punkter $B$ og $B'$ på $l$, et punkt $C\in m$ og et punkt $C'\in m'$, slik at 
    \begin{enumerate}[label=\arabic*)]
        \item $\mu(\angle BAC)=90$
        \item $\mu(\angle B'AC')=90$.
    \end{enumerate}
    Vi har da to muligheter. Linjen $l$ deler ved planseparasjonsaksiomet planet inn i to halvplan 
    $H_1$ og $H_2$, så enten ligger punktene $C$ og $C'$ i samme halvplan, eller så ligger de i 
    ulike halvplan. Dersom de ligger i samme halvplan, for eksempel $H_1$, vet vi fra 
    gradskivepostulatet at strålen $\overrightarrow{AC}$ som danner vinkelen $\mu(\angle BAC)$ er unik. 
    Dermed må vi ha $\overrightarrow{AC}=\overrightarrow{AC'}$, som vil si at linjene $m$ og $m'$ deler
    to punkter -- ergo er de like. Dersom $C$ og $C'$ ligger i ulike halvplan er strålene 
    $\overrightarrow{AC}$ og $\overrightarrow{AC'}$ motsatte stråler. Dermed er 
    $m=\overleftrightarrow{AC}=\overleftrightarrow{AC'}=m'$, som viser at linjene er de samme. 
    
    Altså finnes det for enhver linje $l$ og et punkt $A$ på $l$, en unik linje $m$ slik at $A\in m$ og 
    $l\perp m$.  
\end{oppgave}

\begin{oppgave}[3.5.3]
    La $A$ og $B$ være ulike punkt. Vi må vise at det finnes en unik linje $m$ slik at midtpunktet $M$ av 
    $\overline{AB}$ ligger på $m$ og $\overleftrightarrow{AB}\perp m$. Fra teorem 3.2.22 vet vi at 
    midtpunktet $M$ finnes og er unikt. Fra forrige oppgave vet vi at det finnes en unik linje $m$ som 
    går gjennom $M$ og er perpendikulær til $\overleftrightarrow{AB}$. Denne linja tilfredsstiller alle
    kravene våre, og må være unik fordi både midtpunktet og linjen er det. 
\end{oppgave}

\begin{oppgave}[3.5.4]
    La $\angle ABC$, $\angle DEF$, $\angle GHI$ og $\angle JKL$ være fire vinkler slik at 
    \begin{enumerate}[label = \arabic*)]
        \item $\angle ABC$ og $\angle DEF$ er supplementærvinkler
        \item $\angle GHI$ og $\angle JKL$ er supplementærvinkler
        \item $\angle DEF \cong \angle JKL$. 
    \end{enumerate}
    Vi må vise at $\angle ABC \cong \angle GHI$. 

    Fra punkt 1) og 2) vet vi at $\mu(\angle ABC)+\mu(\angle DEF) = 180$ og 
    $\mu(\angle GHI)+\mu(\angle JKL) = 180$. Fra punkt 3) vet vi at $\mu(\angle DEF)=\mu(\angle JKL)$. 
    Vi har dermed 
    \begin{align*}
        \mu(\angle ABC)+\mu(\angle DEF) 
        &= 180 \\
        &= \mu(\angle GHI)+\mu(\angle JKL) \\
        &= \mu(\angle GHI)+ \mu(\angle DEF)
    \end{align*} 
    Ved å trekke fra $\mu(\angle DEF)$ i ligningen står vi igjen med $\angle ABC \cong \angle GHI$, 
    som var det vi ville vise. 
\end{oppgave}

\begin{oppgave}[3.5.5]
    Vi ønsker å vise at dersom to vinkler $\angle BAC$ og $\angle DAE$ er slik at 
    \begin{enumerate}[label=\arabic*)]
        \item $\overrightarrow{AB}$ og $\overrightarrow{AE}$ er motsatte stråler,
        \item $\overrightarrow{AC}$ og $\overrightarrow{AD}$ er motsatte stråler,
    \end{enumerate}
    så er $\mu(\angle BAC)=\mu(\angle DAE)$. 

    Siden $\overrightarrow{AB}$ og $\overrightarrow{AE}$ er motsatte stråler, gir teorem 3.5.5 oss at 
    $\mu(\angle BAC)+\mu(\angle CAE)=180$, altså at de er supplementære. Helt tilsvarende får vi at 
    $\mu(\angle CAE)+\mu(\angle DAE)=180$. Ved å trekke den andre likningen fra den første får vi 
    $$\mu(\angle BAC)-\mu(\angle DAE) = 0,$$
    som gir oss det vi ønsket: $\mu(\angle BAC)=\mu(\angle DAE)$. 
\end{oppgave}

\begin{oppgave}[3.5.6]
    Fra linjalpostulatet har vi et punkt $F$ på $\overleftrightarrow{DB}$ slik at $F\ast B\ast D$. Siden
    $\overrightarrow{BD}$ og $\overrightarrow{BF}$ er motsatte stråler, kan vi bruke teorem 3.5.13 til å
    konkludere med at $\mu(\angle DBC)=\mu(\angle ABF)$. Siden vi antar at 
    $\mu(\angle DBC)=\mu(\angle ABE)$, har vi $$\mu(\angle ABE)=\mu(\angle ABF).$$
    Fra konstruksjonen vår vet vi at $E$ og $F$ ligger op samme side av $\overleftrightarrow{AB}$. 
    Resultatet følger nå fra unikhetsdelen av punkt 3) i gradskivepostulatet, altså: siden 
    $\mu(\angle ABE)=\mu(\angle ABF)$ og $E$ og $F$ ligger på samme side av $\overleftrightarrow{AB}$, 
    må $\overrightarrow{BE}=\overrightarrow{BF}$. Siden $\overrightarrow{BF}$ og $\overrightarrow{BD}$ 
    er motsatte stråler får vi at også $\overrightarrow{BD}$ og $\overrightarrow{BE}$ er motsatte 
    stråler, som var det vi skulle vise. 
\end{oppgave}