
\begin{oppgave}[3.6.1]
    La $A$, $B$, $C$, være tre punkter som ikke ligger på linje. Vi skal finne en stråle $\overrightarrow{AB}$ slik at $M$ ligger i det indre av vinkelen $\angle BAC$ og $\mu(\angle BAM)=\mu(\angle MAC)$.
    Vi bruker først teorem 3.2.23 til å finne et punkt $C'$ på $\overrightarrow{AC}$ slik at $AB=AC'$. 
    Vi bruker så teorem 3.2.22 til å finne midtpunktet $M$ på linjestykket $\overline{BC'}$.
    Det at punktet $M$ ligger i det indre av vinkelen vår får vi fra teorem 3.3.10. 
    Rent intuitivt burde strålen $\overrightarrow{AM}$ være vinkelhalveringsstrålen vi søker, og vi skal nå vise at dette faktisk stemmer. 

    Trekanten $\triangle ABC$ er likebeint per konstruksjon, og av teorem 3.6.5 følger det at $\angle BC'A \cong \angle C'BA$. 
    Nå kan vi bruke side-vinkel-side postulatet på trekantene $\triangle ABM$ og $\triangle AMC'$ til å konkludere med at $\triangle ABM \cong \triangle AMC'$. 
    Dette betyr da spesielt at vi har $\mu(\angle BAM) = \mu(\angle MAC')$.
    Siden $C'$ ligger på strålen $\overrightarrow{AC}$ har vi også $\mu(\angle MAC') = \mu(\angle MAC)$.
    Dermed kan vi konkludere med at $\mu(\angle BAM) = \mu(\angle MAC)$, som viser at $\overrightarrow{AM}$ er en vinkelhalveringsstråle til vinkelen. 

    Unikhet følger ettersom både punktet $C'$ og midtpunktet $M$ er unike. 

    \begin{figure}[H]
        \centering
        

\definecolor{qqqqff}{rgb}{0,0,1}
\begin{tikzpicture}[line cap=round,line join=round,>=triangle 45,x=1cm,y=1cm]
\clip(-5,-1.6) rectangle (6,5);
\draw [shift={(2.580611412565809,2.558493359853798)},line width=2pt,color=qqqqff,fill=qqqqff,fill opacity=0.10000000149011612] (0,0) -- (-145.91991251835023:0.5433730839157709) arc (-145.91991251835023:-72.9599562591751:0.5433730839157709) -- cycle;
\draw [shift={(3.8,-1.42)},line width=2pt,color=qqqqff,fill=qqqqff,fill opacity=0.10000000149011612] (0,0) -- (107.04004374082488:0.5433730839157709) arc (107.04004374082488:180:0.5433730839157709) -- cycle;
\draw[line width=2pt,color=qqqqff,fill=qqqqff,fill opacity=0.10000000149011612] (2.8229502233327057,0.45665403670144733) -- (2.9355428665581575,0.08929855375124851) -- (3.3028983495083564,0.20189119697670033) -- (3.1903057062829046,0.5692466799268991) -- cycle; 
\draw [line width=2pt] (-3.3,-1.42)-- (3.1903057062829046,0.5692466799268991);
\draw [line width=2pt] (2.580611412565809,2.558493359853798)-- (3.1903057062829046,0.5692466799268991);
\draw [line width=2pt] (3.1903057062829046,0.5692466799268991)-- (3.8,-1.42);
\draw [line width=2pt] (-3.3,-1.42)-- (2.580611412565809,2.558493359853798);
\draw [line width=2pt] (-3.3,-1.42)-- (3.8,-1.42);
\draw [line width=2pt] (2.580611412565809,2.558493359853798)-- (10.176145409068669,7.697207593727221);
\draw [line width=2pt] (3.8,-1.42)-- (13.761892580342884,-1.42);
\draw [fill=qqqqff] (-3.3,-1.42) circle (2.5pt);
\draw[color=qqqqff] (-3.3123628715083647,-1.0527231948449258) node {$A$};
\draw [fill=qqqqff] (3.8,-1.42) circle (2.5pt);
\draw[color=qqqqff] (4.068836452962964,-1.1613978116280799) node {$B$};
\draw [fill=qqqqff] (5.046908004011351,4.227051937203316) circle (2.5pt);
\draw[color=qqqqff] (5.13747018466398,3.819522124266487) node {$C$};
\draw [fill=qqqqff] (2.580611412565809,2.558493359853798) circle (2pt);
\draw[color=qqqqff] (2.692291307043011,2.9776754454791504) node {$C'$};
\draw [fill=qqqqff] (3.1903057062829046,0.5692466799268991) circle (2pt);
\draw[color=qqqqff] (3.5892384967861414,0.6679582375550155) node {$M$};
\draw [fill=qqqqff] (10.176145409068669,7.697207593727221) circle (2.5pt);
\draw[color=qqqqff] (10.317626917994328,8.094057051070552) node {$I$};
\draw [fill=qqqqff] (13.761892580342884,-1.42) circle (2.5pt);
\draw[color=qqqqff] (10.879112438040623,-1.867782820718582) node {$J$};
\draw[color=qqqqff] (2.3,1.8) node {$\alpha$};
\draw[color=qqqqff] (3.2,-0.8) node {$\alpha$};
\end{tikzpicture}
    \end{figure}
\end{oppgave}

\begin{oppgave}[3.6.2]
    Vi skal vise ved eksempel at side-vinkel-side postulatet ikke holder for $\R^2$ med kvadratmetrikken.
    Med andre ord må vi finne to trekanter $\triangle ABC$ og $\triangle DEF$ slik at $\overline{AB}\cong \overline{DE}$, $\angle BAC \cong \angle DEF$ og $\overline{AC}\cong \overline{EF}$, men ikke $\triangle ABC\cong\triangle DEF$.
    Et slikt eksempel er vist ved følgende to figurer.
    
    \begin{figure}[H]
        \centering
        \input{oving_4/362_1.tex}
    \end{figure}

    \begin{figure}[H]
        \centering
        
\definecolor{qqqqff}{rgb}{0,0,1}
\begin{tikzpicture}[line cap=round,line join=round,>=triangle 45,x=1cm,y=1cm]
\begin{axis}[
x=3cm,y=3cm,
axis lines=middle,
xmin=-1.4,
xmax=1.4,
ymin=-0.2,
ymax=1.4,
xtick={-1,-0.5,0,0.5, 1},
ytick={-1,-0.5,0,0.5, 1},]
\clip(-1.4,-1.4) rectangle (1.4,1.4);
\draw [line width=2pt] (0,0)-- (0,1);
\draw [line width=2pt] (0,1)-- (1,0);
\draw [line width=2pt] (1,0)-- (0,0);
\draw [fill=qqqqff] (0,1) circle (2.5pt);
\draw [fill=qqqqff] (0,1) circle (2.5pt);
\draw[color=qqqqff] (0.1,1.1) node {$D$};
\draw [fill=qqqqff] (1,0) circle (2.5pt);
\draw[color=qqqqff] (1.09,0.09) node {$F$};
\draw [fill=qqqqff] (0,0) circle (2.5pt);
\draw[color=qqqqff] (-0.1,0.1) node {$E$};
\end{axis}
\end{tikzpicture}
    \end{figure}
    
    Vi har $AB=DE=1$, $\mu(\angle BAC) = \mu(\angle DEF) = 90$ og $AC=EF=1$ (husk at vi måler avstand i kvadratmetrikken her). 
    Men vi har $BC=2$ og $DF=1$, så vi kan ikke ha $\triangle ABC\cong \triangle DEF$. 
\end{oppgave}

\begin{oppgave}[3.7.2]
\begin{punkt}
    Eksistenspostulatet (aksiom 3.1.1) sier at det finnes minst to punkter. 
    Insidenspostulatet (aksiom 3.1.3) sier at to punkter alltid bestemmer en unik linje. 
    Sammen impliserer disse aksiomene at vi har minst en linje. 
\end{punkt}

\begin{punkt}
    Av eksistenspostulatet kan vi finne to punkter $A$ og $B$. 
    Av planseparasjonspostulatet (aksiom 3.3.2) deler linja $\overleftrightarrow{AB}$ punktene utenfor linja inn i to disjunkte, ikke-tomme halvplan $H_1$ og $H_2$.
    Siden $H_1$ ikke er tomt finnes det et punkt $C$ i $H_1$. 
    Siden $H_1$ består av av punkter som ikke ligger på $\overleftrightarrow{AB}$ kan ikke $A$, $B$ og $C$ være kolineære.
\end{punkt}

\begin{punkt}
    Fra a) vet vi at det finnes en linje $l$. 
    Av linjalpostulatet finnes det en koordinatfunksjon $f_l\longrightarrow \R$. 
    Siden $f$ er en en-til-en-korrespondanse (bijeksjon) mellom $l$ og den uendelige mengden $\R$, må også $l$ ha uendelig mange elementer.
\end{punkt}

\begin{punkt}
    Vi vet fra a) at det finnes en linje $l$, og planseparasjonspostulatet gir oss, som i b), at det finnes et punkt $P$ som ikke ligger på denne linjen.
    Hvis $A$ er et punkt på $l$, så bestemmer $A$ og $P$ en linje $\overleftrightarrow{AP}$. 
    
    \textbf{Påstand:} Dersom $A$ og $B$ er to ulike punkter på $l$, vil $\overleftrightarrow{AP}\neq \overleftrightarrow{BP}$. 
    La oss vise dette. 
    Dersom vi har $\overleftrightarrow{AP} = \overleftrightarrow{BP}$ ligger både punktet $A$ og $B$ på linja $\overleftrightarrow{AP}$. 
    Men, vi vet også at $A$ og $B$ ligger på $l$. 
    Fra insidenspostulatet vet vi at to punkter bestemmer en unik linje, alstå må vi ha $\overleftrightarrow{AB}=l$. 
    Dette betyr at $P$ ligger på $l$, noe som er en selvmotsigelse til antagelsen vår om at $P$ var et punkt som ikke lå på $l$. 

    Dermed bestemmer alle punkter $A$ på en linje en ny linje $\overleftrightarrow{AP}$.
    Fra c) vet vi at en linje inneholder uendelig mange punkt, så vi få da uendelig mange linjer $\overleftrightarrow{AP}$ ved å variere $A$ langs linjen $l$.
\end{punkt}

\begin{punkt}
    Aksiom 1 for insidensgeometri er det samme som insidenspostulatet for nøytral geometri. 
    Aksiom 2 følger fra c), der vi viste at det var uendelig mange punkter på en linje.
    Aksiom 3 følger fra b), da vi viste at det fantes tre punkter $A$, $B$, $C$ som ikke alle var kolineære.
    
    I enhver modell for nøytral geometri må aksiomene for nøytral geometri være oppfylt, og siden aksiomene for insidensgeometri følger fra disse må også aksiomene for insidensgeometri være sanne i en slik modell. 
    Med andre ord, enhver modell for nøytral geometri er en modell for insidensgeometri.
\end{punkt}

\begin{punkt}
    Et teorem i insidensgeometri er et utsagn som kan deduseres logisk fra aksiomene for insidensgeometri. 
    Siden disse aksiomene kan deduseres fra aksiomene i nøytral geometri, kan vi også dedusere alle teoremer i insidensgeometri fra aksiomene i nøytral geometri. 
    Derfor er alle teorem i insidensgeometri også teoremer i nøytral geometri. 
\end{punkt}
\end{oppgave}

\begin{oppgave}[4.1.1]
    Dette resultatet følger fra ytre vinkel-teoremet og lineært par-teoremet.
    Oppgaven sier at vi har en trekant $\triangle ABC$ der en av vinklene er rett eller stump (obtuse).
    Vi antar, uten tap av generalitet, at det er vinkelen $\angle BAC$ som er rett eller stump, altså at $\mu(\angle BAC)\geq 90$. 
    Se også figuren under. 

    Anta så at $D$ er et punkt på $\overleftrightarrow{AB}$ slik at $D\ast A\ast B$ og $\angle DAC$ er en ytre vinkel til $\angle BAC$.
    Av lineært par-teoremet vet vi at $\mu(\angle BAC)+\mu(\angle DAC) = 180$, altså $$ \mu(\angle DAC) = 180-\mu(\angle BAC) \leq 90,$$ der vi har brukt at vi vet $\mu(\angle BAC)\geq 90$. 
    Av ytre vinkel-teoremet har vi $\mu(\angle DAC)>\mu(\angle ACB)$ og $\mu(\angle DAC)>\mu(\angle ABC)$, som da gir oss 
    $$90\geq \mu(\angle DAC)>\mu(\angle ACB)$$ og 
    $$90\geq \mu(\angle DAC)>\mu(\angle ABB),$$
    som var det vi skulle vise. 

    \begin{figure}[H]
        \centering
        
\definecolor{qqqqff}{rgb}{0,0,1}
\begin{tikzpicture}[line cap=round,line join=round,>=triangle 45,x=3cm,y=3cm]
\clip(-1.4,-0.2) rectangle (1.8,1);
\draw [line width=2pt,domain=-1.815475572808079:1.596707451476368] plot(\x,{(--0.09772670735916965-0*\x)/-1.3693629242194159});
\draw [line width=2pt] (-0.08693548813766808,0.8116106362841631)-- (0.22734452725695198,-0.07136654982453043);
\draw [line width=2pt] (-0.08693548813766808,0.8116106362841631)-- (1.596707451476368,-0.07136654982453043);
\draw [fill=qqqqff] (0.22734452725695198,-0.07136654982453043) circle (2.5pt);
\draw[color=qqqqff] (0.3,0.05) node {$A$};
\draw [fill=qqqqff] (1.596707451476368,-0.07136654982453043) circle (2.5pt);
\draw[color=qqqqff] (1.65,0.05) node {$B$};
\draw [fill=qqqqff] (-0.08693548813766808,0.8116106362841631) circle (2.5pt);
\draw[color=qqqqff] (-0.02,0.9) node {$C$};
\draw [fill=qqqqff] (-0.48333736871769406,-0.07136654982453043) circle (2.5pt);
\draw[color=qqqqff] (-0.4,0.05) node {$D$};
\end{tikzpicture}

    \end{figure}
\end{oppgave}

\begin{oppgave}[4.2.1]
    Vi skal vise følgende: dersom $\triangle ABC$ er en trekant der $\angle ABC\cong \angle ACB$, så er $\overline{AB}\cong \overline{AC}$. 
    Vi antar derfor at $\triangle ABC$ tilfredstiller $\angle ABC\cong \angle ACB$.
    Som boka antyder på side 74, er det mulig å vise at $\overline{AB}\cong \overline{AC}$ ved en lur anvendelse av vinkel-side-vinkel-postulatet (VSV).
    Vi får nemlig fra VSV at $\triangle BCA\cong \triangle CBA$, siden vi har $\angle ACB\cong \angle ABC$, $\overline{BC}\cong \overline{CB}$ og $\angle ABC\cong \angle ACB$.
    Siden trekantene er kongruente må vi spesielt ha $\overline{AB}\cong \overline{AC}$, som var det vi skulle vise. 
    Det kan være enklere å visualisere argumentet ved å se på figuren under. 

    \begin{figure}[H]
        \centering
        
\definecolor{qqqqff}{rgb}{0,0,1}
\begin{tikzpicture}[line cap=round,line join=round,>=triangle 45,x=1cm,y=1cm]
\clip(-4,-4) rectangle (12,2);
\draw [line width=2pt] (-0.46,-3.88)-- (3.82,-3.88);
\draw [line width=2pt] (-0.46,-3.88)-- (1.68,1.0458427513442228);
\draw [line width=2pt] (1.68,1.0458427513442228)-- (3.82,-3.88);
\draw [line width=2pt] (5.22,-3.9)-- (9.4,-3.9);
\draw [line width=2pt] (7.31,1.04)-- (5.22,-3.9);
\draw [line width=2pt] (7.31,1.04)-- (9.4,-3.9);
\draw [fill=qqqqff] (-0.46,-3.88) circle (2.5pt);
\draw[color=qqqqff] (-0.66,-3.47) node {$B$};
\draw [fill=qqqqff] (3.82,-3.88) circle (2.5pt);
\draw[color=qqqqff] (4.08,-3.57) node {$C$};
\draw [fill=qqqqff] (1.68,1.0458427513442228) circle (2.5pt);
\draw[color=qqqqff] (1.92,1.33) node {$A$};
\draw [fill=qqqqff] (5.22,-3.9) circle (2.5pt);
\draw[color=qqqqff] (5,-3.55) node {$C$};
\draw [fill=qqqqff] (9.4,-3.9) circle (2.5pt);
\draw[color=qqqqff] (9.62,-3.49) node {$B$};
\draw [fill=qqqqff] (7.31,1.04) circle (2.5pt);
\draw[color=qqqqff] (7.56,1.37) node {$A$};
\end{tikzpicture}
    \end{figure}
\end{oppgave}

\begin{oppgave}[4.2.2]
    Vi skal vise: Dersom $\triangle ABC$ og $\triangle DEF$ er to trekanter der $\angle ABC\cong \angle DEF$, $\overline{AC}\cong \overline{DF}$ og $\angle BCA\cong \angle EFD$, så må $\triangle ABC\cong \triangle DEF$. 

    Merk først at dersom $\overline{CB}\cong \overline{FE}$, så følger resultatet fra side-vinkel-side-postulatet (SVS). 
    Så, dersom vi klarer å vise at dette er tilfelle er vi ferdige. 
    Anta uten tap av generalitet at $CB\geq FE$. 
    Fra teorem 3.2.23 kan vi finne et punkt $P$ på $\overrightarrow{CB}$ slik at $CP=FE$. 
    Ettersom vi antar at $CP=FE\leq CB$ følger det fra korollar 3.2.18 at $P$ må ligge på linjestykket $\overline{CB}$. 
    Dersom $P=B$ er vi i mål, siden vi da kan bruke SVS slik vi allerede har nevnt. 
    Vi antar dermed at $P\neq B$ og viser at dette fører til en selvmotsigelse. 

    Per konstruksjon og antagelser har vi $\overline{AC}\cong \overline{CP}$, $\angle ACP\cong \angle DFE$ og $\overline{CP}\cong \overline{FE}$. 
    Dermed gir SVS at vi har $\triangle ACP\cong \triangle DFE$, og dermed spesielt at
    $$\mu(\angle APC)=\mu(\angle DEF)=\mu(\angle ABC),$$
    der med siste likheten er en av de oprinnelige antagelsene. 
    Men vi har også at $\angle APC$ er en ytre vinkel til $\triangle ABP$, og ytre vinkel-teoremet gir da at 
    $$\mu(\angle APC)>\mu(\angle ABC)=\mu(\angle ABP).$$
    Disse to likningene gir oss da at $\mu(\angle APC)$ er både lik og ekte større enn $\mu(\angle ABC)$, noe som åpenbart ikke kan stemme samtidig. 
    Vi har alstå funnet en selvmotsigelse, som betyr at antagelsen vår var feil. 
    Dermed har vi $B=P$, som fullfører beviset. 

    \begin{figure}[H]
        \centering
        
\definecolor{qqqqff}{rgb}{0,0,1}
\begin{tikzpicture}[line cap=round,line join=round,>=triangle 45,x=1cm,y=1cm]
\clip(-3,-4) rectangle (11,0);
\draw [line width=2pt] (-1.3404949093937215,-3.8444129438882584)-- (3.82,-3.88);
\draw [line width=2pt] (-1.3404949093937215,-3.8444129438882584)-- (1.2589039737334464,-1.0850492771513376);
\draw [line width=2pt] (1.2589039737334464,-1.0850492771513376)-- (3.82,-3.88);
\draw [line width=2pt] (5.22,-3.9)-- (10.405791538575313,-3.864254643969287);
\draw [line width=2pt] (7.793351247985521,-1.0466872970753194)-- (5.22,-3.9);
\draw [line width=2pt] (7.793351247985521,-1.0466872970753194)-- (10.405791538575313,-3.864254643969287);
\draw [line width=1.5pt,dotted] (-1.3404949093937215,-3.8444129438882584)-- (2.947598748764495,-2.9279394436025035);
\draw [fill=qqqqff] (-1.3404949093937215,-3.8444129438882584) circle (2.5pt);
\draw[color=qqqqff] (-1.5389119102040092,-3.437658092227169) node {$A$};
\draw [fill=qqqqff] (3.82,-3.88) circle (2.5pt);
\draw[color=qqqqff] (4.076289212727134,-3.5765499927943702) node {$B$};
\draw [fill=qqqqff] (1.2589039737334464,-1.0850492771513376) circle (2.5pt);
\draw[color=qqqqff] (1.4968682021933932,-0.7987119814503453) node {$C$};
\draw [fill=qqqqff] (5.22,-3.9) circle (2.5pt);
\draw[color=qqqqff] (5.008849116535486,-3.5567082927133415) node {$D$};
\draw [fill=qqqqff] (10.405791538575313,-3.864254643969287) circle (2.5pt);
\draw[color=qqqqff] (10.62405023946663,-3.4574997923081976) node {$E$};
\draw [fill=qqqqff] (7.793351247985521,-1.0466872970753194) circle (2.5pt);
\draw[color=qqqqff] (8.04462922893289,-0.7193451811262304) node {$F$};
\draw [fill=qqqqff] (2.947598748764495,-2.9279394436025035) circle (2.5pt);
\draw[color=qqqqff] (3.2230961092428965,-2.6241483889049904) node {$P$};
\end{tikzpicture}
    \end{figure}

\end{oppgave}

\begin{oppgave}[4.2.4]
    La $\triangle ABC$ og $\triangle DEF$ være to rettvinklede trekanter, der vinklene ved henholdsvis $C$ og $F$ er rette. 
    Anta videre at $\overline{AB}\cong \overline{DE}$ og $\overline{BC}\cong \overline{EF}$. 
    Vi skal vise at $\triangle DEF \cong \triangle ABC$. 

    En måte å vise dette på er å kombinere de to trekantene til å lage en likebeint trekant, altså slik: 

    \begin{figure}[H]
        \centering
        
\definecolor{qqqqff}{rgb}{0,0,1}
\begin{tikzpicture}[line cap=round,line join=round,>=triangle 45,x=1cm,y=1cm]
\clip(-2,-3) rectangle (12,2);
\draw [line width=2pt] (2.04,-2.44)-- (4.04,-2.44);
\draw [line width=2pt] (8.001856621180085,-2.4456621180088405)-- (10.001856621180085,-2.4456621180088405);
\draw [line width=2pt] (4.04,-2.44)-- (6.04,-2.44);
\draw [line width=2pt] (4.04,1.0753834326350167)-- (2.04,-2.44);
\draw [line width=2pt] (4.04,1.0753834326350167)-- (6.04,-2.44);
\draw [line width=2pt] (4.04,-2.44)-- (4.04,1.0753834326350167);
\draw [line width=2pt] (8.001856621180085,1.0753834326350167)-- (8.001856621180085,-2.4456621180088405);
\draw [line width=2pt] (10.001856621180085,-2.4456621180088405)-- (8.001856621180085,1.0753834326350167);
\draw [fill=qqqqff] (2.04,-2.44) circle (2.5pt);
\draw[color=qqqqff] (1.9438131847011337,-2.1612108399000576) node {$P$};
\draw [fill=qqqqff] (4.04,-2.44) circle (2.5pt);
\draw[color=qqqqff] (4.342429367672495,-2.2073380741879682) node {$C$};
\draw [fill=qqqqff] (8.001856621180085,-2.4456621180088405) circle (2.5pt);
\draw[color=qqqqff] (7.7712204497405315,-2.191962329425331) node {$F$};
\draw [fill=qqqqff] (10.001856621180085,-2.4456621180088405) circle (2.5pt);
\draw[color=qqqqff] (10.308218335575624,-2.2227138189506053) node {$D$};
\draw [fill=qqqqff] (6.04,-2.44) circle (2.5pt);
\draw[color=qqqqff] (6.233645973476838,-2.191962329425331) node {$A$};
\draw [fill=qqqqff] (4.04,1.0753834326350167) circle (2.5pt);
\draw[color=qqqqff] (4.234799154334037,1.344458965981163) node {$B$};
\draw [fill=qqqqff] (8.001856621180085,1.0753834326350167) circle (2pt);
\draw[color=qqqqff] (8.170989813569092,1.329083221218526) node {$E$};
\end{tikzpicture}
    \end{figure}

    Fra teorem 3.2.23 kan vi finne et punkt $P$ på linjen $\overleftrightarrow{AC}$ slik at $CP=DF$ og $P\ast C\ast A$. 
    Da gir side-vinkel-side-postulatet oss at $\triangle PBC\cong \triangle DEF$, ettersom vi har $\overline{BC}\cong \overline{EF}$, $\angle BCP \cong \angle EFD$ og $\overline{CP}\cong \overline{FD}$, der kongruens av de to vinklene følger fra lineære par-teoremet. 
    Dette er fordi $\angle ACB$ og $\angle BCP$ er et lineært par, der $\mu(\angle ACB)=90$, så dermed må vi ha $\mu(\angle BCP)=180-\mu(\angle ACB)=90$. 
    
    Siden vi nå har vist at $\triangle PBC\cong \triangle DEF$ kan vi fullføre beviset ved å vise at $\triangle PBC\cong \triangle ABC$. 
    Som en konsekvens av kongruensen vi nettopp viste får vi spesielt at $\overline{PB}\cong \overline{DE}$, og siden vi har antatt fra starten at $\overline{DE}\cong \overline{AB}$ ser vi at vi må ha $\overline{PB}\cong \overline{AB}$. 
    Med andre ord er trekanten $\triangle PAB$ likebeint. 

    Fra likebeint trekant-teoremet får vi at $\angle CPB \cong \angle BAC$. Nå vet vi altså følgende 
    \begin{itemize}
        \item $\angle BCP \cong \angle BCA$
        \item $\overline{BP}\cong \overline{BA}$
        \item $\angle CPB \cong \angle BAC$
    \end{itemize}
    Fra vinkel-side-vinkel-postulatet gir dette oss at $\triangle ABC\cong \triangle PBC$, som fullfører beviset. 
\end{oppgave}