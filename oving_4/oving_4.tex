
\begin{oppgave}[3.6.1]
    La $A$, $B$, $C$, være tre punkter som ikke ligger på linje. Vi skal finne en stråle $\overrightarrow{AB}$ slik at $M$ ligger i det indre av vinkelen $\angle BAC$ og $\mu(\angle BAM)=\mu(\angle MAC)$.
    Vi bruker først teorem 3.2.23 til å finne et punkt $C'$ på $\overrightarrow{AC}$ slik at $AB=AC'$. 
    Vi bruker så teorem 3.2.22 til å finne midtpunktet $M$ på linjestykket $\overline{BC'}$.
    Det at punktet $M$ ligger i det indre av vinkelen vår får vi fra teorem 3.3.10. 
    Rent intuitivt burde strålen $\overrightarrow{AM}$ være vinkelhalveringsstrålen vi søker, og vi skal nå vise at dette faktisk stemmer. 

    Trekanten $\triangle ABC$ er likebeint per konstruksjon, og av teorem 3.6.5 følger det at $\angle BC'A \cong \angle C'BA$. 
    Nå kan vi bruke side-vinkel-side postulatet på trekantene $\triangle ABM$ og $\triangle AMC'$ til å konkludere med at $\triangle ABM \cong \triangle AMC'$. 
    Dette betyr da spesielt at vi har $\mu(\angle BAM) = \mu(\angle MAC')$.
    Siden $C'$ ligger på strålen $\overrightarrow{AC}$ har vi også $\mu(\angle MAC') = \mu(\angle MAC)$.
    Dermed kan vi konkludere med at $\mu(\angle BAM) = \mu(\angle MAC)$, som viser at $\overrightarrow{AM}$ er en vinkelhalveringsstråle til vinkelen. 

    Unikhet følger ettersom både punktet $C'$ og midtpunktet $M$ er unike. 

    \begin{figure}
        
    \end{figure}
\end{oppgave}

\begin{oppgave}[3.6.2]
    Vi skal vise ved eksempel at side-vinkel-side postulatet ikke holder for $\R^2$ med kvadratmetrikken.
    Med andre ord må vi finne to trekanter $\triangle ABC$ og $\triangle DEF$ slik at $\overline{AB}\cong \overline{DE}$, $\angle BAC \cong \angle DEF$ og $\overline{AC}\cong \overline{EF}$, men ikke $\triangle ABC\cong\triangle DEF$.
    Et slikt eksempel er vist ved følgende to figurer.
    
    \begin{figure}
        
    \end{figure}
    
    Vi har $AB=DE=1$, $\mu(\angle BAC) = \mu(\angle DEF) = 90$ og $AC=EF=1$ (husk at vi måler avstand i kvadratmetrikken her). 
    Men vi har $BC=2$ og $DF=1$, så vi kan ikke ha $\triangle ABC\cong \triangle DEF$. 
\end{oppgave}

\begin{oppgave}[3.7.2]
\begin{punkt}
    Eksistenspostulatet (aksiom 3.1.1) sier at det finnes minst to punkter. 
    Insidenspostulatet (aksiom 3.1.3) sier at to punkter alltid bestemmer en unik linje. 
    Sammen impliserer disse aksiomene at vi har minst en linje. 
\end{punkt}

\begin{punkt}
    Av eksistenspostulatet kan vi finne to punkter $A$ og $B$. 
    Av planseparasjonspostulatet (aksiom ???) deler linja $\overleftrightarrow{AB}$ punktene utenfor linja inn i to disjunkte, ikke-tomme halvplan $H_1$ og $H_2$.
    Siden $H_1$ ikke er tomt finnes det et punkt $C$ i $H_1$. 
    Siden $H_1$ består av av punkter som ikke ligger på $\overleftrightarrow{AB}$ kan ikke $A$, $B$ og $C$ være kolineære.
\end{punkt}

\begin{punkt}
    Fra a) vet vi at det finnes en linje $l$. 
    Av linjalpostulatet finnes det en koordinatfunksjon $f_l\longrightarrow \R$. 
    Siden $f$ er en en-til-en-korrespondanse (bijeksjon) mellom $l$ og den uendelige mengden $\R$, må også $l$ ha uendelig mange elementer.
\end{punkt}

\begin{punkt}
    Vi vet fra a) at det finnes en linje $l$, og planseparasjonspostulatet gir oss, som i b), at det finnes et punkt $P$ som ikke ligger på denne linjen.
    Hvis $A$ er et punkt på $l$, så bestemmer $A$ og $P$ en linje $\overleftrightarrow{AP}$. 
    
    \textbf{Påstand:} Dersom $A$ og $B$ er to ulike punkter på $l$, vil $\overleftrightarrow{AP}\neq \overleftrightarrow{BP}$. 
    La oss vise dette. 
    Dersom vi har $\overleftrightarrow{AP} = \overleftrightarrow{BP}$ ligger både punktet $A$ og $B$ på linja $\overleftrightarrow{AP}$. 
    Men, vi vet også at $A$ og $B$ ligger på $l$. 
    Fra insidenspostulatet vet vi at to punkter bestemmer en unik linje, alstå må vi ha $\overleftrightarrow{AB}=l$. 
    Dette betyr at $P$ ligger på $l$, noe som er en selvmotsigelse til antagelsen vår om at $P$ var et punkt som ikke lå på $l$. 

    Dermed bestemmer alle punkter $A$ på en linje en ny linje $\overleftrightarrow{AP}$.
    Fra c) vet vi at en linje inneholder uendelig mange punkt, så vi få da uendelig mange linjer $\overleftrightarrow{AP}$ ved å variere $A$ langs linjen $l$.
\end{punkt}

\begin{punkt}
    Aksiom 1 for insidensgeometri er det samme som insidenspostulatet for nøytral geometri. 
    Aksiom 2 følger fra c), der vi viste at det var uendelig mange punkter på en linje.
    Aksiom 3 følger fra b), da vi viste at det fantes tre punkter $A$, $B$, $C$ som ikke alle var kolineære.
    
    I enhver modell for nøytral geometri må aksiomene for nøytral geometri være oppfylt, og siden aksiomene for insidensgeometri følger fra disse må også aksiomene for insidensgeometri være sanne i en slik modell. 
    Med andre ord, enhver modell for nøytral geometri er en modell for insidensgeometri.
\end{punkt}

\begin{punkt}
    Et teorem i insidensgeometri er et utsagn som kan deduseres logisk fra aksiomene for insidensgeometri. 
    Siden disse aksiomene kan deduseres fra aksiomene i nøytral geometri, kan vi også dedusere alle teoremer i insidensgeometri fra aksiomene i nøytral geometri. 
    Derfor er alle teorem i insidensgeometri også teoremer i nøytral geometri. 
\end{punkt}
\end{oppgave}

\begin{oppgave}[4.1.1]
    
\end{oppgave}

\begin{oppgave}[4.2.1]

\end{oppgave}

\begin{oppgave}[4.2.2]

\end{oppgave}

\begin{oppgave}[4.2.4]

\end{oppgave}