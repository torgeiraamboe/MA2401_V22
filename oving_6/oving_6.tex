
\begin{oppgave}[4.4.1]
    Vi skal vise at dersom $\angle CBP\cong \angle FEP$, så må $l$ og $m$ være paralelle. 
    Se figuren for oppsettet av punkter og linjer. 

    Av toppvinkelteoremet (teorem 3.5.13) får vi at $\angle FEP\cong \angle DEB$. 
    Fra dette gir alternerende-indre-vinkel-teoremet oss at $m$ og $l$ er paralelle, som var det vi ville vise. 

    \begin{figure}[H]
        \centering
        
\definecolor{qqqqff}{rgb}{0,0,1}
\begin{tikzpicture}[line cap=round,line join=round,>=triangle 45,x=1cm,y=1cm]
\clip(-7,-2.5) rectangle (6,4.5);
\draw [shift={(-1,3)},line width=2pt,color=qqqqff,fill=qqqqff,fill opacity=0.10000000149011612] (0,0) -- (-71.56505117707798:0.6) arc (-71.56505117707798:0:0.6) -- cycle;
\draw [shift={(0,0)},line width=2pt,color=qqqqff,fill=qqqqff,fill opacity=0.10000000149011612] (0,0) -- (-71.56505117707798:0.6) arc (-71.56505117707798:0:0.6) -- cycle;
\draw [line width=2pt,domain=-14.98:15.42] plot(\x,{(--27-0*\x)/9});
\draw [line width=2pt,domain=-14.98:15.42] plot(\x,{(-0-0*\x)/10});
\draw [line width=2pt,domain=-14.98:15.42] plot(\x,{(-0-3*\x)/1});
\draw [fill=qqqqff] (0,0) circle (2pt);
\draw[color=qqqqff] (0.32,0.39) node {$E$};
\draw [fill=qqqqff] (-5,0) circle (2.5pt);
\draw[color=qqqqff] (-4.68,0.43) node {$D$};
\draw [fill=qqqqff] (5,0) circle (2.5pt);
\draw[color=qqqqff] (5.32,0.43) node {$F$};
\draw [fill=qqqqff] (-1,3) circle (2.5pt);
\draw[color=qqqqff] (-0.68,3.43) node {$B$};
\draw [fill=qqqqff] (4,3) circle (2.5pt);
\draw[color=qqqqff] (4.32,3.43) node {$C$};
\draw [fill=qqqqff] (-5,3) circle (2.5pt);
\draw[color=qqqqff] (-4.68,3.43) node {$A$};
\draw[color=qqqqff] (0,2.53) node {$\alpha$};
\draw[color=qqqqff] (1,-0.55) node {$\alpha$};
\draw [fill=qqqqff] (0.436,-1.308) circle (2.5pt);
\draw[color=qqqqff] (0.94,-1.35) node {$P$};
\draw [fill=qqqqff] (-8,3) circle (2.5pt);
\draw[color=qqqqff] (-6.08,3.29) node {$l$};
\draw [fill=qqqqff] (-8,0) circle (2.5pt);
\draw[color=qqqqff] (-6.1,0.19) node {$m$};
\end{tikzpicture}
    \end{figure}
\end{oppgave}

\begin{oppgave}[4.4.2]
    Vi bruker igjen figuren fra forrige oppgave til oppsettet av punkter og linjer. 
    Vi skal nå vise at dersom $\angle CBP$ og $\angle FEB$ er supplementærvinkler, så er $m$ og $l$ paralelle. 

    Anta at $\angle CBP$ og $\angle FEB$ er supplementærvinkler. 
    Per definisjon betyr dette at $\mu(\angle CBP)+\mu(\angle FEB) = 180$. 
    Av lineært par-teoremet får vi også at $\mu(\angle DEB)+\mu(\angle FEB)=180$. 
    Dermed må vi ha 
    $$\mu(\angle CBP) = 180 - \mu(\angle FEB) = \mu(\angle DEB),$$
    eller med andre ord $\angle CBP\cong \angle DEB$.
    Det alternerende-indre-vinkel-teoremet gir oss da at $l$ og $m$ er paralelle. 
\end{oppgave}

\begin{oppgave}[4.4.3]
    La $l$, $m$ og $n$ være tre linjer slik at $m\perp l$ og $n\perp l$. 
    Vi vil vise at vi enten har $m=n$ eller $m \parallel n$. 

    La $A$ være punktet der linjene $l$ og $m$ skjærer hverandre, og $B$ være punktet der linjene $l$ og $m$ skjærer hverandre.
    Vi har to tilfeller: enten er $A=B$, eller så er $A\neq B$. 
    Dersom $A=B$ vet vi at $n=m$ grunnet unikhet av vinkelrette linjer gjennom et gitt punkt. 
    Siden linjene $m$ og $n$ begge står vinkelrette på $l$ kan vi konkludere med at $m\parallel n$ fra oppgave 4.4.1.  

    \begin{figure}[H]
        \centering
        
\definecolor{qqqqff}{rgb}{0,0,1}
\begin{tikzpicture}[line cap=round,line join=round,>=triangle 45,x=1cm,y=1cm]
\clip(-6,-2) rectangle (6,4.5);
\draw[line width=2pt,color=qqqqff,fill=qqqqff,fill opacity=0.10000000149011612] (3,-0.42426406871192873) -- (3.4242640687119286,-0.42426406871192884) -- (3.4242640687119286,0) -- (3,0) -- cycle; 
\draw[line width=2pt,color=qqqqff,fill=qqqqff,fill opacity=0.10000000149011612] (-3,-0.42426406871192873) -- (-2.5757359312880714,-0.42426406871192884) -- (-2.5757359312880714,0) -- (-3,0) -- cycle; 
\draw [line width=2pt,domain=-14.9:18.16] plot(\x,{(-0-0*\x)/6});
\draw [line width=2pt] (-3,5)-- (-3,-3);
\draw [line width=2pt] (3,0)-- (3,-3);
\draw [line width=2pt] (3,0)-- (3,5);
\draw [fill=qqqqff] (-3,0) circle (2.5pt);
\draw[color=qqqqff] (-2.68,0.43) node {$A$};
\draw [fill=qqqqff] (3,0) circle (2.5pt);
\draw[color=qqqqff] (3.32,0.43) node {$B$};
\draw [fill=qqqqff] (-3,5) circle (2.5pt);
\draw[color=qqqqff] (-2.58,3.83) node {$m$};
\draw [fill=qqqqff] (-3,-3) circle (2.5pt);
\draw[color=qqqqff] (-2.84,-2.57) node {$D$};
\draw [fill=qqqqff] (3,-3) circle (2.5pt);
\draw[color=qqqqff] (3.16,-2.57) node {$E$};
\draw[color=qqqqff] (3.8,-0.71) node {$90$};
\draw[color=qqqqff] (-2.36,-0.81) node {$90$};
\draw [fill=qqqqff] (3,5) circle (2.5pt);
\draw[color=qqqqff] (3.44,3.83) node {$n$};
\draw [fill=qqqqff] (7,0) circle (2.5pt);
\draw[color=qqqqff] (5.72,0.29) node {$l$};
\end{tikzpicture}

    \end{figure}
\end{oppgave}

\begin{oppgave}[4.5.1]
    La $\triangle ABC$ være en trekant, og la $D$ være et punkt på $\overleftrightarrow{AB}$ slik at $A\ast B\ast D$. 
    Vi vil vise at $\mu(\angle CAB)+\mu(\angle BCA)\leq \mu(\angle DBC)$. 

    \begin{figure}[H]
        \centering
        
\definecolor{qqqqff}{rgb}{0,0,1}
\begin{tikzpicture}[line cap=round,line join=round,>=triangle 45,x=1.5cm,y=1.5cm]
\clip(-1,-0.3) rectangle (7,4);
\draw [shift={(0,0)},line width=2pt,color=qqqqff,fill=qqqqff,fill opacity=0.10000000149011612] (0,0) -- (0:0.6) arc (0:45:0.6) -- cycle;
\draw [shift={(4,0)},line width=2pt,color=qqqqff,fill=qqqqff,fill opacity=0.10000000149011612] (0,0) -- (0:0.6) arc (0:108.43494882292202:0.6) -- cycle;
\draw [shift={(3,3)},line width=2pt,color=qqqqff,fill=qqqqff,fill opacity=0.10000000149011612] (0,0) -- (-135:0.6) arc (-135:-71.56505117707799:0.6) -- cycle;
\draw [line width=2pt] (0,0)-- (3,3);
\draw [line width=2pt,domain=-14.2:18.86] plot(\x,{(-0-0*\x)/6});
\draw [line width=2pt] (4,0)-- (3,3);
\draw [fill=qqqqff] (6,0) circle (2.5pt);
\draw[color=qqqqff] (6.32,0.43) node {$D$};
\draw [fill=qqqqff] (3,3) circle (2.5pt);
\draw[color=qqqqff] (3.32,3.43) node {$C$};
\draw [fill=qqqqff] (0,0) circle (2pt);
\draw[color=qqqqff] (-0.28,0.35) node {$A$};
\draw [fill=qqqqff] (4,0) circle (2.5pt);
\draw[color=qqqqff] (3.36,0.43) node {$B$};
\end{tikzpicture}
    \end{figure}

    Fra Sacherri-Legendres teorem vet vi at $$\mu(\angle CAB)+\mu(\angle BCA)+\mu(\angle ABC)\leq 180.$$
    Dermed har vi $$\mu(\angle CAB)+\mu(\angle BCA)\leq 180-\mu(\angle ABC).$$
    Fra lineært par-teoremet vet vi også at $\mu(\angle DBC)=180-\mu(\angle ABC)$. 
    Ved å kombinere de to forrige ligningene får vi at
    $$\mu(\angle CAB)+\mu(\angle BCA)\leq \mu(\angle DBC),$$
    som var det vi ville vise. 
\end{oppgave}

\begin{oppgave}[4.5.2]
    La $l$ og $m$ være to ulike linjer som begge skjæres av en linje $t$. 
    Vi vil vise at summen av vinkelmålene til de indre vinklene på den siden av $t$ der $l$ og $m$ skjærer hverandre er strengt mindre en $180$. 
    Det kan være enklere å skjønne hva vi vil vise ved å se på figuren under.  

    Vi fikserer ført litt notasjon, slik at det er klart hva vi faktisk vil vise. 
    La $B$ være skjæringspunktet mellom $t$ og $l$, og $B'$ være skjæringspunktet til $t$ og $m$. 
    La også $D$ være skjæringspunktet til $l$ og $m$. 
    Det vi da vil vise er at $$\mu(\angle DBB')+\mu(\angle BB'D)< 180.$$

    \begin{figure}[H]
        \centering
        
\definecolor{qqqqff}{rgb}{0,0,1}
\begin{tikzpicture}[line cap=round,line join=round,>=triangle 45,x=1cm,y=1cm]
\clip(-4,-5) rectangle (6.3,2.5);
\draw [line width=2pt,domain=-13.62:19.5] plot(\x,{(-0-0*\x)/5});
\draw [line width=2pt,domain=-13.62:19.5] plot(\x,{(-0-4*\x)/-2});
\draw [line width=2pt,domain=-13.62:19.5] plot(\x,{(-20--4*\x)/7});
\draw [fill=qqqqff] (0,0) circle (2pt);
\draw[color=qqqqff] (0.58,0.31) node {$B$};
\draw [fill=qqqqff] (5,0) circle (2.5pt);
\draw[color=qqqqff] (5.32,0.43) node {$D$};
\draw [fill=qqqqff] (-2,-4) circle (2.5pt);
\draw[color=qqqqff] (-1.28,-4.09) node {$B'$};
\draw [fill=qqqqff] (-5,0) circle (2.5pt);
\draw[color=qqqqff] (-3.1,0.23) node {$l$};
\draw [fill=qqqqff] (1.468,2.936) circle (2.5pt);
\draw[color=qqqqff] (1.34,1.83) node {$t$};
\draw [fill=qqqqff] (-4.772,-5.584) circle (2.5pt);
\draw[color=qqqqff] (-3.1,-4.31) node {$m$};
\end{tikzpicture}
    \end{figure}

    Ved å anvende Saccheri-Legendres teorem på trekanten $\triangle DBB'$ får vi 
    $$\mu(\angle DBB')+\mu(\angle BB'D)+\mu(\angle B'DB)\leq 180.$$
    Siden $\mu(\angle B'DB)>0$ fra gradskivepostulatet følger resultatet vi ønsker å vise umiddelbart ved å trekke fra $\mu(\angle B'DB)$ fra ligningen over. 
\end{oppgave}

\begin{oppgave}[4.6.1]
    La $\square ABCD$ være en konveks firkant. Vi vil vise at
    $$\mu(\angle ABC) + \mu(\angle BCD) + \mu(\angle CDA) + \mu(\angle DAB) \leq 360.$$

    \begin{figure}[H]
        \centering
        
\definecolor{qqqqff}{rgb}{0,0,1}
\begin{tikzpicture}[line cap=round,line join=round,>=triangle 45,x=1.3cm,y=1.3cm]
\clip(-2,-1.5) rectangle (7.5,5);
\draw [line width=2pt] (0,0)-- (2,4);
\draw [line width=2pt] (2,4)-- (7,2);
\draw [line width=2pt] (7,2)-- (4,-1);
\draw [line width=2pt] (4,-1)-- (0,0);
\draw [line width=2pt] (0,0)-- (7,2);
\draw [fill=qqqqff] (0,0) circle (2pt);
\draw[color=qqqqff] (-0.32,0.39) node {$A$};
\draw [fill=qqqqff] (4,-1) circle (2.5pt);
\draw[color=qqqqff] (3.98,-0.61) node {$B$};
\draw [fill=qqqqff] (7,2) circle (2.5pt);
\draw[color=qqqqff] (7.28,2.33) node {$C$};
\draw [fill=qqqqff] (2,4) circle (2.5pt);
\draw[color=qqqqff] (2.32,4.43) node {$D$};
\end{tikzpicture}
    \end{figure}

    Vi deler firkanten inn i to trekanter, $\triangle ABC$ og $\triangle CDA$. 
    Merk at dette faktisk gir oss to trekanter da disse punktene ikke kan ligge på linje per definisjon av en firkant. 
    Ved å anvende Saccheri-Legendre teoremet på trekantene $\triangle ABC$ og $\triangle CDA$ får vi 
    $$\mu(\angle ABC)+\mu(\angle BCA)+\mu(\angle CAB)\leq 180$$
    og
    $$\mu(\angle CDA)+\mu(\angle DAC)+\mu(\angle ACD)\leq 180.$$
    Legger vi disse to ligningene sammen får vi at 
    $$\mu(\angle ABC)+\mu(\angle BCA)+\mu(\angle CAB)+\mu(\angle CDA)+\mu(\angle DAC)+\mu(\angle ACD)\leq 360.$$
    Siden firkanten er konveks vet vi at $A$ ligger i det indre av vinkelen $\angle BCD$ og at $C$ ligger i det indre av vinkelen $\angle DAB$. 
    Fra del 4 av gradskivepostulatet får vi at 
    $$\mu(\angle BCA)+\mu(\angle ACD)=\mu(\angle BCD),$$
    og 
    $$\mu(\angle DAC)+\mu(\angle CAB)=\mu(\angle DAB).$$

    Setter vi disse likhetene i den vi hadde over, får vi nøyaktig det vi var ute etter, altså 
    $$\mu(\angle ABC) + \mu(\angle BCD) + \mu(\angle CDA) + \mu(\angle DAB) \leq 360.$$
\end{oppgave}

\begin{oppgave}[4.6.2]
    Anta at $\square ABCD$ er et paralellogram, altså at $\overleftrightarrow{AB}\parallel \overleftrightarrow{CD}$ og $\overleftrightarrow{AD}\parallel \overleftrightarrow{BC}$.
    Vi vil vise at $\square ABCD$ er konveks, som for firkanter betyr at hvert hjørne ligger i det indre av vinkelen definert av de tre andre hjørnene. 
    Vi viser at dette stemmer kun for ett av hjørnene, $A$, da beviset for de tre andre er helt likt. 

    Vi minner oss selv på hva det vil si å ligge i det indre av $\angle BCD$: 
    det betyr at $A$ og $B$ ligger på samme side av $\overleftrightarrow{CD}$, og at $A$ og $D$ ligger på samme side av $\overleftrightarrow{BC}$. 
    La oss vise at dette stemmer. 

    Siden vi vet at $\overleftrightarrow{AD}\parallel \overleftrightarrow{BC}$ vet vi at $\overline{AD}\cap \overline{BC}=\emptyset$. 
    Dette betyr at $A$ og $D$ ligger på samme side av $\overleftrightarrow{BC}$. Se proposisjon 3.3.4 hvis dette er uklart. 

    På nøyaktig samme måte får vi at $\overline{AB}\cap \overline{CD}=\emptyset$, som gir oss at $A$ og $B$ liger på samme side av $\overleftrightarrow{CD}$.
    
    Dermed har vi at $A$ ligger i det indre av vinkelen $\angle BCD$. 
    At de andre hjørnene ligger i det indre av de andre respektive vinklene bevises på tilsvarende måte. 
    Dermed er firkanten $\square ABCD$ konveks. 
\end{oppgave}

\begin{oppgave}[4.6.5]
    La $\triangle ABC$ være en trekant, $D$ et punkt mellom $A$ og $B$, og $E$ et punkt mellom $A$ og $C$. 
    
    \begin{figure}[H]
        \centering
        
\definecolor{qqqqff}{rgb}{0,0,1}
\begin{tikzpicture}[line cap=round,line join=round,>=triangle 45,x=1.3cm,y=1.3cm]
\clip(-1,-0.3) rectangle (7,5);
\draw [line width=2pt] (0,0)-- (2,2);
\draw [line width=2pt] (2,2)-- (4,4);
\draw [line width=2pt] (4,4)-- (6,0);
\draw [line width=2pt] (6,0)-- (3,0);
\draw [line width=2pt] (3,0)-- (0,0);
\draw [line width=2pt] (2,2)-- (3,0);
\draw [fill=qqqqff] (0,0) circle (2pt);
\draw[color=qqqqff] (-0.18,0.41) node {$A$};
\draw [fill=qqqqff] (3,0) circle (2.5pt);
\draw[color=qqqqff] (3.32,0.43) node {$D$};
\draw [fill=qqqqff] (6,0) circle (2.5pt);
\draw[color=qqqqff] (6.32,0.43) node {$B$};
\draw [fill=qqqqff] (2,2) circle (2.5pt);
\draw[color=qqqqff] (2,2.37) node {$E$};
\draw [fill=qqqqff] (4,4) circle (2.5pt);
\draw[color=qqqqff] (4.32,4.43) node {$C$};
\end{tikzpicture}
    \end{figure}

    Vi skal vise at $\square BCED$ er en konveks firkant, altså at hjørnene ligger i det indre av vinklene definert av de resterende hjørnene. 

    \begin{itemize}
        \item $E$ ligger i det indre av $\angle DBC$:
        Vi bemerker oss først at $\angle DBC = \angle ABC$, siden $\overrightarrow{BD} = \overrightarrow{BA}$. 
        Siden vi vet at $E$ ligger mellom $A$ og $C$, vet vi at strålen $\overrightarrow{BE}$ skjærer det indre av linjestykket $\overline{AC}$. 
        Fra teorem 3.5.3 kan vi konkludere med at $E$ ligger i det indre av $\angle ABC = \angle DBC$.  
        \item $D$ ligger i det indre av $\angle BCE$: 
        Agumentet er veldig likt det over. Vi har at $\angle BCE=\angle BCA$, og teorem 3.5.3 gir oss at $D$ ligger i det indre av $\angle BCA$ ettersom $\overrightarrow{CD}$ skjærer $\overline{AB}$ i punktet $D$. 
        \item $B$ ligger i det indre av $\angle CED$: 
        Av definisjonen av det indre av en vinkel, må vi vise at $B$ og $D$ ligger på samme side av $\overleftrightarrow{EC}$, og at $B$ og $C$ ligger på samme side av $\overleftrightarrow{ED}$. 
        Siden både $B$ og $D$ ligger på strålen $\overrightarrow{AB}$ gir stråleteoremet (teorem 3.3.9) at $B$ og $D$ ligger på samme side av $\overleftrightarrow{AC}=\overleftrightarrow{EC}$. 
        Merk så at $B$ og $A$ ligger på motsatt side av $\overleftrightarrow{ED}$, siden $\overline{AB}$ skjærer $\overleftrightarrow{ED}$ i punktet $D$. 
        På samme vis ligger $C$ og $A$ på motsatt side av $\overleftrightarrow{ED}$, ettersom $\overline{AC}$ skjærer $\overleftrightarrow{ED}$ i punktet $E$. 
        Siden $A$ ligger på motsatt side av både $B$ og $C$ kan vi konkludere med at $B$ og $C$ ligger på samme side av $\overleftrightarrow{ED}$. 
        \item $C$ ligger i det indre av $\angle EDB$: Denne bevises på samme måte som forrige punkt. 
    \end{itemize}
    Siden alle hjørnene ligger i det indre av vinkelen definert av de resterende hjørnene er firkanten $\square BCED$ konveks. 
\end{oppgave}

\begin{oppgave}[4.7.1]
    Anta Euklids femte postulat. La $l$ være en linje og $P$ et punkt som ikke ligger på $l$. 
    Vi vil vise at det finnes en unik linje $m$ slik at $P\in m$ og $m\parallel l$. 

    Vi kan konstruere en linje $n$ vinkelrett på $l$ slik at $P\in n$. 
    La punktet der $l$ og $n$ skjærer hverandre være $Q$. 
    Fra gradskivepostulatet kan vi finne en linje $m$ slik at $m\perp \overleftrightarrow{PQ}$ og $P\in m$. 
    Fra oppgave 4.4.3 vet vi nå at $m\parallel l$. 

    Vi må vise at denne linjen er unik. 
    Anta at det finnes en annen linje $m'\neq m$ slik at $P\in m'$. 
    Vi viser at $m' \nparallel l$. 

    Linjen $\overleftrightarrow{PQ}$ er transversal for $l$ og $m'$. 
    Siden vi har antatt at $m'\neq m$ har vi at de indre vinklene dannet av $m'$ og $\overleftrightarrow{PQ}$ ikke er rette vinkler. 
    Men, de to indre vinklene dannet av $m'$ og $\overleftrightarrow{PQ}$ er supplementærvinkler, så en av vinklene må ha vinkelmål større enn $90$ og en må ha vinkelmål mindre enn $90$.
    Fra Euklids femte postulat må linjene $m'$ og $l$ skjære hverandre på den siden av $\overleftrightarrow{PQ}$ der vinkelmålet er mindre enn $90$. 
    Siden $m'$ og $l$ skjærer hverandre er de ikke paralelle. Altså er den paralelle linjen $m$ unik. 

    \begin{figure}[H]
        \centering
        
\definecolor{qqqqff}{rgb}{0,0,1}
\begin{tikzpicture}[line cap=round,line join=round,>=triangle 45,x=1cm,y=1cm]
\clip(-4,-2) rectangle (6.8,6);
\draw[line width=2pt,color=qqqqff,fill=qqqqff,fill opacity=0.10000000149011612] (0,0.42426406871192873) -- (-0.4242640687119287,0.4242640687119288) -- (-0.42426406871192873,0) -- (0,0) -- cycle; 
\draw [line width=2pt,domain=-10.54:18.26] plot(\x,{(-0-0*\x)/-7});
\draw [line width=2pt] (0,-5.2) -- (0,8.98);
\draw [line width=2pt,domain=-10.54:18.26] plot(\x,{(-20-0*\x)/-5});
\draw [line width=2pt,domain=-10.54:18.26] plot(\x,{(-20--1*\x)/-5});
\draw [fill=qqqqff] (0,0) circle (2pt);
\draw[color=qqqqff] (0.42,-0.51) node {$Q$};
\draw [fill=qqqqff] (0,4) circle (2.5pt);
\draw[color=qqqqff] (0.32,4.43) node {$P$};
\draw [fill=qqqqff] (7,0) circle (2.5pt);
\draw[color=qqqqff] (5.72,0.27) node {$l$};
\draw [fill=qqqqff] (-5,4) circle (2.5pt);
\draw[color=qqqqff] (-4.84,4.43) node {$D$};
\draw [fill=qqqqff] (-5,5) circle (2.5pt);
\draw[color=qqqqff] (-4.84,5.43) node {$E$};
\draw [fill=qqqqff] (7,2.6) circle (2.5pt);
\draw[color=qqqqff] (5.94,2.39) node {$m'$};
\draw [fill=qqqqff] (7,4) circle (2.5pt);
\draw[color=qqqqff] (5.74,4.19) node {$m$};
\draw[color=qqqqff] (-0.74,0.57) node {$90$};
\end{tikzpicture}

    \end{figure}
\end{oppgave}