

\begin{oppgave}[5.1.1]
    La $l$ og $l'$ være paralelle linjer, og $t$ en tredje linje som skjærer både $l$ og $l'$ transversalt. 
    Vi ønsker å vise at alle fire par av vinkler er kongruente. 

    La $B$ og $B'$ være skjæringspunktene mellom respektivt $l$ og $t$ og $l'$ og $t$. 
    La videre $A, C\in l$ slik at $A\ast B\ast C$, $A', C'\in l'$ slik at $A'\ast B'\ast C'$ og $B''\in t$ slik at $B\ast B'\ast B''$. 

    \begin{figure}[H]
        \centering
        
\definecolor{qqqqff}{rgb}{0,0,1}
\begin{tikzpicture}[line cap=round,line join=round,>=triangle 45,x=1cm,y=1cm]
\clip(-6,-2.3) rectangle (6,4);
\draw [line width=2pt,domain=-14.72:15.68] plot(\x,{(-0-0*\x)/8});
\draw [line width=2pt,domain=-14.72:15.68] plot(\x,{(--24-0*\x)/8});
\draw [line width=2pt,domain=-14.72:15.68] plot(\x,{(--3-3*\x)/1});
\draw [fill=qqqqff] (-4,0) circle (2.5pt);
\draw[color=qqqqff] (-3.6,0.42) node {$A'$};
\draw [fill=qqqqff] (4,0) circle (2.5pt);
\draw[color=qqqqff] (4.4,0.42) node {$C'$};
\draw [fill=qqqqff] (-5,3) circle (2.5pt);
\draw[color=qqqqff] (-4.68,3.42) node {$A$};
\draw [fill=qqqqff] (3,3) circle (2.5pt);
\draw[color=qqqqff] (3.32,3.42) node {$C$};
\draw [fill=qqqqff] (0,3) circle (2pt);
\draw[color=qqqqff] (0.32,3.38) node {$B$};
\draw [fill=qqqqff] (1,0) circle (2.5pt);
\draw[color=qqqqff] (1.4,0.42) node {$B'$};
\draw [fill=qqqqff] (1.451,-1.353) circle (2.5pt);
\draw[color=qqqqff] (1.94,-0.92) node {$B''$};
\draw [fill=qqqqff] (7,3) circle (2.5pt);
\draw[color=qqqqff] (5.72,3.38) node {$l$};
\draw [fill=qqqqff] (7,0) circle (2.5pt);
\draw[color=qqqqff] (5.8,0.4) node {$l'$};
\draw [fill=qqqqff] (2.125,-3.375) circle (2.5pt);
\draw[color=qqqqff] (1.4,-2) node {$t$};
\end{tikzpicture}
 
    \end{figure}

    Fra det vertikale vinkel-teoremet vet vi at $\angle B''B'C'\cong \angle A'B'B$. 
    Fra det motsatte alternerende indre vinkel-teoremet (teorem 5.1.1) får vi $\angle A'B'B\cong \angle B'BC$.
    Dermed følger det at alle de fire parene med vinkler dannet er kongruente. 
\end{oppgave}

\begin{oppgave}[5.1.9]
    La $l$ og $m$ være to paralelle linjer. 
    Vi skal vise at det finnes en linje $t$ slik at $t\perp l$ og $t\perp m$. 
    
    La $P$ være et vilkårlig punkt på $l$. 
    Fra teorem 4.1.3 vet vi at det finnes en unik linje $t$ slik at $P\in t$ og $t\perp l$.
    Det er også klart at $t$ må skjære $m$, for dersom $t$ ikke skjærer $m$ må vi ha $m\parallel t$ og $l\perp t$, noe som ikke er mulig for paralelle linjer i euklidsk geometri.
    Det gjenstår å vise at $t\perp m$. 
    Dette følger fra det motsatte alternerende indre vinkel-teoremet ettersom vinkelen mellom $t$ og $l$ er rett, må vinkelen mellom $t$ og $m$ også være rett. 
\end{oppgave}

\begin{oppgave}[5.3.1]
    Vi har to trekanter $\triangle ABC$ og $\triangle DEF$ slik at $\triangle ABC \sim \triangle DEF$. 
    Vi skal finne $r>0$ slik at $DE=r\cdot AB$, $DF=r\cdot AC$ og $EF=r\cdot BC$. 

    \begin{figure}[H]
        \centering
        
\definecolor{qqqqff}{rgb}{0,0,1}
\begin{tikzpicture}[line cap=round,line join=round,>=triangle 45,x=1cm,y=1cm]
\clip(-6.2,-0.3) rectangle (6.5,4);
\draw [line width=2pt] (-6,0)-- (-2,3);
\draw [line width=2pt] (-2,3)-- (0,0);
\draw [line width=2pt] (0,0)-- (-6,0);
\draw [line width=2pt] (2,0)-- (5,2);
\draw [line width=2pt] (5,2)-- (6,0);
\draw [line width=2pt] (6,0)-- (2,0);
\draw [fill=qqqqff] (-6,0) circle (2.5pt);
\draw[color=qqqqff] (-6,0.42107438016528687) node {$A$};
\draw [fill=qqqqff] (0,0) circle (2pt);
\draw[color=qqqqff] (0.31967541834575075,0.38107438016528694) node {$B$};
\draw [fill=qqqqff] (-2,3) circle (2.5pt);
\draw[color=qqqqff] (-1.6803245816542498,3.421074380165286) node {$C$};
\draw [fill=qqqqff] (2,0) circle (2.5pt);
\draw[color=qqqqff] (2,0.42107438016528687) node {$D$};
\draw [fill=qqqqff] (6,0) circle (2.5pt);
\draw[color=qqqqff] (6.319675418345753,0.42107438016528687) node {$E$};
\draw [fill=qqqqff] (5,2) circle (2.5pt);
\draw[color=qqqqff] (5.319675418345753,2.4210743801652863) node {$F$};
\end{tikzpicture} 
    \end{figure}

    Vi prøver verdien $r=\frac{DE}{AB}$. 
    Vi må da vise at denne verdien oppfyller kravene. 
    Vi får umiddelbart $DE= \frac{DE}{AB}\cdot AB = r\cdot AB$, så vi har vist den første likheten. 
    
    Ettersom $\triangle ABC\sim\triangle DEF$ gir teorem 5.3.1 oss at 
    $$\frac{AB}{AC}=\frac{DE}{DF},$$
    noe som gir oss 
    $$\frac{DF}{AC}=\frac{DE}{AB}.$$
    Vi får da
    $$DF = \frac{DF}{AC}\cdot AC=\frac{DE}{AB}\cdot AC = r\cdot AC,$$
    som var den andre ligningen vi skulle vise. 
    
    Helt tilsvarende bruker vi teorem 5.3.1 til å få 
    $$\frac{BC}{AB}=\frac{EF}{DE},$$
    som igjen gir oss 
    $$\frac{EF}{BC}=\frac{DE}{AB}.$$
    Vi får 
    $$EF=\frac{EF}{BC}\cdot BC = \frac{DE}{AB}\cdot BC = r\cdot BC,$$
    som viser den siste ligningen vi skulle vise. 
\end{oppgave}

\begin{oppgave}[5.3.2]
    Anta at $\triangle ABC$ og $\triangle DEF$ er to trekanter slik at $\angle CAB\cong \angle FDE$ og $\frac{AB}{AC}=\frac{DE}{DF}$. 
    Vi vil vise at $\triangle ABC \sim \triangle DEF$. 

    Vi bemerker oss først at fra ligningen $\frac{AB}{AC}=\frac{DE}{DF}$ får vi $\frac{AB}{DE}=\frac{AC}{DF}$, slik at dersom $AB=DE$ får vi automatisk også $AC=DF$. 
    I dette tilfellet gir side-vinkel-side-postulatet (SVS) oss at $\triangle ABC\cong \triangle DEF$. 
    Vi kan derfor uten tap av generalitet anta at $AB>DE$. 
    Ved linjalpostulatet kan vi finne et punkt $P$ på strålen $\overrightarrow{AB}$ slik at $AP=DE$. 
    Videre lar vi $m$ være den unike linjen som er paralell med $\overleftrightarrow{BC}$ gjennom $P$. 
    Eksistens og unikhet av en slik linje finnes grunnet det euklidske paralellpostulatet. 
    Siden $m$ skjærer $\overline{AB}$ i $P$ og er paralell med $\overleftrightarrow{BC}$ kan vi bruke Paschs aksiom til å konkludere med at linjen $m$ skjærer $\overline{AC}$. 
    Kall dette skjæringspunktet $Q$. 

    \begin{figure}[H]
        \centering
        
\definecolor{qqqqff}{rgb}{0,0,1}
\begin{tikzpicture}[line cap=round,line join=round,>=triangle 45,x=1cm,y=1cm]
\clip(-6.2,-1) rectangle (6.5,4);
\draw [line width=2pt] (-6,0)-- (-2,3);
\draw [line width=2pt] (-2,3)-- (0,0);
\draw [line width=2pt] (0,0)-- (-6,0);
\draw [line width=2pt] (2,0)-- (4.719675418345752,1.9710743801652864);
\draw [line width=2pt] (4.719675418345752,1.9710743801652864)-- (6,0);
\draw [line width=2pt] (6,0)-- (2,0);
\draw [line width=2pt,domain=-16.260324581654256:15.039675418345755] plot(\x,{(-6-3*\x)/2});
\draw [fill=qqqqff] (-6,0) circle (2.5pt);
\draw[color=qqqqff] (-6,0.42107438016528687) node {$A$};
\draw [fill=qqqqff] (0,0) circle (2pt);
\draw[color=qqqqff] (0.31967541834575075,0.38107438016528694) node {$B$};
\draw [fill=qqqqff] (-2,3) circle (2.5pt);
\draw[color=qqqqff] (-1.6803245816542498,3.421074380165286) node {$C$};
\draw [fill=qqqqff] (2,0) circle (2.5pt);
\draw[color=qqqqff] (2.1,0.42107438016528687) node {$D$};
\draw [fill=qqqqff] (6,0) circle (2.5pt);
\draw[color=qqqqff] (6.319675418345753,0.42107438016528687) node {$E$};
\draw [fill=qqqqff] (5,2) circle (2.5pt);
\draw[color=qqqqff] (5.319675418345753,2.4210743801652863) node {$F$};
\draw [fill=qqqqff] (-2,0) circle (2.5pt);
\draw[color=qqqqff] (-1.6803245816542498,0.42107438016528687) node {$P$};
\draw [fill=qqqqff] (-3.3333333333333335,2) circle (2pt);
\draw[color=qqqqff] (-3.2,2.5) node {$Q$};
\draw [fill=qqqqff] (-4.239057277508364,3.3585859162625447) circle (2.5pt);
\draw[color=qqqqff] (-3.920324581654251,3.6) node {$R$};
\draw [fill=qqqqff] (-0.41444189289297906,-2.3783371606605317) circle (2.5pt);
\draw[color=qqqqff] (-1.2203245816542498,-0.6) node {$m$};
\end{tikzpicture} 
    \end{figure}

    Vi kan nå anvende paralellprojeksjonsteoremet (teorem 5.2.1), som i dette tilfelle gir oss $\frac{AP}{AB}=\frac{AQ}{AC}$. 
    Men, siden vi har konstruer $P$ slik at $AP=DE$ gir dette oss at 
    $$AQ=\frac{DE}{AB}\cdot AC = DF,$$
    der den siste likheten følger av den opprinnelige antagelsen $\frac{AB}{AC}=\frac{DE}{DF}$. 
    Vi kan nå bruke side-vinkel-side-postulatet (SVS) til å konkludere med at $\triangle APQ\cong \triangle DEF$, ettersom vi vet at $AP=DE$, $\angle CAB\cong \angle FDE$ og $AQ=DF$. 
    Men, vi kan også bruke det motsatte av alternerende indre vinkel-teoremet til å få $\angle APQ\cong \angle ABC$ og $\angle PQA\cong \angle BCA$, slik at $\triangle ABC\sim\triangle APQ$. 
    Ettersom vi har $\triangle ABC\sim\triangle APQ$ og $\triangle APQ\cong \triangle DEF$ kan vi konkludere med at $\triangle ABC\sim\triangle DEF$.  
\end{oppgave}

\begin{oppgave}[5.3.3]
    La $\triangle ABC$ og $\triangle DEF$ være to trekanter slik at 
    $$\frac{AB}{DE}=\frac{AC}{DF}=\frac{BC}{EF}.$$
    Vi ønsker å vise at $\triangle ABC\sim\triangle DEF$, altså altså at vi har $\angle ABC\cong \angle DEF$, $\angle BCA\cong \angle EFD$ og $\angle CAB\cong \angle FDE$. 

    Vi definerer
    \begin{equation}
        \label{eq:r}
        r=\frac{AB}{DE}=\frac{AC}{DF}=\frac{BC}{EF}
    \end{equation}
    og deler beviset inn i tre tilfeller basert på hvor stor $r$ er: $r=1$, $r>1$ og $r<1$. 

    \begin{enumerate}
        \item $\mathbf{r=1:}$ 
        Fra \cref{eq:r} får vi i dette tilfellet at $AB=DE$, $AC=DF$ og $BC=EF$. 
        Vi kan da bruke SSS (side-side-side) til å konkludere med at $\triangle ABC\cong \triangle DEF$, noe som spesielt betyr $\triangle ABC\sim \triangle DEF$. 
        
        \item $\mathbf{r>1:}$
        \Cref{eq:r} sier oss i dette tilfellet av vi blant annet har $AB>DE$. 
        Vi bruker linjalpostulatet til å finne et punkt $P$ på $\overline{AB}$ slik at $AP=DE$.
        Fra korollar 4.4.6 kan vi finne en unik linje $m$ gjennom $P$ som er paralell med $\overleftrightarrow{BC}$. 
        Vi vet at $m$ skjærer $\overline{AB}$ i punktet $P$, og at $m$ ikke skjærer $\overline{BC}$, så Paschs aksiom sier oss at $m$ må skjære $\overline{AC}$ i et punkt $Q$. 
        Vi ønsker nå å vise to ting: $\triangle APQ\sim \triangle ABC$ og $\triangle APQ\cong \triangle DEF$. 

        Linjen $\overleftrightarrow{AC}$ skjærer de paralelle linjene $m$ og $\overleftrightarrow{BC}$ i henholdsvis $Q$ og $C$. 
        Hvis vi lar $R$ være et punkt på $m$ slik at $R$ og $P$ ligger på motsatt side av $\overleftrightarrow{AC}$, gir det motsatte alternerende indre vinkel-teoremet oss at $\angle BCA\cong \angle CQR$. 
        Men, vi har også at $\angle CQR$ og $\angle PQA$ er toppvinkler, slik at teorem 3.5.13 gir 
        $$\angle BCA\cong \angle CQR \cong \angle PQA.$$
        Et helt tilsvarende argument gir at $\angle APQ\cong \angle ABC$, slik at vi til sammen har $\angle CAB\cong \angle QAP$, $\angle BCA\cong \angle PQA$ og $\angle ABC\cong APQ$, altså $\triangle APQ\sim\triangle ABC$. 

        \begin{figure}[H]
            \centering
            
\definecolor{qqqqff}{rgb}{0,0,1}
\begin{tikzpicture}[line cap=round,line join=round,>=triangle 45,x=1cm,y=1cm]
\clip(-6.2,-1) rectangle (6.5,4);
\draw [line width=2pt] (-6,0)-- (-2,3);
\draw [line width=2pt] (-2,3)-- (0,0);
\draw [line width=2pt] (0,0)-- (-6,0);
\draw [line width=2pt] (2,0)-- (5,2);
\draw [line width=2pt] (5,2)-- (6,0);
\draw [line width=2pt] (6,0)-- (2,0);
\draw [line width=2pt,domain=-16.260324581654256:15.039675418345755] plot(\x,{(-6-3*\x)/2});
\draw [fill=qqqqff] (-6,0) circle (2.5pt);
\draw[color=qqqqff] (-6,0.42107438016528687) node {$A$};
\draw [fill=qqqqff] (0,0) circle (2pt);
\draw[color=qqqqff] (0.31967541834575075,0.38107438016528694) node {$B$};
\draw [fill=qqqqff] (-2,3) circle (2.5pt);
\draw[color=qqqqff] (-1.6803245816542498,3.421074380165286) node {$C$};
\draw [fill=qqqqff] (2,0) circle (2.5pt);
\draw[color=qqqqff] (2.1,0.42107438016528687) node {$D$};
\draw [fill=qqqqff] (6,0) circle (2.5pt);
\draw[color=qqqqff] (6.319675418345753,0.42107438016528687) node {$E$};
\draw [fill=qqqqff] (5,2) circle (2.5pt);
\draw[color=qqqqff] (5.319675418345753,2.4210743801652863) node {$F$};
\draw [fill=qqqqff] (-2,0) circle (2.5pt);
\draw[color=qqqqff] (-1.6803245816542498,0.42107438016528687) node {$P$};
\draw [fill=qqqqff] (-3.3333333333333335,2) circle (2pt);
\draw[color=qqqqff] (-3.2,2.5) node {$Q$};
\draw [fill=qqqqff] (-4.239057277508364,3.3585859162625447) circle (2.5pt);
\draw[color=qqqqff] (-3.920324581654251,3.6) node {$R$};
\draw [fill=qqqqff] (-0.41444189289297906,-2.3783371606605317) circle (2.5pt);
\draw[color=qqqqff] (-1.2203245816542498,-0.6) node {$m$};
\end{tikzpicture} 
        \end{figure}

        Neste steg er å vise at $\triangle APQ\cong\triangle DEF$. 
        Siden $\triangle APQ\sim \triangle ABC$ kan vi bruke teorem 5.3.1 til å finne at 
        $$\frac{AB}{AP}=\frac{AC}{AQ}=\frac{BC}{PQ}.$$
        Men per konstruksjon har vi $AP=DE$, og per definisjon har vi $r=\frac{AB}{DE}$.
        Den første likheten sier oss dermed at $r=\frac{BC}{PQ}$, som gir $PQ=\frac{BC}{r}=EF$.
        På samme måte kan vi vise at $AQ=DF$, slik at vi kan bruke SSS til å konkludere med at $\triangle APQ\cong \triangle DEF$. 
        Dermed har vi både at $\triangle APQ\cong \triangle DEF$ og $\triangle APQ\sim \triangle ABC$, som tilsammen gir $\triangle ABC\sim\triangle DEF$, som var det vi ville vise. 
        
        \item $\mathbf{r<1:}$ I dette tilfelle kan vi gjøre akkuratt det samme beviset som over, bare med rollene til de to trekantene byttet. 
    \end{enumerate}
\end{oppgave}

\begin{oppgave}[5.4.1]
    La $\triangle ABC$ være en rettvinklet trekant med rett vinkel i $C$. 
    Vi ønsker å vise at høyden i trekanten er det geometriske snittet av lengdene av projeksjonene av katetene. 
    For å forstå hva dette betyr må vi se litt på konstruksjonen i beviset av Pythagoras' teorem, ettersom læreboka bruker denne konstruksjonen for å definere høyde og projeksjone. 
    Vi nedfeller en normal fra punktet $C$ til linjen $\overleftrightarrow{AB}$, og fra lemma 4.8.6 vet vi at denne normalen skjærer $\overleftrightarrow{AB}$ i et punkt $D\in \overline{AB}$. 
    Høyden til trekanten er da definert til å være $h=CD$. 
    Lar vi også $x=AD$ og $y=DB$ kan vi utrykke det vi ønsker å vise som $h=\sqrt{x\cdot y}$. 

    \begin{figure}[H]
        \centering
        
\definecolor{qqwuqq}{rgb}{0,0.39215686274509803,0}
\definecolor{qqqqff}{rgb}{0,0,1}
\begin{tikzpicture}[line cap=round,line join=round,>=triangle 45,x=1cm,y=1cm]
\clip(-3,-0.5) rectangle (5,4);
\draw[line width=2pt,color=qqwuqq,fill=qqwuqq,fill opacity=0.10000000149011612] (-0.23533936216582102,2.6469909567512686) -- (0.11766968108291034,2.4116515945854475) -- (0.35300904324873134,2.764660637834179) -- (0,3) -- cycle; 
\draw [line width=2pt] (-2,0)-- (0,3);
\draw [line width=2pt] (0,3)-- (4.5,0);
\draw [line width=2pt] (4.5,0)-- (-2,0);
\draw [color=qqqqff](0.06,2.26) node[anchor=north west] {$h$};
\draw [line width=2pt] (0,3)-- (0,0);
\draw [fill=qqqqff] (-2,0) circle (2.5pt);
\draw[color=qqqqff] (-2.2,0.43) node {$A$};
\draw [fill=qqqqff] (0,3) circle (2.5pt);
\draw[color=qqqqff] (0.32,3.43) node {$C$};
\draw [fill=qqqqff] (4.5,0) circle (2pt);
\draw[color=qqqqff] (4.82,0.39) node {$B$};
\draw [fill=qqqqff] (0,0) circle (2pt);
\draw[color=qqqqff] (0.32,0.39) node {$D$};
\draw [fill=qqqqff] (-1,-2) circle (2.5pt);
\draw[color=qqqqff] (-0.9,-0.37) node {$x$};
\draw [fill=qqqqff] (2,-2) circle (2.5pt);
\draw[color=qqqqff] (2.06,-0.37) node {$y$};
\end{tikzpicture}
 
    \end{figure}

    Dette følger nå fra tre anvendelser av Pythagoras' teorem. 
    Teoremet anvendt på trekanten $\triangle ABC$ gir at 
    $$AC^2+BC^2=AB^2.$$
    Teoreme anvendt på trekanten $\triangle ADC$ gir 
    $$x^2+h^2=AC^2,$$
    og teoremet anvendt på trekanten $\triangle DBC$ gir 
    $$y^2+h^2=BC^2.$$

    Legger vi sammen de to siste likningene får vi 
    $$x^2+y^2+2h^2=AC^2+BC^2,$$
    som vi fra den øverste likningen vet at er lik $AB^2$. 
    Men, vi vet jo også at $AB=x+y$, så hvis vi setter dette inn i forrige likning får vi 
    \begin{align*}
        2h^2
        &=AB^2-x^2-y^2\\
        &=(x+y)^2-x^2-y^2\\
        &= 2xy.
    \end{align*}
    Løser vi dette for $h$ får vi $h=\sqrt{x\cdot y}$, som var det vi ville vise.  
\end{oppgave}

\begin{oppgave}[5.4.3]
    La $\triangle ABC$ være en trekant slik at $a^2+b^2=c^2$, der $a$ er lengden på siden motsatt av hjørnet $A$, altså $a=BC$, og tilsvarende for $b$ og $c$. 
    Vi vil vise at vinkelen $\angle BCA$ er en rett vinkel. 
    Planen vår er å konstruere en rettvinklet trekant der to av sidene har lengde $a$ og $b$, for å så bruke Pythagoras til å vise at denne nye trekanten er kongruent med den opprinnelige trekanten. 

    Vi begynner med å konstruere en rett vinkel. 
    Plukk to vilkårlige punkt $F$ og $P$. 
    Fra del 3 av gradskivepostulatet kan vi da finne et punkt $Q$ slik at $\mu(\angle PFQ)=90$. 
    Vi bruker så linjalpostulatet til å finne et punkt $E$ på strålen $\overrightarrow{FP}$ slik at $a=FE$, og et punkt $D$ på $\overrightarrow{FQ}$ slik at $b=FD$. 
    Fra Pythagors' teorem anvendt på den rettvinklede trekanten $\triangle DEF$ gir oss $a^2+b^2=DE^2$. 
    Men, vi vet også at $a^2+b^2=c^2$, så vi må ha $c=DE$. 
    Vi vet nå at alle sidene i $\triangle ABC$ og $\triangle DEF$ er like lange, slik at SSS (side-side-side) gir oss at $\triangle ABC\cong \triangle DEF$. 
    Spesielt betyr dette at $\mu(\angle BCA)=\mu(\angle EFD)=90$, som var det vi ville vise. 

    \begin{figure}[H]
        \centering
        
\definecolor{qqwuqq}{rgb}{0,0.39215686274509803,0}
\definecolor{qqqqff}{rgb}{0,0,1}
\begin{tikzpicture}[line cap=round,line join=round,>=triangle 45,x=1cm,y=1cm]
\clip(-5.5,-1.2) rectangle (9,7);
\draw[line width=2pt,color=qqwuqq,fill=qqwuqq,fill opacity=0.10000000149011612] (1.4242640687119286,0) -- (1.4242640687119288,0.42426406871192857) -- (1,0.4242640687119286) -- (1,0) -- cycle; 
\draw [line width=2pt] (-5,0)-- (-5,4);
\draw [line width=2pt] (-1,0)-- (-5,0);
\draw [line width=2pt] (1,0)-- (5,0);
\draw [line width=2pt] (1,4)-- (5,0);
\draw [line width=2pt] (1,0)-- (1,4);
\draw [line width=2pt] (-1,0)-- (-5,4);
\draw [line width=2pt] (1,4) -- (1,11.76);
\draw [line width=2pt,domain=5:16.260000000000005] plot(\x,{(-0-0*\x)/3});
\draw [fill=qqqqff] (-5,0) circle (2.5pt);
\draw[color=qqqqff] (-4.68,0.43) node {$C$};
\draw [fill=qqqqff] (-5,4) circle (2.5pt);
\draw[color=qqqqff] (-4.68,4.43) node {$A$};
\draw[color=black] (-5.32,1.95) node {$b$};
\draw [fill=qqqqff] (-1,0) circle (2.5pt);
\draw[color=qqqqff] (-0.68,0.43) node {$B$};
\draw[color=black] (-3.02,-0.45) node {$a$};
\draw [fill=qqqqff] (1,0) circle (2.5pt);
\draw[color=qqqqff] (0.62,0.43) node {$F$};
\draw [fill=qqqqff] (5,0) circle (2.5pt);
\draw[color=qqqqff] (5.32,0.43) node {$E$};
\draw[color=black] (3.02,-0.39) node {$a$};
\draw [fill=qqqqff] (1,4) circle (2.5pt);
\draw[color=qqqqff] (1.32,4.43) node {$D$};
\draw[color=black] (0.72,1.91) node {$b$};
\draw[color=black] (-2.74,2.09) node {$c$};
\draw [fill=qqqqff] (8,0) circle (2.5pt);
\draw[color=qqqqff] (8.32,0.43) node {$P$};
\draw [fill=qqqqff] (1,6) circle (2.5pt);
\draw[color=qqqqff] (1.32,6.43) node {$Q$};
\end{tikzpicture} 
    \end{figure}
\end{oppgave}